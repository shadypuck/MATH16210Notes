\documentclass[../main.tex]{subfiles}

\pagestyle{main}
\renewcommand{\chaptermark}[1]{\markboth{\chaptername\ \thechapter}{}}
\setcounter{chapter}{5}

\begin{document}




\chapter{Construction of the Real Numbers}\label{sct:6}
\section{Journal}
\begin{definition}\label{dfn:6.1}\marginnote{1/12:}
    A subset $A$ of $\Q$ is said to be a \textbf{cut} (or \textbf{Dedekind cut}) if it satisfies the following:
    \begin{enumerate}[label={(\alph*)},ref={\thedefinition\alph*}]
        \item \label{dfn:6.1a}$A\neq\emptyset$ and $A\neq\Q$.
        \item \label{dfn:6.1b}If $r\in A$ and $s\in\Q$ satisfy $s<r$, then $s\in A$.
        \item \label{dfn:6.1c}$A$ does not have a last point; i.e., if $r\in A$, then there is some $s\in A$ with $s>r$.
    \end{enumerate}
    We denote the collection of all cuts by $\R$.
\end{definition}

\begin{lemma}\label{lem:6.2}
    Let $A$ be a Dedekind cut and $x\in\Q$. Then $x\notin A$ if and only if $x$ is an upper bound for $A$.
    \begin{proof}
        Suppose first that $x\notin A$. To prove that $x$ is an upper bound for $A$, Definition \ref{dfn:5.6} tells us that it will suffice to show that for all $r\in A$, $r\leq x$. Let $r$ be an arbitrary element of $A$. Then by the contrapositive of Definition \ref{dfn:6.1b} and the hypothesis that $x\notin A$, we know that $r\notin A$, $x\notin\Q$, or $x\not<r$. But since $r\in A$ and $x\in\Q$, it must be that $x\not<r$. Therefore, $r\leq x$, as desired.\par
        Now suppose that $x$ is an upper bound for $A$. By Definition \ref{dfn:5.6}, this implies that for all $r\in A$, $r\leq x$. Therefore, since there is no $r\in A$ with $r>x$, by the contrapositive of Definition \ref{dfn:6.1c}, $x\notin A$, as desired.
    \end{proof}
\end{lemma}

\begin{exercise}\label{exr:6.3}\leavevmode
    \begin{enumerate}[label={(\alph*)},ref={\thetheorem\alph*}]
        \item \label{exr:6.3a}Prove that for any $q\in\Q$, $\{x\in\Q\mid x<q\}$ is a Dedekind cut. We then define $\bm{0}=\{x\in\Q\mid x<0\}$.
        \item \label{exr:6.3b}Prove that $\{x\in\Q\mid x\leq 0\}$ is not a Dedekind cut.
        \item \label{exr:6.3c}Prove that $\{x\in\Q\mid x<0\}\cup\{x\in\Q\mid x^2<2\}$ is a Dedekind cut.
    \end{enumerate}
    \begin{proof}[Proof of a]
        Let $q$ be an arbitrary element of $\Q$. To prove that $A=\{x\in\Q\mid x<q\}$ is a Dedekind cut, Definition \ref{dfn:6.1} tells us that it will suffice to show that $A\neq\emptyset$; $A\neq\Q$; if $r\in A$ and $s\in\Q$ satisfy $s<r$, then $s\in A$; and if $r\in A$, then there is some $s\in A$ with $s>r$. We will take this one claim at a time.\par
        To show that $A\neq\emptyset$, Definition \ref{dfn:1.8} tells us that it will suffice to find an element of $A$. By Exercise \ref{exr:3.9d}, $q$ is not the first point of $\Q$. Thus, by Definition \ref{dfn:3.3}, there exists an object $x\in\Q$ such that $x<q$. By the definition of $A$, this implies that $x\in A$, as desired.\par
        To show that $A\neq\Q$, Definition \ref{dfn:1.2} tells us that it will suffice to find an element of $\Q$ that is not an element of $A$. By hypothesis, $q\in\Q$. By Exercise \ref{exr:3.9d}, $q\not<q$. Therefore, $q\in\Q$ but $q\notin A$, as desired.\par
        To show that if $r\in A$ and $s\in\Q$ satisfy $s<r$, then $s\in A$, we let $r\in A$ and $s\in\Q$ be arbitrary elements of their respective sets that satisfy $s<r$ and seek to verify that $s\in A$. Since $r\in A$, $r<q$. This combined with the fact that $s<r$ implies by transitivity that $s<q$. Therefore, since $s\in\Q$ and $s<q$, $s\in A$, as desired.\par
        To show that if $r\in A$, then there is some $s\in A$ with $s>r$, we let $r\in A$ and seek to find such an $s$. By the definition of $A$, $r<q$. Thus, by Additional Exercise \ref{axr:3.1}, there exists a point $s\in\Q$ such that $r<s<q$. Since $s\in\Q$ and $s<q$, $s\in A$. It follows that $s$ is the desired element of $A$ satisfying $s>r$.
    \end{proof}
    \begin{proof}[Proof of b]
        To prove that $A=\{x\in\Q\mid x\leq 0\}$ is not a Dedekind cut, Definition \ref{dfn:6.1} tells us that it will suffice to show that $A$ does have a last point. To show this, we will demonstrate that 0 is the last point of $A$. To demonstrate this, Definition \ref{dfn:3.1} tells us that it will suffice to confirm that $0\in A$ and for all $x\in A$, $x\leq 0$. Since $0\leq 0$ and $0\in\Q$, $0\in A$. Additionally, by the definition of $A$, it is true that for all $x\in A$, $x\leq 0$.
    \end{proof}
    \begin{proof}[Proof of c]
        Let $B=\{x\in\Q\mid x<0\}$ and let $C=\{x\in\Q\mid x^2<2\}$. To prove that $A=B\cup C$ is a Dedekind cut, Definition \ref{dfn:6.1} tells us that it will suffice to show that $A\neq\emptyset$; $A\neq\Q$; if $r\in A$ and $s\in\Q$ satisfy $s<r$, then $s\in A$; and if $r\in A$, then there is some $s\in A$ with $s>r$. We will take this one claim at a time.\par
        To show that $A\neq\emptyset$, Definition \ref{dfn:1.8} tells us that it will suffice to find an element of $A$. Since $-1\in\Q$ and $-1<0$, $-1\in B$. Therefore, by Definition \ref{dfn:1.5}, $-1\in A$, as desired.\par
        To show that $A\neq\Q$, Definition \ref{dfn:1.2} tells us that it will suffice to find an element of $\Q$ that is not an element of $A$. Since $2\in\Q$ and $2\geq 0$, $2\notin B$. Additionally, since $2^2\geq 2$, $2\notin C$. Therefore, by Definition \ref{dfn:1.5}, $2\notin A$, as desired.\par
        To show that if $r\in A$ and $s\in\Q$ satisfy $s<r$, then $s\in A$, we let $r\in A$ and $s\in\Q$ be arbitrary elements of their respective sets that satisfy $s<r$ and seek to verify that $s\in A$. Since $r\in A$, Definition \ref{dfn:1.5} tells us that $r\in B$ or $r\in C$. We now divide into two cases. Suppose first that $r\in B$. Then $s<r<0$, which implies that $s\in B$, meaning that $s\in A$. Now suppose that $r\in C$. We divide into two cases again ($r\leq 0$ and $r>0$). If $r\leq 0$, then $s<r\leq 0$ implies that $s<0$. Thus, by the definition of $B$, $s\in B$, implying that $s\in A$. On the other hand, if $r>0$, then $0<s^2<r^2<2$. Thus, by the definition of $C$, $s\in C$, implying that $s\in A$.\par
        To show that $A$ does not have a last point, suppose for the sake of contradiction that $A$ has a last point $p$. We now divide into two cases ($p\leq 0$ and $p>0$). Suppose first that $p\leq 0$. Since $p$ is the last point of $A$, Definition \ref{dfn:3.3} tells us that $x\leq p$ for all $x\in A$. But $1\in A$ (since $1\in\Q$ and $1^2=1<2$ implies $1\in B$, implies $1\in A$) and $1>0\geq p$, a contradiction. Now suppose that $p>0$. Definition \ref{dfn:3.3} tells us that $p\in A$, but the condition that $p>0$ means $p\notin B$, so we must have $p\in C$. However, by the proof of Exercise \ref{exr:4.24}, $\frac{2(p+1)}{p+2}$ will be an element of $B$ (and therefore $A$) that is greater than $p$ no matter how large $p$ is, a contradiction.
    \end{proof}
\end{exercise}

\begin{definition}\label{dfn:6.4}
    If $A,B\in\R$, we say that $A<B$ if $A$ is a proper subset of $B$.
\end{definition}

\begin{exercise}\label{exr:6.5}
    Show that $\R$ satisfies Axioms \ref{axm:3.1}, \ref{axm:3.2}, and \ref{axm:3.3}.
    \begin{proof}
        By Exercise \ref{exr:6.3a}, $\{x\in\Q\mid x<0\}\in\R$ since $0\in\Q$. Therefore, Axiom \ref{axm:3.1} is immediately satisfied.\par\medskip
        Axiom \ref{axm:3.2} asserts that $\R$ must have an ordering $<$. As such, it will suffice to verify that the ordering given by Definition \ref{dfn:6.4} satisfies the stipulations of Definition \ref{dfn:3.1}. To prove that $<$ satisfies the trichotomy, it will suffice to show that for all $A,B\in\R$, exactly one of the following holds: $A<B$, $B<A$, or $A=B$.\par
        We first show that \emph{no more than one} of the three statements can simultaneously be true. Let $A,B$ be arbitrary elements of $\R$. We divide into three cases. First, suppose for the sake of contradiction that $A<B$ and $B<A$. By Definition \ref{dfn:6.4}, this implies that $A\subsetneq B$ and $B\subsetneq A$. Thus, by Definition \ref{dfn:1.3}, $A\subset B$, $B\subset A$, and $A\neq B$. But by Theorem \ref{trm:1.7}, $A\subset B$ and $B\subset A$ implies that $A=B$, a contradiction. Second, suppose for the sake of contradiction that $A<B$ and $A=B$. By Definition \ref{dfn:6.4}, the former statement implies that $A\subsetneq B$. Thus, by Definition \ref{dfn:1.3}, $A\neq B$, a contradiction. The proof of the third case ($B<A$ and $A=B$) is symmetric to that of the second case.\par
        We now show that \emph{at least one} of the three statements is always true. Let $A,B$ be arbitrary elements of $\R$, and suppose for the sake of contradiction that $A\not<B$, $B\not<A$, and $A\neq B$. Since $A\not<B$ and $B\not<A$, we have by Definition \ref{dfn:6.4} that $A\not\subsetneq B$ and $B\not\subsetneq A$. Thus, by Definition \ref{dfn:1.3}, $A\not\subset B$ or $A=B$, and $B\not\subset A$ or $A=B$. But $A\neq B$ by hypothesis, so it must be that $A\not\subset B$ and $B\not\subset A$. It follows from the first statement by Definition \ref{dfn:1.3} that there exists an object $x\in A$ such that $x\notin B$, and there exists an object $y\in B$ such that $y\notin A$. Since $x\notin B$, Lemma \ref{lem:6.2} implies that $x$ is an upper bound of $B$. Consequently, by Definition \ref{dfn:5.6}, $p\leq x$ for all $p\in B$, including $y$. Similarly, $p\leq y$ for all $p\in A$, including $x$. Thus, we have $y\leq x$ and $x\leq y$, implying that $x=y$. But since $y\in B$, this implies that $x\in B$, a contradiction.\par\smallskip
        To prove that $<$ is transitive, it will suffice to show that for all $A,B,C\in\R$, if $A<B$ and $B<C$, then $A<C$. Let $A,B,C$ be arbitrary elements of $\R$ for which it is true that $A<B$ and $B<C$. By Definition \ref{dfn:6.4}, we have $A\subsetneq B$ and $B\subsetneq C$. Thus, by Script \ref{sct:1}, $A\subsetneq C$. Therefore, by Definition \ref{dfn:6.4}, $A<C$.\par\medskip
        Axiom \ref{axm:3.3} asserts that $\R$ must have no first or last point. We will take this one argument at a time\par
        Suppose for the sake of contradiction that $\R$ has some first point $A$. Then by Definition \ref{dfn:3.3}, $A\leq X$ for every $X\in\R$. Now since $A$ is a Dedekind cut, Definition \ref{dfn:6.1} tells us that $A\neq\emptyset$. Thus, by Definition \ref{dfn:1.8}, there exists some $q\in A$. Additionally, $A\subset\Q$ by Definition \ref{dfn:6.1}, so $q\in A$ implies that $q\in\Q$. It follows by Exercise \ref{exr:6.3a} that $B=\{x\in\Q\mid x<q\}$ is a Dedekind cut. We now seek to prove that $B\subsetneq A$. To do this, Definition \ref{dfn:1.3} tells us that it will suffice to show that $B\neq A$ and $B\subset A$. To show that $B\neq A$, Definition \ref{dfn:1.2} tells us that it will suffice to find an element of $A$ that is not an element of $B$. Conveniently, $q$ is clearly such an object. To show that $B\subset A$, Definition \ref{dfn:1.3} tells us that we must confirm that every element of $B$ is an element of $A$. Let $p$ be an arbitrary element of $B$. Then by the definition of $B$, $p\in\Q$ and $p<q$. It follows by Definition \ref{dfn:6.1b} (which clearly applies to $A$) that $p\in A$, as desired. Having proven that $B\subsetneq A$, Definition \ref{dfn:6.4} tells us that $B<A$. But this contradicts the previously demonstrated fact that $A\leq X$ for every $X\in\R$, including $B$.\par
        Suppose for the sake of contradiction that $\R$ has some last point $A$. Then by Definition \ref{dfn:3.3}, $X\leq A$ for every $X\in\R$. Now since $A$ is a Dedekind cut, Definition \ref{dfn:6.1} tells us that $A\neq\Q$. Thus, by Definition \ref{dfn:1.2}, there exists some $q\in\Q$ such that $q\notin A$. It follows by Lemma \ref{lem:6.2} that $q$ is an upper bound of $A$. Consequently, by Definition \ref{dfn:5.6}, $x\leq q$ for all $x\in A$. Additionally, by Exercise \ref{exr:6.3a}, $B=\{x\in\Q\mid x<q+1\}$\footnote{Note that we add 1 to $q$ to treat the case that $q=\sup A$, a case in which we would have $B=A$ if $B$ were defined as $\{x\in\Q\mid x<q\}$.} is a Dedekind cut. We now seek to prove that $A\subsetneq B$. As before, this means we must show that $A\neq B$ and $A\subset B$. To show that $A\neq B$, Definition \ref{dfn:1.2} tells us that it will suffice to find an element of $B$ that is not an element of $A$. Since $x\leq q$ for all $x\in A$ and $q<q+0.5<q+1$, $q+0.5\notin A$ and $q+0.5\in B$ is the desired object. To show that $A\subset B$, Definition \ref{dfn:1.3} tells us that we must confirm that every element of $A$ is an element of $B$. Let $p$ be an arbitrary element of $A$. As an element of $A$, we know that $p\leq q$. Thus, $p<q+1$, so $p\in B$, as desired. Having proven that $A\subsetneq B$, Definition \ref{dfn:6.4} tells us that $A<B$. But this contradicts the previously demonstrated fact that $X\leq A$ for every $X\in\R$, including $B$.
    \end{proof}
\end{exercise}




\end{document}