\documentclass[../main.tex]{subfiles}

\pagestyle{main}
\renewcommand{\chaptermark}[1]{\markboth{\chaptername\ \thechapter}{}}
\setcounter{chapter}{5}
\setcounter{axiom}{4}

\begin{document}




\chapter{Construction of the Real Numbers}\label{sct:6}
\section{Journal}
\begin{definition}\label{dfn:6.1}\marginnote{1/12:}
    A subset $A$ of $\Q$ is said to be a \textbf{cut} (or \textbf{Dedekind cut}) if it satisfies the following:
    \begin{enumerate}[label={(\alph*)},ref={\thedefinition\alph*}]
        \item \label{dfn:6.1a}$A\neq\emptyset$ and $A\neq\Q$.
        \item \label{dfn:6.1b}If $r\in A$ and $s\in\Q$ satisfy $s<r$, then $s\in A$.
        \item \label{dfn:6.1c}$A$ does not have a last point; i.e., if $r\in A$, then there is some $s\in A$ with $s>r$.
    \end{enumerate}
    We denote the collection of all cuts by $\R$.
\end{definition}

\begin{lemma}\label{lem:6.2}
    Let $A$ be a Dedekind cut and $x\in\Q$. Then $x\notin A$ if and only if $x$ is an upper bound for $A$.
    \begin{proof}
        Suppose first that $x\notin A$. To prove that $x$ is an upper bound for $A$, Definition \ref{dfn:5.6} tells us that it will suffice to show that for all $r\in A$, $r\leq x$. Let $r$ be an arbitrary element of $A$. Then by the contrapositive of Definition \ref{dfn:6.1b} and the hypothesis that $x\notin A$, we know that $r\notin A$, $x\notin\Q$, or $x\not<r$. But since $r\in A$ and $x\in\Q$, it must be that $x\not<r$. Therefore, $r\leq x$, as desired.\par
        Now suppose that $x$ is an upper bound for $A$. By Definition \ref{dfn:5.6}, this implies that for all $r\in A$, $r\leq x$. Therefore, since there is no $r\in A$ with $r>x$, by the contrapositive of Definition \ref{dfn:6.1c}, $x\notin A$, as desired.
    \end{proof}
\end{lemma}

\begin{exercise}\label{exr:6.3}\leavevmode
    \begin{enumerate}[label={(\alph*)},ref={\thetheorem\alph*}]
        \item \label{exr:6.3a}Prove that for any $q\in\Q$, $\{x\in\Q\mid x<q\}$ is a Dedekind cut. We then define $\bm{0}=\{x\in\Q\mid x<0\}$.
        \item \label{exr:6.3b}Prove that $\{x\in\Q\mid x\leq 0\}$ is not a Dedekind cut.
        \item \label{exr:6.3c}Prove that $\{x\in\Q\mid x<0\}\cup\{x\in\Q\mid x^2<2\}$ is a Dedekind cut.
    \end{enumerate}
    \begin{proof}[Proof of a]
        Let $q$ be an arbitrary element of $\Q$. To prove that $A=\{x\in\Q\mid x<q\}$ is a Dedekind cut, Definition \ref{dfn:6.1} tells us that it will suffice to show that $A\neq\emptyset$; $A\neq\Q$; if $r\in A$ and $s\in\Q$ satisfy $s<r$, then $s\in A$; and if $r\in A$, then there is some $s\in A$ with $s>r$. We will take this one claim at a time.\par
        To show that $A\neq\emptyset$, Definition \ref{dfn:1.8} tells us that it will suffice to find an element of $A$. By Exercise \ref{exr:3.9d}, $q$ is not the first point of $\Q$. Thus, by Definition \ref{dfn:3.3}, there exists an object $x\in\Q$ such that $x<q$. By the definition of $A$, this implies that $x\in A$, as desired.\par
        To show that $A\neq\Q$, Definition \ref{dfn:1.2} tells us that it will suffice to find an element of $\Q$ that is not an element of $A$. By hypothesis, $q\in\Q$. By Exercise \ref{exr:3.9d}, $q\not<q$. Therefore, $q\in\Q$ but $q\notin A$, as desired.\par
        To show that if $r\in A$ and $s\in\Q$ satisfy $s<r$, then $s\in A$, we let $r\in A$ and $s\in\Q$ be arbitrary elements of their respective sets that satisfy $s<r$ and seek to verify that $s\in A$. Since $r\in A$, $r<q$. This combined with the fact that $s<r$ implies by transitivity that $s<q$. Therefore, since $s\in\Q$ and $s<q$, $s\in A$, as desired.\par
        To show that if $r\in A$, then there is some $s\in A$ with $s>r$, we let $r\in A$ and seek to find such an $s$. By the definition of $A$, $r<q$. Thus, by Additional Exercise \ref{axr:3.1}, there exists a point $s\in\Q$ such that $r<s<q$. Since $s\in\Q$ and $s<q$, $s\in A$. It follows that $s$ is the desired element of $A$ satisfying $s>r$.
    \end{proof}
    \begin{proof}[Proof of b]
        To prove that $A=\{x\in\Q\mid x\leq 0\}$ is not a Dedekind cut, Definition \ref{dfn:6.1} tells us that it will suffice to show that $A$ does have a last point. To show this, we will demonstrate that 0 is the last point of $A$. To demonstrate this, Definition \ref{dfn:3.1} tells us that it will suffice to confirm that $0\in A$ and for all $x\in A$, $x\leq 0$. Since $0\leq 0$ and $0\in\Q$, $0\in A$. Additionally, by the definition of $A$, it is true that for all $x\in A$, $x\leq 0$.
    \end{proof}
    \begin{proof}[Proof of c]
        Let $B=\{x\in\Q\mid x<0\}$ and let $C=\{x\in\Q\mid x^2<2\}$. To prove that $A=B\cup C$ is a Dedekind cut, Definition \ref{dfn:6.1} tells us that it will suffice to show that $A\neq\emptyset$; $A\neq\Q$; if $r\in A$ and $s\in\Q$ satisfy $s<r$, then $s\in A$; and if $r\in A$, then there is some $s\in A$ with $s>r$. We will take this one claim at a time.\par
        To show that $A\neq\emptyset$, Definition \ref{dfn:1.8} tells us that it will suffice to find an element of $A$. Since $-1\in\Q$ and $-1<0$, $-1\in B$. Therefore, by Definition \ref{dfn:1.5}, $-1\in A$, as desired.\par
        To show that $A\neq\Q$, Definition \ref{dfn:1.2} tells us that it will suffice to find an element of $\Q$ that is not an element of $A$. Since $2\in\Q$ and $2\geq 0$, $2\notin B$. Additionally, since $2^2\geq 2$, $2\notin C$. Therefore, by Definition \ref{dfn:1.5}, $2\notin A$, as desired.\par
        To show that if $r\in A$ and $s\in\Q$ satisfy $s<r$, then $s\in A$, we let $r\in A$ and $s\in\Q$ be arbitrary elements of their respective sets that satisfy $s<r$ and seek to verify that $s\in A$. Since $r\in A$, Definition \ref{dfn:1.5} tells us that $r\in B$ or $r\in C$. We now divide into two cases. Suppose first that $r\in B$. Then $s<r<0$, which implies that $s\in B$, meaning that $s\in A$. Now suppose that $r\in C$. We divide into two cases again ($r\leq 0$ and $r>0$). If $r\leq 0$, then $s<r\leq 0$ implies that $s<0$. Thus, by the definition of $B$, $s\in B$, implying that $s\in A$. On the other hand, if $r>0$, then $0<s^2<r^2<2$. Thus, by the definition of $C$, $s\in C$, implying that $s\in A$.\par
        To show that $A$ does not have a last point, suppose for the sake of contradiction that $A$ has a last point $p$. We now divide into two cases ($p\leq 0$ and $p>0$). Suppose first that $p\leq 0$. Since $p$ is the last point of $A$, Definition \ref{dfn:3.3} tells us that $x\leq p$ for all $x\in A$. But $1\in A$ (since $1\in\Q$ and $1^2=1<2$ implies $1\in B$, implies $1\in A$) and $1>0\geq p$, a contradiction. Now suppose that $p>0$. Definition \ref{dfn:3.3} tells us that $p\in A$, but the condition that $p>0$ means $p\notin B$, so we must have $p\in C$. However, by the proof of Exercise \ref{exr:4.24}, $\frac{2(p+1)}{p+2}$ will be an element of $B$ (and therefore $A$) that is greater than $p$ no matter how large $p$ is, a contradiction.
    \end{proof}
\end{exercise}

\begin{definition}\label{dfn:6.4}
    If $A,B\in\R$, we say that $A<B$ if $A$ is a proper subset of $B$.
\end{definition}

\begin{exercise}\label{exr:6.5}
    Show that $\R$ satisfies Axioms \ref{axm:3.1}, \ref{axm:3.2}, and \ref{axm:3.3}.
    \begin{proof}
        By Exercise \ref{exr:6.3a}, $\{x\in\Q\mid x<0\}\in\R$ since $0\in\Q$. Therefore, Axiom \ref{axm:3.1} is immediately satisfied.\par\medskip
        Axiom \ref{axm:3.2} asserts that $\R$ must have an ordering $<$. As such, it will suffice to verify that the ordering given by Definition \ref{dfn:6.4} satisfies the stipulations of Definition \ref{dfn:3.1}. To prove that $<$ satisfies the trichotomy, it will suffice to show that for all $A,B\in\R$, exactly one of the following holds: $A<B$, $B<A$, or $A=B$.\par
        We first show that \emph{no more than one} of the three statements can simultaneously be true. Let $A,B$ be arbitrary elements of $\R$. We divide into three cases. First, suppose for the sake of contradiction that $A<B$ and $B<A$. By Definition \ref{dfn:6.4}, this implies that $A\subsetneq B$ and $B\subsetneq A$. Thus, by Definition \ref{dfn:1.3}, $A\subset B$, $B\subset A$, and $A\neq B$. But by Theorem \ref{trm:1.7}, $A\subset B$ and $B\subset A$ implies that $A=B$, a contradiction. Second, suppose for the sake of contradiction that $A<B$ and $A=B$. By Definition \ref{dfn:6.4}, the former statement implies that $A\subsetneq B$. Thus, by Definition \ref{dfn:1.3}, $A\neq B$, a contradiction. The proof of the third case ($B<A$ and $A=B$) is symmetric to that of the second case.\par
        We now show that \emph{at least one} of the three statements is always true. Let $A,B$ be arbitrary elements of $\R$, and suppose for the sake of contradiction that $A\not<B$, $B\not<A$, and $A\neq B$. Since $A\not<B$ and $B\not<A$, we have by Definition \ref{dfn:6.4} that $A\not\subsetneq B$ and $B\not\subsetneq A$. Thus, by Definition \ref{dfn:1.3}, $A\not\subset B$ or $A=B$, and $B\not\subset A$ or $A=B$. But $A\neq B$ by hypothesis, so it must be that $A\not\subset B$ and $B\not\subset A$. It follows from the first statement by Definition \ref{dfn:1.3} that there exists an object $x\in A$ such that $x\notin B$, and there exists an object $y\in B$ such that $y\notin A$. Since $x\notin B$, Lemma \ref{lem:6.2} implies that $x$ is an upper bound of $B$. Consequently, by Definition \ref{dfn:5.6}, $p\leq x$ for all $p\in B$, including $y$. Similarly, $p\leq y$ for all $p\in A$, including $x$. Thus, we have $y\leq x$ and $x\leq y$, implying that $x=y$. But since $y\in B$, this implies that $x\in B$, a contradiction.\par\smallskip
        To prove that $<$ is transitive, it will suffice to show that for all $A,B,C\in\R$, if $A<B$ and $B<C$, then $A<C$. Let $A,B,C$ be arbitrary elements of $\R$ for which it is true that $A<B$ and $B<C$. By Definition \ref{dfn:6.4}, we have $A\subsetneq B$ and $B\subsetneq C$. Thus, by Script \ref{sct:1}, $A\subsetneq C$. Therefore, by Definition \ref{dfn:6.4}, $A<C$.\par\medskip
        Axiom \ref{axm:3.3} asserts that $\R$ must have no first or last point. We will take this one argument at a time\par
        Suppose for the sake of contradiction that $\R$ has some first point $A$. Then by Definition \ref{dfn:3.3}, $A\leq X$ for every $X\in\R$. Now since $A$ is a Dedekind cut, Definition \ref{dfn:6.1} tells us that $A\neq\emptyset$. Thus, by Definition \ref{dfn:1.8}, there exists some $q\in A$. Additionally, $A\subset\Q$ by Definition \ref{dfn:6.1}, so $q\in A$ implies that $q\in\Q$. It follows by Exercise \ref{exr:6.3a} that $B=\{x\in\Q\mid x<q\}$ is a Dedekind cut. We now seek to prove that $B\subsetneq A$. To do this, Definition \ref{dfn:1.3} tells us that it will suffice to show that $B\neq A$ and $B\subset A$. To show that $B\neq A$, Definition \ref{dfn:1.2} tells us that it will suffice to find an element of $A$ that is not an element of $B$. Conveniently, $q$ is clearly such an object. To show that $B\subset A$, Definition \ref{dfn:1.3} tells us that we must confirm that every element of $B$ is an element of $A$. Let $p$ be an arbitrary element of $B$. Then by the definition of $B$, $p\in\Q$ and $p<q$. It follows by Definition \ref{dfn:6.1b} (which clearly applies to $A$) that $p\in A$, as desired. Having proven that $B\subsetneq A$, Definition \ref{dfn:6.4} tells us that $B<A$. But this contradicts the previously demonstrated fact that $A\leq X$ for every $X\in\R$, including $B$.\par
        Suppose for the sake of contradiction that $\R$ has some last point $A$. Then by Definition \ref{dfn:3.3}, $X\leq A$ for every $X\in\R$. Now since $A$ is a Dedekind cut, Definition \ref{dfn:6.1} tells us that $A\neq\Q$. Thus, by Definition \ref{dfn:1.2}, there exists some $q\in\Q$ such that $q\notin A$. It follows by Lemma \ref{lem:6.2} that $q$ is an upper bound of $A$. Consequently, by Definition \ref{dfn:5.6}, $x\leq q$ for all $x\in A$. Additionally, by Exercise \ref{exr:6.3a}, $B=\{x\in\Q\mid x<q+1\}$\footnote{Note that we add 1 to $q$ to treat the case that $q=\sup A$, a case in which we would have $B=A$ if $B$ were defined as $\{x\in\Q\mid x<q\}$.} is a Dedekind cut. We now seek to prove that $A\subsetneq B$. As before, this means we must show that $A\neq B$ and $A\subset B$. To show that $A\neq B$, Definition \ref{dfn:1.2} tells us that it will suffice to find an element of $B$ that is not an element of $A$. Since $x\leq q$ for all $x\in A$ and $q<q+0.5<q+1$, $q+0.5\notin A$ and $q+0.5\in B$ is the desired object. To show that $A\subset B$, Definition \ref{dfn:1.3} tells us that we must confirm that every element of $A$ is an element of $B$. Let $p$ be an arbitrary element of $A$. As an element of $A$, we know that $p\leq q$. Thus, $p<q+1$, so $p\in B$, as desired. Having proven that $A\subsetneq B$, Definition \ref{dfn:6.4} tells us that $A<B$. But this contradicts the previously demonstrated fact that $X\leq A$ for every $X\in\R$, including $B$.
    \end{proof}
\end{exercise}

\begin{lemma}\label{lem:6.6}\marginnote{\emph{1/14:}}
    A nonempty subset of $\R$ that is bounded above has a supremum.
    \begin{proof}
        Let $X$ be an arbitrary nonempty subset of $\R$ that is bounded above. To prove that $\sup X$ exists, we will show that $\sup X=U=\bigcup\{Y\mid Y\in X\}$. To show this, Definition \ref{dfn:5.7} tells us that it will suffice to demonstrate that $U\in\R$, $U$ is an upper bound of $X$, and if $U'$ is an upper bound of $X$, then $U\leq U'$. Let's begin.\par\smallskip
        To demonstrate that $U\in\R$, Definition \ref{dfn:6.1} tells us that it will suffice to confirm that $U\neq\emptyset$; $U\neq\Q$; if $r\in U$ and $s\in\Q$ satisfy $s<r$, then $s\in U$; and if $r\in U$, then there is some $s\in U$ with $s>r$.\par
        As the union of a nonempty subset of nonempty sets, Script \ref{sct:1} implies that $U\neq\emptyset$.\par
        To demonstrate that $U\neq\Q$, Definition \ref{dfn:1.2} tells us that it will suffice to find a point $p\in\Q$ such that $p\notin U$. Since $X$ is bounded above, we have by Definition \ref{dfn:5.6} that there exists a Dedekind cut $V\in\R$ such that $Y\leq V$ for all $Y\in X$. It follows by Definition \ref{dfn:6.4} that $Y\subset V$ for all $Y\in X$. Thus, by Script \ref{sct:1}, $U\subset V$. Now since $V$ is a Dedekind cut, we know by Definition \ref{dfn:6.1} that $V\subset\Q$ and $V\neq\Q$, meaning that there exists a point $p\in\Q$ such that $p\notin V$. Consequently, since $U\subset V$, $p\notin U$, as desired.\par
        To demonstrate that if $r\in U$ and $s\in\Q$ satisfy $s<r$, then $s\in U$, we let $r\in U$ and $s\in\Q$ be arbitrary elements of their respective sets that satisfy $s<r$ and seek to verify that $s\in U$. Since $r\in U$, Definition \ref{dfn:1.13} tells us that $r\in Y$ for some $Y\in X$. Thus, since $Y$ is a Dedekind cut, $s\in\Q$ and $s<r$ implies that $s\in Y$. Therefore, $s\in U$.\par
        To demonstrate that if $r\in U$, then there is some $s\in U$ with $s>r$, we let $r\in U$ and seek to find such an $s$. Since $r\in U$, Definition \ref{dfn:1.13} tells us that $r\in Y$ for some $Y\in X$. Thus, since $Y$ is a Dedekind cut, there exists a point $s\in Y$ with $s>r$. Therefore, $s\in U$.\par\smallskip
        To demonstrate that $U$ is an upper bound of $X$, Definition \ref{dfn:5.6} tells us that it will suffice to confirm that $Y\leq U$ for all $Y\in X$. To confirm this, Definition \ref{dfn:6.4} tells us that it will suffice to verify that $Y\subset U$ for all $Y\in X$. But by an extension of Theorem \ref{trm:1.7b}, this is true.\par\smallskip
        Now suppose for the sake of contradiction that there exists an upper bound $U'$ of $X$ such that $U'<U$. It follows by Definitions \ref{dfn:6.4} and \ref{dfn:1.3} that there exists a point $p\in U$ such that $p\notin U'$. Thus, by the former statement and Definition \ref{dfn:1.13}, $p\in Y$ for some $Y\in X$. Additionally, since $U'$ is an upper bound of $X$, we have by Definitions \ref{dfn:5.6} and \ref{dfn:6.4} that $Y\subset U'$ for all $Y\in X$. But this implies by Definition \ref{dfn:1.3} that $p\in U'$, a contradiction.
    \end{proof}
\end{lemma}

\begin{exercise}\label{exr:6.7}\marginnote{1/19:}
    Show that $\R$ satisfies Axiom \ref{axm:5.4}.
    \begin{proof}
        Suppose for the sake of contradiction that $\R$ does not satisfy Axiom \ref{axm:5.4}. It follows that $\R$ is not connected, implying by Definition \ref{dfn:4.22} that $\R=A\cup B$ where $A,B$ are disjoint, nonempty, open sets. Since $A,B$ are disjoint and nonempty, we know that there exist distinct objects $a\in A$ and $b\in B$. WLOG, let $a<b$.\par
        We now seek to prove that the set $A\cap\underline{ab}$ is nonempty and bounded above. To prove that $A\cap\underline{ab}$ is nonempty, Definition \ref{dfn:1.8} tells us that it will suffice to find an element of $A\cap\underline{ab}$. Since $a\in A$ and $A$ is open, we have by Theorem \ref{trm:4.10} that there exists a region $\underline{cd}$ such that $a\in\underline{cd}$ and $\underline{cd}\subset A$. It follows by Definitions \ref{dfn:3.10} and \ref{dfn:3.6} that $a<d$, implying by Lemma \ref{lem:6.10}\footnote{We may use this lemma since it does not depend on this result, Definition \ref{dfn:6.8}, or Lemma \ref{lem:6.9}.} that there exists some point $x\in\R$ such that $c<a<x<d<b$ (note that $d<b$ since if $b<d$, then $b\in\underline{cd}$ would contradict the fact that $\underline{cd}\subset A$). Consequently, $x\in\underline{cd}$, meaning that $x\in A$, and $x\in\underline{ab}$. Therefore, $x\in A\cap\underline{ab}$, as desired. To prove that $A\cap\underline{ab}$ is bounded above, Definition \ref{dfn:5.6} tells us that it will suffice to show that $b$ is an upper bound of $A\cap\underline{ab}$. To show this, Definition \ref{dfn:5.6} tells us that it will suffice to confirm that $y\leq b$ for all $y\in A\cap\underline{ab}$. Let $y$ be an arbitrary element of $A\cap\underline{ab}$. Then by Definition \ref{dfn:1.6}, $y\in A$ and $y\in\underline{ab}$. It follows from the latter statement by Definitions \ref{dfn:3.10} and \ref{dfn:3.6} that $y<b$, i.e., $y\leq b$, as desired.\par
        Having established that $A\cap\underline{ab}\subset\R$ is nonempty and bounded above, we can invoke Lemma \ref{lem:6.6} to learn that $A\cap\underline{ab}$ has a supremum $\sup(A\cap\underline{ab})$. We now divide into two cases ($\sup(A\cap\underline{ab})\in A$ and $\sup(A\cap\underline{ab})\in B$; it follows from the definitions of $A$ and $B$ that exactly one of these cases is true). Suppose first that $\sup(A\cap\underline{ab})\in A$. Then since $A$ is open, we have by Theorem \ref{trm:4.10} that there exists a region $\underline{ef}$ such that $\sup(A\cap\underline{ab})\in\underline{ef}$ and $\underline{ef}\subset A$. It follows from the former condition that $\sup(A\cap\underline{ab})<f$. Thus, by Lemma \ref{lem:6.10}, there exists an object $z\in\R$ such that $e<\sup(A\cap\underline{ab})<z<f<b$ (note that $f<b$ for the same reason that $d<b$). Consequently, $z\in\underline{ef}$, implying that $z\in A$, and $z\in\underline{ab}$. Thus, we have found an element of $A\cap\underline{ab}$ that is greater than $\sup(A\cap\underline{ab})$, contradicting Definitions \ref{dfn:5.7} and \ref{dfn:5.6}. The proof is symmetric in the other case.
    \end{proof}
\end{exercise}

\begin{definition}\label{dfn:6.8}\marginnote{1/14:}
    Let $C$ be a continuum satisfying Axioms 1-4. Consider a subset $X\subset C$. We say that $X$ is \textbf{dense} in $C$ if every $p\in C$ is a limit point of $X$.
\end{definition}

\begin{lemma}\label{lem:6.9}
    A subset $X\subset C$ is dense in $C$ if and only if $\overline{X}=C$.
    \begin{proof}
        Suppose first that $X\subset C$ is dense in $C$. To prove that $\overline{X}=C$, Definition \ref{dfn:1.2} tells us that it will suffice to show that every point $p\in\overline{X}$ is an element of $C$ and vice versa. Clearly, every element of $\overline{X}$ is an element of $C$. On the other hand, let $p$ be an arbitrary element of $C$. Since $X$ is dense in $C$, Definition \ref{dfn:6.8} tells us that $p\in LP(X)$. Therefore, by Definitions \ref{dfn:1.5} and \ref{dfn:4.4}, $p\in\overline{X}$.\par
        Now suppose that $\overline{X}=C$. To prove that $X$ is dense in $C$, Definition \ref{dfn:6.8} tells us that it will suffice to show that every $p\in C$ is a limit point of $X$. Let $p$ be an arbitrary element of $C$. By Corollary \ref{cly:5.4}, this implies that $p\in LP(C)$. It follows that $p\in LP(\overline{X})$. Thus, by Definition \ref{dfn:4.4}, $p\in LP(X\cup LP(X))$. Consequently, by Theorem \ref{trm:3.20}, $p\in LP(X)$ or $p\in LP(LP(X))$. We now divide into two cases. If $p\in LP(X)$, then we are done. On the other hand, if $p\in LP(LP(X))$, the lemma from Theorem \ref{trm:4.6} asserts that $p\in LP(X)$, and we are done again.
    \end{proof}
\end{lemma}

Our next goal is to prove that $\Q$ is dense in $\R$. Just to make sense of that statement, we need to decide how to think of $\Q$ as a subset of $\R$. For every rational number $q\in\Q$, define the corresponding real number as the Dedekind cut
\begin{equation*}
    i(q) = \{x\in\Q\mid x<q\}
\end{equation*}
For example, $\bm{0}=i(0)$. It can be verified that this gives a well-defined injective function $i:\Q\to\R$. We identify $\Q$ with its image $i(\Q)\subset\R$ so that the rational numbers $\Q$ are a subset of the real numbers $\R$. (Similarly, $\N$ and $\Z$ can be understood as subsets of $\R$.)

\begin{lemma}\label{lem:6.10}
    Given $A,B\in\R$ with $A<B$, there exists $p\in\Q$ such that $A<i(p)<B$.
    \begin{proof}
        Since $A<B$, Definition \ref{dfn:6.4} tells us that $A\subsetneq B$. Thus, by Definition \ref{dfn:1.3}, there exists a point $q$ such that $q\in B$ and $q\notin A$. Since $q\in B$ where $B$ is a Dedekind cut, we have by Definition \ref{dfn:6.1} that there exists a point $p\in B$ with $p>q$. Additionally, since $q\notin A$ implies that $q$ is an upper bound of $A$ by Lemma \ref{lem:6.2}, we know by Definition \ref{dfn:5.6} that $x\leq q$ for all $x\in A$. It follows since $q<p$ that $x\leq p$ for all $x\in A$, meaning by Definition \ref{dfn:5.6} and Lemma \ref{lem:6.2} that $p\notin A$. Having established that $p,q\in B$, $p,q\notin A$, and $q<p$, we are now ready to prove that $A<i(p)<B$. Definition \ref{dfn:6.4} tells us that we may do so by showing that $A\subsetneq i(p)$ and $i(p)\subsetneq B$. We will take this one argument at a time.\par
        To show that $A\subsetneq i(p)$, Definition \ref{dfn:1.3} tells us that it will suffice to verify that every element of $A$ is an element of $i(p)$ and that there exists an element of $i(p)$ that is not an element of $A$. We treat the former statement first. As previously mentioned, $x\leq p$ for all $x\in A$. This combined with the fact that $p\notin A$ implies that $x<p$ for all $x\in A$. Thus, by the definition of $i(p)$, $x\in i(p)$ for all $x\in A$, as desired. As to the latter statement, since $q<p$, we have by the definition of $i(p)$ that $q\in i(p)$. However, we also know that $q\notin A$, as desired.\par
        To show that $i(p)\subsetneq B$, we must verify symmetric arguments to before. For the former statement, let $r$ be an arbitrary element of $i(p)$. Then by the definition of $i(p)$, $r<p$. Since $p\in B$ and $r\in\Q$ satisfy $r<p$, we have by Definition \ref{dfn:6.1} that $r\in B$, as desired. As to the latter statement, $p$ is clearly an element of $B$ that is not an element of $i(p)$, as desired.
    \end{proof}
\end{lemma}

\begin{theorem}\label{trm:6.11}\marginnote{\emph{1/19:}}
    $i(\Q)$ is dense in $\R$.
    \begin{proof}
        To prove that $i(\Q)$ is dense in $\R$, Definition \ref{dfn:6.8} tells us that it will suffice to show the every point $X\in\R$ is a limit point of $i(\Q)$. Let $X$ be an arbitrary element of $\R$. To show that $X\in LP(i(\Q))$, Definition \ref{dfn:3.13} tells us that it will suffice to verify that for every region $\underline{AB}$ with $X\in\underline{AB}$, we have $\underline{AB}\cap(i(\Q)\setminus\{X\})\neq\emptyset$. Let $\underline{AB}$ be an arbitrary region with $X\in\underline{AB}$. It follows by Definitions \ref{dfn:3.10} and \ref{dfn:3.6} that $A<X<B$. Thus, by Lemma \ref{lem:6.10}, there exists $p\in\Q$ such that $A<i(p)<X<B$. By Definitions \ref{dfn:3.6} and \ref{dfn:3.10}, $i(p)\in\underline{AB}$. By Definition \ref{dfn:1.18}, $i(p)\in i(\Q)$. By Exercise \ref{exr:6.5}, $i(p)<X$ implies that $i(p)\neq X$. Combining the last three results with Definitions \ref{dfn:1.11} and \ref{dfn:1.6}, we have that $i(p)\in\underline{AB}\cap(i(\Q)\setminus\{X\})$, as desired.
    \end{proof}
\end{theorem}

\begin{corollary}[The Archimedean Property]\label{cly:6.12}
    Let $A\in\R$ be a positive real number. Then there exist nonzero natural numbers $n,m\in\N$ such that $i(\frac{1}{n})<A<i(m)$.
    \begin{proof}
        We will first prove that there exists a nonzero natural number $n$ such that $i(\frac{1}{n})<A$. We will then prove that there exists a nonzero natural number $m$ such that $A<i(m)$. Let's begin.\par
        Since $A\in\R$ is positive, we know that $0<A$. Thus, by Lemma \ref{lem:6.10}, there exists $\frac{p}{n}\in\Q$ such that $0<i(\frac{p}{n})<A$. As permitted by Exercise \ref{exr:3.9b}, we choose $\frac{p}{n}\in\eqclass{p}{n}$ to be an object such that $0<n$ (this means that $n\in\N$). Consequently, by Scripts \ref{sct:2} and \ref{sct:3}, we know that $0<\frac{1}{n}\leq\frac{p}{n}$. It follows that $i(\frac{1}{n})\leq i(\frac{p}{n})$ since $x\in i(\frac{1}{n})$ implies $x<\frac{1}{n}\leq\frac{p}{n}$ implies $x\in i(\frac{p}{n})$, implies $i(\frac{1}{n})\subset i(\frac{p}{n})$. Therefore, $i(\frac{1}{n})\leq i(\frac{p}{n})<A$, as desired.\par
        By Exercise \ref{exr:6.5}, Axiom \ref{axm:3.3}, and Definition \ref{dfn:3.3}, there exists a point $B\in\R$ such that $A<B$. It follows by Lemma \ref{lem:6.10} that there exists $\frac{m}{q}\in\Q$ such that $A<i(\frac{m}{q})<B$. As before, let $\frac{m}{q}$ be an object such that $0<q$. Consequently, by Scripts \ref{sct:2} and \ref{sct:3}, we know that $0<\frac{m}{q}\leq m$. Once again, for the same reasons as before, $i(\frac{m}{q})\leq i(m)$. Therefore, $A<i(\frac{m}{q})\leq i(m)$, as desired.
    \end{proof}
\end{corollary}

\begin{corollary}\label{cly:6.13}
    $i(\N)$ is an unbounded subset of $\R$.
    \begin{proof}
        Suppose for the sake of contradiction that $i(\N)$ is bounded above. Then by Definition \ref{dfn:5.6}, there exists a point $A\in\R$ such that $i(n)\leq A$ for all $n\in\N$. Note that $A$ is a positive real number since $0=i(0)\leq A$. But by Corollary \ref{cly:6.12}, $A<i(n)$ for some $n\in\N$, a contradiction.
    \end{proof}
\end{corollary}

\begin{corollary}\label{cly:6.14}\marginnote{\emph{1/21:}}
    If $A\in\R$ is a real number, then there is an integer $n$ such that $i(n-1)\leq A<i(n)$.
    \begin{proof}
        Let $X$ be be the set of all integers $z$ such that $i(z)\leq A$. Symbolically,
        \begin{equation*}
            X = \{z\mid z\in\Z\text{ and }i(z)\leq A\}
        \end{equation*}
        Since $A\neq\emptyset$ by Definition \ref{dfn:6.1}, there exists a point $\frac{p}{q}\in\Q$ such that $\frac{p}{q}\in A$. As in Corollary \ref{cly:6.12}, we let $q>0$. It follows by Scripts \ref{sct:2} and \ref{sct:3} that if $p\geq 0$, then $0\leq\frac{p}{q}$, i.e.\footnote{For the same reasons as in Corollary \ref{cly:6.12}.}, $i(0)\leq A$ and if $p<0$, then $p\leq\frac{p}{q}$, i.e., $i(p)\leq A$. Thus, in either case, $X$ is nonempty.\par
        Now by Corollary \ref{cly:6.12}, there exists a nonzero natural number $m$ such that $A<i(m)$. Let $f:X\to\N$ be defined by the rule
        \begin{equation*}
            f(x) = m-x
        \end{equation*}
        By Script \ref{sct:1}, $f$ is an injective function, $f(X)\subset\N$, and $f(X)$ is nonempty (since $X$ is nonempty). Thus, by Additional Exercise \ref{axr:0.1}, there is a least element, which we shall call $y$, in $f(X)$. Since $f$ is injective, there exists exactly one object $n-1\in X$ such that $f(n-1)=y$.\par
        By the definition of $X$, $i(n-1)\leq A$. To prove that $A<i(n)$, suppose for the sake of contradiction that $i(n)\leq A$. This coupled with the fact that $n\in\Z$ implies that $n\in X$. Thus, $f(n)\in f(X)$. But $f(n)=m-n<m-n+1=m-(n-1)=f(n-1)$, contradicting the fact that $f(n-1)$ is the least element of $f(X)$.
    \end{proof}
\end{corollary}

\begin{axiom}\label{axm:6.5}\marginnote{\emph{1/26:}}
    The continuum contains a countable dense subset.
\end{axiom}

\begin{definition}\label{dfn:6.15}
    Let $X$ and $Y$ be sets with orderings $<_X$ an $<_Y$, respectively. A function $f:X\to Y$ is \textbf{order-preserving} if for all $r,s\in X$,
    \begin{equation*}
        r<_Xs \Longrightarrow f(r)<_Yf(s)
    \end{equation*}
    Note that the function $i:\Q\to\R$ discussed above is order preserving.
\end{definition}

\begin{exercise}\label{exr:6.16}
    Let $C$ satisfy Axioms 1-5. Let $K\subset C$ be a countable dense subset of $C$. Construct an order-preserving bijection $f:\Q\to K$.
    \begin{lemma*}\leavevmode
        \begin{enumerate}[label={\alph*\textup{)}}]
            \item $K$ satisfies Axiom \ref{axm:3.3}.
            \item \emph{(Density Lemma)} For all $x,y\in K$, if $x<y$, then there exists a point $z\in K$ such that $z$ is between $x$ and $y$.
        \end{enumerate}
        \begin{proof}[Proof of a]
            To prove that $K$ satisfies Axiom \ref{axm:3.3}, we must verify that $K$ has neither a first nor a last point. We will address the first point question first. Suppose for the sake of contradiction that $K$ has a first point $x$. Then by Definition \ref{dfn:3.3}, $x\leq y$ for all $y\in K$. However, since $C$ satisfies Axiom \ref{axm:3.3}, there exists an object $a\in C$ such that $a<x$. Now consider the region $\underline{ax}$. We have by Corollary \ref{cly:5.3} that there exists a point $p\in\underline{ax}$. Additionally, we have by Script \ref{sct:3} that $\underline{ax}\cap K=\emptyset$. Thus, $\underline{ax}\cap(K\setminus\{p\})=\emptyset$, implying by Definition \ref{dfn:3.13} that $p\notin LP(K)$. But since $p\in C$ and $p\notin LP(K)$, we have by Definition \ref{dfn:6.8} that $K$ is not dense in $C$, a contradiction.\par
            The proof is symmetric for last points.
        \end{proof}
        \begin{proof}[Proof of b]
            Suppose for the sake of contradiction that that there exist $x,y\in K$ with $x<y$ such that no point $z\in K$ is between $x$ and $y$. By Theorem \ref{trm:5.2}, there exists $p\in C$ such that $p$ is between $x$ and $y$. Consequently, by Definition \ref{dfn:3.10}, $p\in\underline{xy}$. Additionally, we have by Script \ref{sct:3} that $\underline{xy}\cap K=\emptyset$. It follows that $\underline{xy}\cap(K\setminus\{p\})=\emptyset$, implying by Definition \ref{dfn:3.13} that $p\notin LP(K)$. But since $p\in C$ and $p\notin LP(K)$, we have by Definition \ref{dfn:6.8} that $K$ is not dense in $C$, a contradiction.
        \end{proof}
    \end{lemma*}
    \begin{proof}[Proof of Exercise \ref{exr:6.16}]
        By Theorem \ref{trm:2.11}, $\Q$ is countable, implying by Definition \ref{dfn:1.35} that there exists a bijection $g:\N\to\Q$. The existence of this bijection means that we can refer to an arbitrary element $q$ of $\Q$ by the number $n$ for which $g(n)=q$; in another notation, we can refer to $q$ as $q_n$. Thus, since every element of $\Q$ can be written as $q_n$ for some $n\in\N$, we can write $\Q=\{q_1,q_2,\dots\}$. Similarly, we can express $K$ as $K=\{k_1,k_2,\dots\}$. We will use this method of referring to the elements of $\Q$ to construct $f$.\par
        We define $f$ recursively with strong induction. For the base case $q_1$, we define $f(q_1)=k_1$. Now suppose inductively that we have defined $f(q_1),f(q_2),\dots,f(q_n)$; we now seek to define $f(q_{n+1})$. By Theorem \ref{trm:3.5}, the symbols $a_1,\dots,a_{n+1}$ can be assigned to $q_1,\dots,q_{n+1}$ so that $a_1<_\Q a_2<_\Q \cdots<_\Q a_{n+1}$. We divide into three cases ($q_{n+1}=a_1$, $q_{n+1}=a_{n+1}$, and $q_{n+1}=a_i$ where $1<i<n+1$). First, suppose that $q_{n+1}=a_1$. By the inductive hypothesis, $f(a_2),f(a_3),\dots,f(a_{n+1})$ are defined elements of $K$. At this point, define the set $X=\{k\in K\mid k<_Kf(a_2)\}$. It follows by Lemma (a) that this set is nonempty. Thus, by the well-ordering principle, there exists a $k_i\in X$ such that $i\leq j$ for all $k_j\in X$. We let $f(q_{n+1})=k_i$. The second case is symmetric to the first. Third, suppose that $q_{n+1}=a_i$ where $1<i<n+1$. By the inductive hypothesis, $f(a_1),\dots,f(a_{i-1}),f(a_{i+1}),\dots,f(a_{n+1})$ are defined elements of $K$. At this point, define the set $X=\{k\in K\mid f(a_{i-1})<_Kk<_Kf(a_{i+1})\}$. It follows by Lemma (b) that this set is nonempty. Thus, by the well-ordering principle, there exists a $k_i\in X$ such that $i\leq j$ for all $k_j\in X$. We let $f(q_{n+1})=k_i$.\par\smallskip
        To prove that $f$ is order-preserving, we will first verify the following claim (which we will refer to as Lemma (c) for future reference): Consider the set $\{q_1,\dots,q_n\}\subset\Q$; if the symbols $a_1,\dots,a_n$ are assigned to $q_1,\dots,q_n$ such that $a_1<_\Q a_2<_\Q \cdots<_\Q a_n$, then $f(a_1)<_Kf(a_2)<_K\cdots<_Kf(a_n)$. We will then use this result to prove that $f$ is order-preserving for any two arbitrary elements $q_i,q_j\in\Q$. Let's begin.\par
        To verify the above claim, we induct on $n$. The base case $n=1$ is vacuously true. Now suppose inductively that we have proven the claim for $n$; we now seek to prove it for $n+1$. By Theorem \ref{trm:3.5}, the symbols $a_1,\dots,a_{n+1}$ can be assigned to $q_1,\dots,q_{n+1}$ so that $a_1<_\Q a_2<_\Q \cdots<_\Q a_{n+1}$. We divide into three cases ($q_{n+1}=a_1$, $q_{n+1}=a_{n+1}$, and $q_{n+1}=a_i$ where $1<i<n+1$). First, suppose that $q_{n+1}=a_1$. By the definition of $f$, $f(q_{n+1})\in\{k\in K\mid k<_Kf(a_2)\}$, meaning that $f(q_{n+1})=f(a_1)<_Kf(a_2)$. Additionally, by the inductive hypothesis, we know that $f(a_2)<_Kf(a_3)<_K\cdots<_Kf(a_{n+1})$ (since $a_2,\dots,a_{n+1}$ correspond to $q_1,\dots,q_n$). These two results imply that $f(a_1)<_Kf(a_2)<_K\cdots<_Kf(a_{n+1})$. The proof of the second case is symmetric to that of the first. Third, suppose that $q_{n+1}=a_i$ where $1<i<n+1$. By the definition of $f$, $f(q_{n+1})\in\{k\in K\mid f(a_{i-1})<_Kk<_Kf(a_{i+1})\}$, meaning that $f(a_{i-1})<_Kf(q_{n+1})=f(a_i)<_Kf(a_{i+1})$. Additionally, by the inductive hypothesis, we know that $f(a_1)<_K\cdots<_Kf(a_{i-1})<_Kf(a_{i+1})<_K\cdots<_Kf(a_{n+1})$ (for an analogous reason to before). These two results imply that $f(a_1)<_Kf(a_2)<_K\cdots<_Kf(a_{n+1})$.\par
        We are now ready to actually prove that $f$ is order-preserving. To do so, Definition \ref{dfn:6.15} tells us that it will suffice to show that for all $q_i,q_j\in\Q$, $q_i<_\Q q_j$ implies $f(q_i)<_Kf(q_j)$. Let $q_i,q_j$ be arbitrary elements of $\Q$ such that $q_i<_\Q q_j$. Since $q_i<_\Q q_j$, $q_i\neq q_j$, implying that $i\neq j$. We divide into two cases ($i<j$ and $i>j$). Suppose first that $i<j$. By Theorem \ref{trm:3.5}, the symbols $a_1,\dots,a_j$ can be assigned to $q_1,\dots,q_j$ so that $a_1<_\Q a_2<_\Q \cdots<_\Q a_j$. Let $q_j=a_l$. Since $q_i<_\Q q_j$, we know that $q_i=a_m$ where $m<l$. Additionally, by Lemma (c), we know that $f(a_1)<_Kf(a_2)<_K\cdots<_Kf(a_j)$. It follows that $f(a_m)<_Kf(a_l)$, implying that $f(q_i)<_Kf(q_j)$, as desired. The proof is symmetric in the other case.\par\smallskip
        To prove that $f$ is bijective, Definition \ref{dfn:1.20} tells us that it will suffice to show that $f$ is injective and surjective.\par
        To show that $f$ is injective, Definition \ref{dfn:1.20} tells us that it will suffice to demonstrate that $q_i\neq q_j$ implies $f(q_i)\neq f(q_j)$. WLOG let $q_i<_\Q q_j$. Then since $f$ is order-preserving, Definition \ref{dfn:6.15} implies that $f(q_i)<_Kf(q_j)$. It follows that $f(q_i)\neq f(q_j)$, as desired.\par
        To show that $f$ is surjective, Definition \ref{dfn:1.20} tells us that it will suffice to demonstrate that for all $k_n\in K$, there exists a $q_i\in\Q$ such that $f(q_i)=k_n$. To do this, we induct on $n$. For the base case $n=1$, it follows from the definition of $f$ that $f(q_1)=k_1$. Now suppose inductively that for each $k_1,\dots,k_n$, there exists a $q_i\in\Q$ such that $f(q_i)=k_n$; we now seek to prove the claim for $n+1$. By Theorem \ref{trm:3.5}, the symbols $b_1,\dots,b_{n+1}$ can be assigned to $k_1,\dots,k_{n+1}$ so that $b_1<_Kb_2<_K\cdots<_Kb_{n+1}$. We divide into three cases ($k_{n+1}=b_1$, $k_{n+1}=b_{n+1}$, and $k_{n+1}=b_i$ where $1<i<n+1$). First, suppose that $k_{n+1}=b_1$. By the inductive hypothesis, $b_2=f(q_i)<_Kb_3=f(q_j)<_K\cdots<_Kb_{n+1}=f(q_l)$. It follows by Definition \ref{dfn:6.15} that $q_i<_\Q q_j<_\Q \cdots<_\Q q_l$. At this point, define the set $X=\{q\in\Q\mid q<_\Q q_i\}$. It follows from Exercise \ref{exr:3.9d} that this set is nonempty. Thus, by the well-ordering principle, there exists a $q_m\in X$ such that $m\leq m'$ for all $k_{m'}\in X$. By the definition of $f$, $f(q_m)=k_{n+1}$. The proof of the second case is symmetric to that of the first. Third, suppose that $k_{n+1}=b_i$ where $1<i<n+1$. By the inductive hypothesis, $b_2=f(q_j)<_K\cdots<_Kb_{i-1}=f(q_{j'})<_Kb_{i+1}=f(q_l)<_K\cdots<_Kb_{n+1}=f(q_{l'})$. It follows by Definition \ref{dfn:6.15} that $q_j<_\Q \cdots<_\Q q_{j'}<_\Q q_l<_\Q \cdots<_\Q q_{l'}$. At this point, define the set $X=\{q\in\Q\mid q_{j'}<_\Q q<_\Q q_l\}$. It follows from Additional Exercise \ref{axr:3.1} that this set is nonempty. Thus, by the well-ordering principle, there exists a $q_m\in X$ such that $m\leq m'$ for all $k_{m'}\in X$. By the definition of $f$, $f(q_m)=k_{n+1}$.
    \end{proof}
\end{exercise}

\begin{exercise}\label{exr:6.17}
    Let $f:\Q\to K$ be an order-preserving bijection, as found in Exercise \ref{exr:6.16}. Let $A\in\R$. Then $A\subset\Q$ and so $f(A)\subset K\subset C$. Define $F:\R\to C$ by
    \begin{equation*}
        F(A)=\sup f(A)
    \end{equation*}
    \begin{enumerate}
        \item Show $\sup f(A)$ exists, so $F$ is well-defined.
        \item Show $F$ is injective and order-preserving.
    \end{enumerate}
    \begin{proof}[Proof of 1]
        To prove that $\sup f(A)$ exists, Theorem \ref{trm:5.17} tells us that it will suffice to show that $f(A)$ is nonempty and bounded above. To show that $f(A)$ is nonempty, Definition \ref{dfn:1.8} tells us that it will suffice to find an element of $f(A)$. By Definition \ref{dfn:6.1}, $A\neq\emptyset$. Thus, by Definition \ref{dfn:1.8}, there exists an object $x\in A$. It follows by Definition \ref{dfn:1.18} that $f(x)\in f(A)$, as desired. To show that $f(A)$ is bounded above, Definition \ref{dfn:5.6} tells us that it will suffice to find an object $f(x)\in f(A)$ such that $f(x)\geq_Cf(a)$ for all $f(a)\in f(A)$. By Definition \ref{dfn:6.1}, $A\neq\Q$ and $A\subset\Q$. Thus, by Definition \ref{dfn:1.2}, there exists an object $x\in\Q$ such that $x\notin A$. It follows from the latter condition by Lemma \ref{lem:6.2} that $x$ is an upper bound for $A$. Thus, by Definition \ref{dfn:5.6}, $x\geq a$ for all $a\in A$. Consequently, by Definition \ref{dfn:6.15}, $f(x)$ is an object such that $f(x)\geq_Cf(a)$ for all $f(a)\in f(A)$, as desired.
    \end{proof}
    \begin{proof}[Proof of 2]
        To prove that $F$ is order-preserving, Definition \ref{dfn:6.15} tells us that it will suffice to show that for all $A,B\in\R$, $A<_\R B$ implies $F(A)<_CF(B)$. Let $A,B$ be two arbitrary elements of $\R$ satisfying $A<_\R B$. Then by Definitions \ref{dfn:6.4} and \ref{dfn:1.3}, there exists a point $x\in B$ such that $x\notin A$. It follows from the latter condition by Lemma \ref{lem:6.2} and Definition \ref{dfn:5.6} that $x\geq a$ for all $a\in A$. Thus, by Definition \ref{dfn:6.15}, $f(x)\geq_Cf(a)$ for all $f(a)\in f(A)$. Consequently, by Definition \ref{dfn:5.7}, $\sup f(A)\leq_Cf(x)$. Additionally, by Definition \ref{dfn:6.1}, there exists a point $y\in B$ such that $y>x$. Thus, by Definition \ref{dfn:6.15}, we have that $f(y)>_Cf(x)$. It follows by Definitions \ref{dfn:5.6} and \ref{dfn:5.7} that $f(y)\leq_C\sup f(B)$. Combining two results, we therefore have that $\sup f(A)\leq_C f(x)<_Cf(y)\leq_C\sup f(B)$, meaning that $F(A)=\sup f(A)<_C\sup f(B)=F(B)$, as desired.\par
        To prove that $F$ is injective, Definition \ref{dfn:1.20} tells us that it will suffice to show that if $A\neq B$, then $F(A)\neq F(B)$. Let $A,B$ be two distinct real numbers. Then by Exercise \ref{exr:6.5}, $A<B$ or $B<A$. We now divide into two cases. Suppose first that $A<B$. Then $F(A)<F(B)$ by Definition \ref{dfn:6.15} (which we have just proven applies to $F$). This implies by Definition \ref{dfn:3.1} that $F(A)\neq F(B)$, as desired. The proof is symmetric in the other case.
    \end{proof}
\end{exercise}

\begin{theorem}\label{trm:6.18}
    Suppose that $C$ is a continuum satisfying Axioms 1-5. Then $C$ is isomorphic to the real numbers $\R$; i.e., there is an order-preserving bijection $F:\R\to C$.
    \begin{lemma*}
        Let $K$ be a dense subset of $C$. For all $x,y\in C$, if $x<y$, then there exists a point $z\in K$ such that $z$ is between $x$ and $y$.
        \begin{proof}
            Suppose for the sake of contradiction that there exist two points $x,y\in C$ with $x<y$ such that no point $z\in K$ is between $x$ and $y$. By Corollary \ref{cly:5.3}, the region $\underline{xy}$ is infinite. Thus, we can pick a point $p\in\underline{xy}$. Additionally, by Definition \ref{dfn:1.6}, we have that $\underline{xy}\cap K=\emptyset$. Thus, $\underline{xy}\cap(K\setminus\{p\})=\emptyset$, implying by Definition \ref{dfn:3.13} that $p\notin LP(K)$. But since $p\in C$ and $p\notin LP(K)$, we have by Definition \ref{dfn:6.8} that $K$ is not dense in $C$, a contradiction.
        \end{proof}
    \end{lemma*}
    \begin{proof}[Proof of Theorem \ref{trm:6.18}]
        By Axiom \ref{axm:6.5}, $C$ contains a countable dense subset $K$. By Exercise \ref{exr:6.16}, there exists an order-preserving bijection $f:\Q\to K$. By Exercise \ref{exr:6.17}, there exists an order-preserving injection $F:\R\to C$. To prove that there is an order-preserving bijection $F:\R\to C$, all that is left to do is to demonstrate that $F$ (as defined in Exercise \ref{exr:6.17}) is surjective.\par\smallskip
        To do this, Definition \ref{dfn:1.20} tells us that it will suffice to show that for all $X\in C$, there exists an object $A\in\R$ such that $F(A)=X$. Put more simply, we must find a Dedekind cut $A$ such that $\sup f(A)=X$ for every $X\in C$. To do this, we will begin by constructing the set $S=\{k\in K\mid k<X\}$. We will then verify that the preimage $f^{-1}(S)$ is a Dedekind cut. Lastly, we will verify that $\sup f(f^{-1}(S))=X$. Let's begin.\par\smallskip
        Let $X$ be an arbitrary element of $C$. Define $S$ as above. To verify that $f^{-1}(S)$ is a Dedekind cut, Definition \ref{dfn:6.1} tells us that it will suffice to confirm that $f^{-1}(S)\neq\emptyset$; $f^{-1}(S)\neq\Q$; if $r\in f^{-1}(S)$ and $s\in\Q$ satisfy $s<r$, then $s\in f^{-1}(S)$; and if $r\in f^{-1}(S)$, then there is some $s\in f^{-1}(S)$ with $s>r$. We will take this one claim at a time.\par
        To confirm that $f^{-1}(S)\neq\emptyset$, Definition \ref{dfn:1.8} tells us that it will suffice to find an element of $f^{-1}(S)$. By Axiom \ref{axm:3.3} and Definition \ref{dfn:3.3}, there exists some point $Y\in C$ such that $Y<X$. Consequently, by the lemma and Definition \ref{dfn:3.6}, there exists a point $f(p)\in K$\footnote{Note that we know that the element of $K$ (the existence of which is implied by the lemma) can be written in the form $f(p)$ because $f$ is bijective.} such that $Y<f(p)<X$. It follows by the definition of $S$ that $f(p)\in S$. Therefore, by Definition \ref{dfn:1.18}, $p\in f^{-1}(S)$, as desired.\par
        To confirm that $f^{-1}(S)\neq\Q$, Definition \ref{dfn:1.2} tells us that it will suffice to find an element of $\Q$ that is not an element of $f^{-1}(S)$. By Axiom \ref{axm:3.3} and Definition \ref{dfn:3.3}, there exists some point $Y\in C$ such that $X<Y$. Consequently, by the lemma and Definition \ref{dfn:3.6}, there exists a point $f(p)\in K$ such that $X<f(p)<Y$. It follows by the definition of $S$ that $f(p)\notin S$. Therefore, by Definition \ref{trm:6.18}, $p\in\Q$ but $p\notin f^{-1}(S)$, as desired.\par
        To confirm that if $r\in f^{-1}(S)$ and $s\in\Q$ satisfy $s<r$, then $s\in f^{-1}(S)$, we let $r\in f^{-1}(S)$ and $s\in\Q$ be arbitrary elements of their respective sets that satisfy $s<r$ and seek to verify that $s\in f^{-1}(S)$. By Definition \ref{dfn:1.18}, the fact that $r\in f^{-1}(S)$ implies that $f(r)\in S$. Thus, by the definition of $S$, $f(r)<X$. Additionally, by the definition of $f$ and Definition \ref{dfn:6.15}, $f(s)\in K$ and $f(s)<f(r)$, respectively. Since $f(s)<f(r)$ and $f(r)<X$, transitivity implies that $f(s)<X$. This combined with the previously established fact that $f(s)\in K$ implies that $f(s)\in S$. Therefore, by Definition \ref{dfn:1.18}, $s\in f^{-1}(S)$, as desired.\par
        To confirm that if $r\in f^{-1}(S)$, then there is some $s\in f^{-1}(S)$ with $s>r$, we let $r\in f^{-1}(S)$ and seek to find such an $s$. As before, $r\in f^{-1}(S)$ implies that $f(r)\in S$. Thus, by the definition of $S$, $f(r)<X$. It follows by the lemma and Definition \ref{dfn:3.6} that there exists a point $f(s)\in K$ such that $f(r)<f(s)<X$. Consequently, by the definition of $S$, we have that $f(s)\in S$. Therefore, by Definitions \ref{dfn:1.18} and \ref{dfn:6.15}, $s\in f^{-1}(S)$ and $r<s$, respectively, as desired.\par\smallskip
        Since $f$ is bijective, Script \ref{sct:1} asserts that $f(f^{-1}(S))=S$. Thus, $\sup f(f^{-1}(S))=\sup S$. To verify that $\sup S=X$, Definition \ref{dfn:5.7} tells us that it will suffice to confirm that $X$ is an upper bound of $S$ and if $U$ is an upper bound of $S$, $X\leq U$. To confirm the former statement, Definition \ref{dfn:5.6} tells us that it will suffice to show that $k\leq X$ for all $k\in S$. But by the definition of $S$, this is true. To confirm the latter statement, suppose for the sake of contradiction that there exists an upper bound $U$ of $S$ such that $U<X$. Since $U<X$, the lemma and Definition \ref{dfn:3.6} imply that there exists a point $Z\in K$ such that $U<Z<X$. It follows by the definition of $S$ that $Z\in S$. Since there exists an element of $S$ greater than $U$, Definition \ref{dfn:5.6} asserts that $U$ is not an upper bound of $S$, a contradiction.
    \end{proof}
\end{theorem}




\end{document}