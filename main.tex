\documentclass{report}

\usepackage[margin=1in]{geometry}
\usepackage{fancyhdr}
\usepackage{xr}
\usepackage{marginnote}
\usepackage{scrextend}
\usepackage[bottom]{footmisc}
\usepackage{enumitem}
\usepackage{amsmath,amssymb,amsthm}
\usepackage{bm}
\usepackage[hidelinks]{hyperref}

\fancypagestyle{main}{
    \fancyhf{}
    \fancyfoot[R]{Labalme\ \thepage}
    \fancyhead[R]{MATH\ 16210}
    \fancyhead[L]{\leftmark}
}
\fancypagestyle{plain}{
    \fancyhead{}
    \renewcommand{\headrulewidth}{0pt}
}

\externaldocument{main}
\externaldocument{../../../1 Autumn Quarter/Honors Calculus I IBL (Cartee)/MATH16110Notes/main}

\reversemarginpar

\deffootnotemark{\textsuperscript{\textup{[}\thefootnotemark\textup{]}}}
\deffootnote[2.1em]{0em}{0em}{\textsuperscript{\thefootnote}}

\newtheorem{theorem}{Theorem}[chapter]
\newtheorem{proposition}[theorem]{Proposition}
\newtheorem{lemma}[theorem]{Lemma}
\newtheorem{corollary}[theorem]{Corollary}
\theoremstyle{definition}
\newtheorem{definition}[theorem]{Definition}
\newtheorem{exercise}[theorem]{Exercise}
\newtheorem{remark}[theorem]{Remark}

\renewcommand{\chaptername}{Script}

\newcommand{\N}{\mathbb{N}}
\newcommand{\Z}{\mathbb{Z}}
\newcommand{\Q}{\mathbb{Q}}
\newcommand{\R}{\mathbb{R}}
\newcommand{\eqclass}[2]{\left[ \frac{#1}{#2} \right]}

\usepackage{subfiles}

\title{MATH 16210 (Honors Calculus II IBL) Notes}
\author{Steven Labalme}

\begin{document}




\maketitle



\pagenumbering{roman}
\tableofcontents
\newpage



\pagenumbering{arabic}
\pagestyle{main}
\renewcommand{\chaptermark}[1]{\markboth{\chaptername\ \thechapter}{}}
\setcounter{chapter}{5}
\subfile{Script6/script6.tex}


\section{Discussion}
\begin{itemize}
    \item \marginnote{1/12:}Upper limit at signing up for 4-5 across the script.
    \item Lemma \ref{lem:6.2} is probably more straightforward using a contradiction argument.
    \item Briefly restate the algebra of Exercise \ref{exr:4.24} in Exercise \ref{exr:6.3c}.
    \item \marginnote{1/14:}Turning in Script \ref{sct:5} journals is optional --- it will boost your grade a bit if you do.
    \begin{itemize}
        \item Your journal grade will be whichever is higher: the average of all your journal grades with and without Script \ref{sct:5}.
        \item Script \ref{sct:5} will probably be due Wednesday, 1/20.
    \end{itemize}
    \item In Lemma \ref{lem:6.6}, do we need to prove that the union of arbitrarily many Dedekind cuts is, itself, a Dedekind cut? Yes.
    \item \marginnote{1/18:}Is there a way to prove something else besides $A$ is not open in Exercise \ref{exr:6.7}?
    \begin{itemize}
        \item This is probably it as far as proving that continuua are connected.
        \item It may not be possible to prove that \emph{any} of the statements are wrong, but he's not sure.
    \end{itemize}
    \item Is Lemma \ref{lem:6.9} used in the proofs of any subsequent results, or is it just a less important result (hence the lemma designation)?
    \begin{itemize}
        \item We can think of it as an alternate definition for density --- we could prove Definition \ref{dfn:6.8} from it.
    \end{itemize}
    \item Is my handwavey use of Scripts \ref{sct:2} and \ref{sct:3} ok in Corollary \ref{cly:6.12}?
    \begin{itemize}
        \item I'm fine.
    \end{itemize}
    \item Is there a simpler way to prove Corollaries \ref{cly:6.12} and \ref{cly:6.14}?
    \begin{itemize}
        \item hi
    \end{itemize}
    \item Is the math REU still running this summer?
    \begin{itemize}
        \item He's not sure; UChicago's may not be NSF approved, hence why its not on the website rn.
    \end{itemize}
    \item What other summer opportunities would you recommend for a student at my level?
    \begin{itemize}
        \item He did an REU at UWisconsin when he was an undergrad.
        \item Sounds like its pretty much just REUs for undergrads.
        \item I could ask around to see if anyone is a Knot Theorist/willing to sponsor me.
    \end{itemize}
    \item \marginnote{1/19:}Easier Corollary \ref{cly:6.12}:
    \begin{itemize}
        \item Let $B>A$. Then $A<i(\frac{m}{q})<B$. Then $A<i(m)$.
    \end{itemize}
    \item Several proofs were given for Corollary \ref{cly:6.14}. One other correct one constructed the nonempty, bounded above set of all $i(n)$ less than or equal to $A$ and considered its supremum.
    \item \marginnote{1/21:}Now graded a bit more critically on presentations.
    \begin{itemize}
        \item Write big, talk loudly, don't talk to the blackboard.
    \end{itemize}
    \item My original proof of Corollary \ref{cly:6.14} is incorrect because I can't split into cases the way I did (\emph{longer expo}).
    \begin{itemize}
        \item Instead, use Seb's approach.
    \end{itemize}
\end{itemize}




\end{document}