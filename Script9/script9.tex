\documentclass[../main.tex]{subfiles}

\pagestyle{main}
\renewcommand{\chaptermark}[1]{\markboth{\chaptername\ \thechapter}{}}
\setcounter{chapter}{8}

\begin{document}




\chapter{Continuous Functions}\label{sct:9}
\section{Journal}
\begin{lemma}\label{lem:9.1}\marginnote{\emph{2/16:}}
    Let $X\subset\R$ and $f:X\to\R$. If $A,B\subset\R$, then
    \begin{gather*}
        f^{-1}(A\cup B) = f^{-1}(A)\cup f^{-1}(B)\\
        f^{-1}(A\cap B) = f^{-1}(A)\cap f^{-1}(B)\\
        f^{-1}(A\setminus B) = f^{-1}(A)\setminus f^{-1}(B)\\
        f^{-1}(\R) = X
    \end{gather*}
    \begin{proof}
        To prove that $f^{-1}(A\cup B)=f^{-1}(A)\cup f^{-1}(B)$, Definition \ref{dfn:1.2} tells us that it will suffice to show that every $x\in f^{-1}(A\cup B)$ is an element of $f^{-1}(A)\cup f^{-1}(B)$ and vice versa. Suppose first that $x$ is an arbitrary element of $f^{-1}(A\cup B)$. Then by Definition \ref{dfn:1.18}, $f(x)\in A\cup B$. Thus, by Definition \ref{dfn:1.5}, $f(x)\in A$ or $f(x)\in B$. We now divide into two cases. If $f(x)\in A$, then by Definition \ref{dfn:1.18}, $x\in f^{-1}(A)$. It follows by Definition \ref{dfn:1.5} that $x\in f^{-1}(A)\cup f^{-1}(B)$, as desired. The argument is symmetric in the other case. Now suppose that $x\in f^{-1}(A)\cup f^{-1}(B)$. Then by Definition \ref{dfn:1.5}, $x\in f^{-1}(A)$ or $x\in f^{-1}(B)$. We now divide into two cases. If $x\in f^{-1}(A)$, then by Definition \ref{dfn:1.18}, $f(x)\in A$. It follows by Definition \ref{dfn:1.5} that $f(x)\in A\cup B$. Therefore, by Definition \ref{dfn:1.18}, $x\in f^{-1}(A\cup B)$. The argument is symmetric in the other case, as desired.\par
        To prove that $f^{-1}(A\cap B)=f^{-1}(A)\cap f^{-1}(B)$, Definition \ref{dfn:1.2} tells us that it will suffice to show that every $x\in f^{-1}(A\cap B)$ is an element of $f^{-1}(A)\cap f^{-1}(B)$ and vice versa. Suppose first that $x$ is an arbitrary element of $f^{-1}(A\cap B)$. Then by Definition \ref{dfn:1.18}, $f(x)\in A\cap B$. Thus, by Definition \ref{dfn:1.6}, $f(x)\in A$ and $f(x)\in B$. It follows by consecutive applications of Definition \ref{dfn:1.18} that $x\in f^{-1}(A)$ and $x\in f^{-1}(B)$. Therefore, by Definition \ref{dfn:1.6}, $x\in f^{-1}(A)\cap f^{-1}(B)$, as desired. Now suppose that $x\in f^{-1}(A)\cap f^{-1}(B)$. Then by Definition \ref{dfn:1.6}, $x\in f^{-1}(A)$ and $x\in f^{-1}(B)$. It follows by consecutive applications of Definition \ref{dfn:1.18} that $f(x)\in A$ and $f(x)\in B$. Thus, by Definition \ref{dfn:1.6}, $f(x)\in A\cap B$. Therefore, by Definition \ref{dfn:1.18}, $x\in f^{-1}(A\cap B)$, as desired.\par
        To prove that $f^{-1}(A\setminus B)=f^{-1}(A)\setminus f^{-1}(B)$, Definition \ref{dfn:1.2} tells us that it will suffice to show that every $x\in f^{-1}(A\setminus B)$ is an element of $f^{-1}(A)\setminus f^{-1}(B)$ and vice versa. Suppose first that $x$ is an arbitrary element of $f^{-1}(A\setminus B)$. Then by Definition \ref{dfn:1.18}, $f(x)\in A\setminus B$. Thus, by Definition \ref{dfn:1.11}, $f(x)\in A$ and $f(x)\notin B$. It follows by consecutive applications of Definition \ref{dfn:1.18} that $x\in f^{-1}(A)$ and $x\notin f^{-1}(B)$. Therefore, by Definition \ref{dfn:1.11}, $x\in f^{-1}(A)\setminus f^{-1}(B)$, as desired. Now suppose that $x\in f^{-1}(A)\setminus f^{-1}(B)$. Then by Definition \ref{dfn:1.11}, $x\in f^{-1}(A)$ and $x\notin f^{-1}(B)$. It follows by consecutive applications of Definition \ref{dfn:1.18} that $f(x)\in A$ and $f(x)\notin B$. Thus, by Definition \ref{dfn:1.11}, $f(x)\in A\setminus B$. Therefore, by Definition \ref{dfn:1.18}, $x\in f^{-1}(A\setminus B)$, as desired.\par
        To prove that $f^{-1}(\R)=X$, Definition \ref{dfn:1.2} tells us that it will suffice to show that every $x\in f^{-1}(\R)$ is an element of $X$ and vice versa. Suppose first that $x$ is an arbitrary element of $f^{-1}(\R)$. Then by Definition \ref{dfn:1.18}, $x\in X$, as desired. Now suppose that $x\in X$. Then by Definition \ref{dfn:1.16}, $f(x)\in\R$. It follows by Definition \ref{dfn:1.18} that $x\in f^{-1}(\R)$, as desired.
    \end{proof}
\end{lemma}

\begin{exercise}\label{exr:9.2}
    Let $f:X\to\R$. Let $A\subset X$ and $B\subset\R$. Show that
    \begin{gather*}
        f(f^{-1}(B)) \subset B\\
        A\subset f^{-1}(f(A))
    \end{gather*}
    Give examples to show that the inclusions can be proper.
    \begin{proof}
        To prove that $f(f^{-1}(B))\subset B$, Definition \ref{dfn:1.3} tells us that it will suffice to show that every $y\in f(f^{-1}(B))$ is an element of $B$. Let $y$ be an arbitrary element of $f(f^{-1}(B))$. Then by Definition \ref{dfn:1.18}, $y=f(x)$ for some $x\in f^{-1}(B)$. By Definition \ref{dfn:1.18} again, $f(x)\in B$. Therefore, since $y=f(x)$, it follows that $y\in B$, as desired.\par
        To prove that $A\subset f^{-1}(f(A))$, Definition \ref{dfn:1.3} tells us that it will suffice to show that every $x\in A$ is an element of $f^{-1}(f(A))$. Let $x$ be an arbitrary element of $A$. Then by Definition \ref{dfn:1.18}, $f(x)\in f(A)$. Therefore, by Definition \ref{dfn:1.18}, we have $x\in f^{-1}(f(A))$, as desired.\par
        Let $X=\{1,2\}$ and let $f:X\to\R$ be defined by $f(1)=3$ and $f(2)=3$. If we let $B=\{3,4\}$, then $f(f^{-1}(B))=\{3\}\subsetneq\{3,4\}$. Additionally, if we let $A=\{1\}$, then $A\subsetneq f^{-1}(f(A))=\{1,2\}$.
    \end{proof}
\end{exercise}

\begin{exercise}\label{exr:9.3}
    Let $f:X\to\R$. Let $A\subset X$ and $B\subset\R$. Then $f(A)\subset B\Longleftrightarrow A\subset f^{-1}(B)$.
    \begin{proof}
        Suppose first that $f(A)\subset B$. To prove that $A\subset f^{-1}(B)$, Definition \ref{dfn:1.3} tells us that it will suffice to show that every $x\in A$ is an element of $f^{-1}(B)$. Let $x$ be an arbitrary element of $A$. Then by Definition \ref{dfn:1.18}, $f(x)\in f(A)$. It follows by the hypothesis and Definition \ref{dfn:1.3} that $f(x)\in B$. Therefore, by Definition \ref{dfn:1.18} again, $x\in f^{-1}(B)$.\par
        Now suppose that $A\subset f^{-1}(B)$. To prove that $f(A)\subset B$, Definition \ref{dfn:1.3} tells us that it will suffice to show that every $y\in f(A)$ is an element of $B$. Let $y$ be an arbitrary element of $f(A)$. Then by Definition \ref{dfn:1.18}, $y=f(x)$ for some $x\in A$. It follows by the hypothesis and Definition \ref{dfn:1.3} that $x\in f^{-1}(B)$. Therefore, by Definition \ref{dfn:1.18} again, $y=f(x)\in B$.
    \end{proof}
\end{exercise}

\begin{definition}\label{dfn:9.4}
    Let $X\subset\R$. A function $f:X\to\R$ is \textbf{continuous} if for every open set $U\subset\R$, the preimage $f^{-1}(U)$ is open in $X$.
\end{definition}

\begin{proposition}\label{prp:9.5}
    Let $X\subset\R$. A function $f:X\to\R$ is continuous if and only if for every closed set $F\subset\R$, the preimage $f^{-1}(F)$ is closed in $X$.
    \begin{proof}
        Suppose first that $f$ is continuous. We seek to prove that for every closed set $F\subset\R$, the preimage $f^{-1}(F)$ is closed in $X$. Let $F$ be an arbitrary closed subset of $\R$. Then by Definition \ref{dfn:4.8}, $F=\R\setminus U$ for some open set $U\subset\R$. It follows by Definition \ref{dfn:9.4} since $f$ is continuous that $f^{-1}(U)$ is open in $X$. Additionally, by consecutive applications of Lemma \ref{lem:9.1}, $f^{-1}(F)=f^{-1}(\R\setminus U)=f^{-1}(\R)\setminus f^{-1}(U)=X\setminus f^{-1}(U)$. Therefore, since $f^{-1}(U)$ is open in $X$, Exercise \ref{exr:8.13} implies that $X\setminus f^{-1}(U)=f^{-1}(F)$ is closed in $X$.\par
        The proof is symmetric in the other direction.
    \end{proof}
\end{proposition}

\begin{definition}\label{dfn:9.6}
    Let $X\subset Y\subset\R$ and let $f:Y\to\R$. Then the \textbf{restriction} (of $f$ to $X$), written $f|_X$ is the function $f|_X:X\to\R$ defined by
    \begin{equation*}
        f|_X(x) = f(x)
    \end{equation*}
    for all $x\in X$.
\end{definition}

\begin{proposition}\label{prp:9.7}
    Let $X\subset Y\subset\R$. If $f:Y\to\R$ is continuous, then the restriction of $f$ to $X$ is continuous.
    \begin{proof}
        To prove that $f|_X$ is continuous, Definition \ref{dfn:9.4} tells us that it will suffice to show that for every open set $U\subset\R$, the preimage $f|_X^{-1}(U)$ is open in $X$. Let $U$ be an open subset of $\R$. Then
        \begingroup
        \allowdisplaybreaks
        \begin{align*}
            f|_X^{-1}(U) &= \{x\in X\mid f|_X(x)\in U\}\tag*{Definition \ref{dfn:1.18}}\\
            &= \{x\in X\mid f(x)\in U\}\tag*{Definition \ref{dfn:9.6}}\\
            &= \{x\in Y\mid f(x)\in U\}\cap X\tag*{Script \ref{sct:1}}\\
            &= f^{-1}(U)\cap X\tag*{Definition \ref{dfn:1.18}}\\
            &= (Y\cap G)\cap X\tag*{Definitions \ref{dfn:9.4} and \ref{dfn:8.11}}\\
            &= X\cap G\tag*{Script \ref{sct:1}}
        \end{align*}
        \endgroup
        Since $f|_X^{-1}(U)=X\cap G$ where $G$ is an open set, Definition \ref{dfn:8.11} asserts that $f|_X^{-1}(U)$ is open in $X$.
    \end{proof}
\end{proposition}

\begin{exercise}\label{exr:9.8}
    Show that for any $X\subsetneq\R$ that is not open and any continuous function $f:X\to\R$, there is an open set $U$ for which $f^{-1}(U)$ is open in $X$ but is not open in $\R$.
    \begin{proof}
        We will prove that $\R$ is an open set for which $f^{-1}(\R)$ is open in $X$ but not in $\R$. First, by Theorem \ref{trm:5.1}, $\R$ is open. Next, by Lemma \ref{lem:9.1}, $f^{-1}(\R)=X$. It follows since $f^{-1}(\R)=X=X\cap\R$ (where $\R$ is an open set) by Definition \ref{dfn:8.11} that $f^{-1}(\R)$ is open in $X$. Last, since $X$ is not open (in $\R$) by definition, $f^{-1}(\R)=X$ is not open in $\R$.
    \end{proof}
\end{exercise}

\begin{definition}\label{dfn:9.9}
    The function $f:X\to\R$ is \textbf{continuous} (at $x\in X$) if for every region $R$ containing $f(x)$, there exists an open set $S$ containing $x$ such that $S\cap X\subset f^{-1}(R)$.
\end{definition}

\begin{theorem}\label{trm:9.10}
    The function $f:X\to\R$ is continuous if and only if it is continuous at every $x\in X$.
    \begin{proof}
        Suppose first that $f$ is continuous, and suppose for the sake of contradiction that $f$ is not continuous at every $x\in X$. Then by Definition \ref{dfn:9.9}, there exists some $x\in X$ such that $f$ is not continuous at $x$. Thus, there exists a region $R$ with $f(x)\in R$ such that for all open sets $S$ containing $x$, $S\cap X\not\subset f^{-1}(R)$. Since $f$ is continuous by hypothesis and $R$ is open by Corollary \ref{cly:4.11}, $f^{-1}(R)$ is open in $X$. It follows by Definition \ref{dfn:8.11} that $f^{-1}(R)=X\cap S$ for some open set $S$. But this implies that $f^{-1}(R)\not\subset f^{-1}(R)$, a contradiction.\par
        Now suppose that $f$ is continuous at every $x\in X$. To prove that $f$ is continuous, Definition \ref{dfn:9.4} tells us that it will suffice to show that for every open set $U\subset\R$, the preimage $f^{-1}(U)$ is open in $X$. We divide into two cases ($f^{-1}(U)=\emptyset$ and $f^{-1}(U)\neq\emptyset$). If $f^{-1}(U)=\emptyset$, then since $\emptyset\cap X=\emptyset$ by Script \ref{sct:1} where $\emptyset$ is open by Theorem \ref{trm:5.1}, Definition \ref{dfn:8.11} tells us that $\emptyset=f^{-1}(U)$ is open in $X$, as desired. On the other hand, if $f^{-1}(U)\neq\emptyset$, Definition \ref{dfn:8.11} tells us that it will suffice to show that $f^{-1}(U)=S\cap X$ where $S$ is an open set. We first seek to show that for every $x\in f^{-1}(U)$, there exists an open set $S_x$ containing $x$ such that $S_x\cap X\subset f^{-1}(U)$. Let $x$ be an arbitrary element of $f^{-1}(U)$. It follows by Definition \ref{dfn:1.18} that $f(x)\in U$. Thus, since $U$ is open, we have by Theorem \ref{trm:4.10} that there exists a region $R$ such that $f(x)\in R$ and $R\subset U$. Consequently, since $R$ is open by Corollary \ref{cly:4.11}, we have by Definition \ref{dfn:9.9} that there exists an open set $S_x$ containing $x$ such that $S_x\cap X\subset f^{-1}(R)$. Additionally, Script \ref{sct:1} tells us based off of the fact that $R\subset U$ that $f^{-1}(R)\subset f^{-1}(U)$. Thus, by subset transitivity, $S_x\cap X\subset f^{-1}(U)$. At this point, let $S=\bigcup_{x\in f^{-1}(U)}S_x$. It follows immediately from Corollary \ref{cly:4.18} that $S$ is open. Additionally, since the intersection of each set in the union with $X$ is a subset of $f^{-1}(U)$, it follows by Script \ref{sct:1} that $S\cap X\subset f^{-1}(U)$. Furthermore, for all $x\in f^{-1}(U)$, Definition \ref{dfn:1.18} asserts that $x\in X$. In addition, we have defined an $S_x$ such that $x\in S_x$. These last two results combined demonstrate by Definition \ref{dfn:1.6} that $x\in S\cap X$. Thus, by Definition \ref{dfn:1.3}, $f^{-1}(U)\subset S\cap X$. Consequently, by Theorem \ref{trm:1.7}, $f^{-1}(U)=S\cap X$. Since $f^{-1}(U)$ is the intersection of $X$ with an open set, Definition \ref{dfn:8.11} asserts that it is open in $X$, as desired.
    \end{proof}
\end{theorem}




\end{document}