\usepackage[margin=1in]{geometry}
\usepackage{fancyhdr}
\usepackage{xr}
\usepackage{marginnote}
\usepackage{scrextend}
\usepackage[bottom]{footmisc}
\usepackage{enumitem}
\usepackage{amsmath,amssymb,amsthm}
\usepackage{bm,mathtools}
\usepackage[hidelinks]{hyperref}

\fancypagestyle{main}{
    \fancyhf{}
    \fancyfoot[R]{Labalme\ \thepage}
    \fancyhead[R]{MATH\ 16210}
    \fancyhead[L]{\leftmark}
}
\fancypagestyle{plain}{
    \fancyhead{}
    \renewcommand{\headrulewidth}{0pt}
}

\externaldocument{main}
\externaldocument{../../../1 Autumn Quarter/Honors Calculus I IBL (Cartee)/MATH16110Notes/main}

\reversemarginpar

\deffootnotemark{\textsuperscript{\textup{[}\thefootnotemark\textup{]}}}
\deffootnote[2.1em]{0em}{0em}{\textsuperscript{\thefootnote}}

\newtheorem{theorem}{Theorem}[chapter]
\newtheorem{proposition}[theorem]{Proposition}
\newtheorem{lemma}[theorem]{Lemma}
\newtheorem{corollary}[theorem]{Corollary}
\newtheorem*{lemma*}{Lemma}
\newtheorem{axiom}{Axiom}
\newtheorem{examples}[theorem]{Examples}
\theoremstyle{definition}
\newtheorem{definition}[theorem]{Definition}
\newtheorem{exercise}[theorem]{Exercise}
\newtheorem{remark}[theorem]{Remark}

\renewcommand{\chaptername}{Script}

\newcommand{\N}{\mathbb{N}}
\newcommand{\Z}{\mathbb{Z}}
\newcommand{\Q}{\mathbb{Q}}
\newcommand{\R}{\mathbb{R}}
\newcommand{\F}{\mathbb{F}}
\newcommand{\eqclass}[2]{\left[ \frac{#1}{#2} \right]}

\usepackage{subfiles}