\documentclass[../main.tex]{subfiles}

\pagestyle{main}
\renewcommand{\chaptermark}[1]{\markboth{\chaptername\ \thechapter}{}}
\setcounter{chapter}{7}

\begin{document}




\chapter{Intervals}\label{sct:8}
\section{Journal}
\marginnote{2/9:}Now that we have constructed $\R$ and proved the fundamental facts about it, we will work with the real numbers $\R$ instead of an arbitrary continuum $C$. We will leave behind Dedekind cuts and think of elements of $\R$ as numbers. Accordingly, from now on, we will use lower-case letters like $x$ for real numbers and will write $+$ and $\cdot$ for $\oplus$ and $\otimes$. We will also now use the standard notation $(a,b)$ for the region $\underline{ab}=\{x\in\R\mid a<x<b\}$. Even though the notation is the same, this is \emph{not} the same object as the ordered pair $(a,b)$.\par
More generally, we adopt the following standard notation:
\begin{align}
    \begin{split}
        (a,b) &= \{x\in\R\mid a<x<b\}\\
        [a,b) &= \{x\in\R\mid a\leq x<b\}\\
        (a,b] &= \{x\in\R\mid a<x\leq b\}\\
        [a,b] &= \{x\in\R\mid a\leq x\leq b\}\\
        (a,\infty) &= \{x\in\R\mid a<x\}\\
        [a,\infty) &= \{x\in\R\mid a\leq x\}\\
        (-\infty,b) &= \{x\in\R\mid x<b\}\\
        (-\infty,b] &= \{x\in\R\mid x\leq b\}
    \end{split}\label{eqn:8.1}
\end{align}

\begin{exercise}\label{exr:8.1}
    Identify the sets in Equations \ref{eqn:8.1} that are open/closed/neither.
    \begin{proof}
        Note that by Theorem \ref{trm:5.1}, any of these sets proven to be just one of open or closed will not be the other, i.e., a set proven to be open will not be closed and vice versa.\par\smallskip
        By Corollary \ref{cly:4.11}, $(a,b)$ is open.\par
        By an adaptation of Corollary \ref{cly:5.14}, $b\in LP([a,b))$ but $b\notin[a,b)$. Since $[a,b)$ doesn't contain all of its limit points, Definition \ref{dfn:4.1} tells us that it is not closed. Additionally, since $a\in LP(C\setminus[a,b))$ but $a\notin C\setminus[a,b)$, Definition \ref{dfn:4.8} tells us that it is not open. Therefore, it is neither.\par
        The proof that $(a,b]$ is neither is symmetric to the previous case.\par
        By Corollaries \ref{cly:5.15} and \ref{cly:4.7}, $[a,b]$ is closed.\par
        By Corollary \ref{cly:4.13}, $(a,\infty)$ is open.\par
        By Corollary \ref{cly:4.13} and Definition \ref{dfn:4.8}, $[a,\infty)=C\setminus(-\infty,a)$ is closed.\par
        The proofs that $(-\infty,b)$ and $(-\infty,b]$ are open and closed, respectively, are symmetric to the previous two cases, respectively.
    \end{proof}
\end{exercise}

\begin{definition}\label{dfn:8.2}
    A set $I\subset\R$ is an \textbf{interval} if for all $x,y\in I$ with $x<y$, $[x,y]\subset I$.
\end{definition}
\pagebreak

\begin{lemma}\label{lem:8.3}
    A proper subset $I\subsetneq\R$ is an interval if and only if it takes one of the eight forms in Equations \ref{eqn:8.1}.
    \begin{proof}
        Suppose first that $I\subsetneq\R$ is an interval. If $I=\emptyset$, then $I=(a,a)$ for any $a\in\R$, and we are done. Thus, we will assume for the remainder of the proof of the forward direction that $I$ is nonempty. To address this case, we will first prove that the facts that $I\subsetneq\R$, $I\neq\emptyset$, and $I$ is an interval imply that $I$ is bounded above, bounded below, or both. Then in each of these three cases, we will look at whether $\sup I$ and $\inf I$ (if they exist) are elements of the set or not to differentiate between the eight forms in Equations \ref{eqn:8.1}. Let's begin.\par\medskip
        Suppose for the sake of contradiction that there exists a nonempty interval $I\subsetneq\R$ that is neither bounded above nor bounded below. Since $I\subsetneq\R$, we have by Definition \ref{dfn:1.3} that there exists a point $p\in\R$ such that $p\notin I$. Additionally, since $I$ is neither bounded above nor below, Definition \ref{dfn:5.6} implies that $p$ is neither an upper nor a lower bound of $I$. Thus, there exist $x,y\in I$ such that $x<p$ and $y>p$. Now by Definition \ref{dfn:8.2}, $[x,y]\subset I$. But it follows by Definition \ref{dfn:1.3} that every point in $[x,y]$, including $p$, is an element of $I$, a contradiction.\par\medskip
        We now divide into three cases ($I$ is exclusively bounded below, $I$ is exclusively bounded above, and $I$ is bounded both below and above).\par\smallskip
        First, suppose that $I$ is only bounded below. Since $I$ is a nonempty subset of $\R$ that is bounded below, we have by Theorem \ref{trm:5.17} that $\inf I$ exists. We divide into two cases again ($\inf I\in I$ and $\inf I\notin I$).\par
        If $\inf I\in I$, then we can demonstrate that $I=[\inf I,\infty)$. To do this, Definition \ref{dfn:1.2} tells us that it will suffice to verify that every $p\in I$ is an element of $[\inf I,\infty)$ and vice versa. Let $p$ be an arbitrary element of $I$. Then $p\in\R$ and by Definitions \ref{dfn:5.7} and \ref{dfn:5.6}, $\inf I\leq p$. Therefore, $p\in[\inf I,\infty)$, as desired. Now let $p$ be an arbitrary element of $[\inf I,\infty)$. Then $\inf I\leq p$. Additionally, since $I$ is not bounded above, we have by Definition \ref{dfn:5.6} that there exists $y\in I$ such that $y>p$. Since $\inf I\in I$, $y\in I$, and $\inf I<y$ (by transitivity), $[\inf I,y]\subset I$ by Definition \ref{dfn:8.2}. This combined with the fact that $p\in[\inf I,y]$ (we know that $\inf I\leq p<y$, so $\inf I\leq p\leq y$) implies that $p\in I$, as desired.\par
        If $\inf I\notin I$, then we can demonstrate that $I=(\inf I,\infty)$. As before, to do so, it will suffice to verify that every $p\in I$ is an element of $(\inf I,\infty)$ and vice versa. Let $p$ be an arbitrary element of $I$. Then $p\in\R$ and by Definitions \ref{dfn:5.7} and \ref{dfn:5.6}, $\inf I\leq p$. The additional constraint that $\inf I\notin I$ implies that $\inf I<p$. Therefore, $p\in(\inf I,\infty)$, as desired. Now let $p$ be an arbitrary element of $(\inf I,\infty)$. Then $\inf I<p$. It follows by Lemma \ref{lem:5.11} that there exists a $z\in I$ such that $\inf I\leq z<p$. Additionally, since $I$ is not bounded above, we have by Definition \ref{dfn:5.6} that there exists $y\in I$ such that $y>p$. Since $z\in I$, $y\in I$, and $z<y$ (by transitivity), $[z,y]\subset I$ by Definition \ref{dfn:8.2}. This combined with the fact that $p\in[z,y]$ (we know that $z<p<y$, so $z\leq p\leq y$) implies that $p\in I$, as desired.\par\smallskip
        Second, suppose that $I$ is only bounded above. The proof of this case is symmetric to that of the first.\par\smallskip
        Third, suppose that $I$ is bounded below and above. Since $I$ is a nonempty subset of $\R$ that is bounded above and below, we have by consecutive applications of Theorem \ref{trm:5.17} that both $\sup I$ and $\inf I$ exist. We divide into four cases ($\inf I\in I$ and $\sup I\in I$, $\inf I\in I$ and $\sup I\notin I$, $\inf I\notin I$ and $\sup I\in I$, and $\inf I\notin I$ and $\sup I\notin I$).\par
        If $\inf I\in I$ and $\sup I\in I$, then we can demonstrate that $I=[\inf I,\sup I]$. We divide into two cases again ($\inf I=\sup I$ and $\inf I\neq\sup I$). If $\inf I=\sup I\in I$, then $I=\{\inf I\}=\{\sup I\}=[\inf I,\sup I]$, as desired. On the other hand, if $\inf I\neq\sup I$, we continue. To demonstrate that $I=[\inf I,\sup I]$, Theorem \ref{trm:1.7} tells us that it will suffice to verify that $I\subset[\inf I,\sup I]$ and $[\inf I,\sup I]\subset I$. To verify that the former claim, Definition \ref{dfn:1.3} tells us that it will suffice to confirm that every $p\in I$ is an element of $[\inf I,\sup I]$. Let $p$ be an arbitrary element of $I$. Then $p\in\R$ and by consecutive applications of Definitions \ref{dfn:5.7} and \ref{dfn:5.6}, $\inf I\leq p$ and $p\leq\sup I$. Therefore, $p\in[\inf I,\sup I]$, as desired. On the other hand, since $\inf I\in I$, $\sup I\in I$, and $\inf I<\sup I$ (as follows from Definition \ref{dfn:5.7} and the fact that they are unequal), $[\inf I,\sup I]\subset I$ by Definition \ref{dfn:8.2}, as desired.\par
        If $\inf I\in I$ and $\sup I\notin I$, then we can demonstrate that $I=[\inf I,\sup I)$. To do so, it will suffice to verify that every $p\in I$ is an element of $[\inf I,\sup I)$ and vice versa. Let $p$ be an arbitrary element of $I$. Then $p\in\R$ and by Definitions \ref{dfn:5.7} and \ref{dfn:5.6}, $\inf I\leq p$ and $p\leq \sup I$. The additional constraint that $\sup I\notin I$ implies that $p<\sup I$. Therefore, $p\in[\inf I,\sup I)$, as desired. Now let $p$ be an arbitrary element of $[\inf I,\sup I)$. Then $\inf I\leq p<\sup I$. It follows by Lemma \ref{lem:5.11} that there exists a $y\in I$ such that $p<y\leq\sup I$. Since $\inf I\in I$, $y\in I$, and $\inf I<y$ (by transitivity), $[\inf I,y]\subset I$ by Definition \ref{dfn:8.2}. This combined with the fact that $p\in[\inf I,y]$ (we know that $\inf I\leq p<y$, so $\inf I\leq p\leq y$) implies that $p\in I$, as desired.\par
        If $\inf I\notin I$ and $\sup I\in I$, the proof is symmetric to that of the previous case.\par
        If $\inf I\notin I$ and $\sup I\notin I$, then we can demonstrate that $I=(\inf I,\sup I)$. To do so, it will suffice to verify that every $p\in I$ is an element of $(\inf I,\sup I)$ and vice versa. Let $p$ be an arbitrary element of $I$. Then $p\in\R$ and by Definitions \ref{dfn:5.7} and \ref{dfn:5.6}, $\inf I\leq p$ and $p\leq \sup I$. The additional constraints that $\inf I\notin I$ and $\sup I\notin I$ imply that $\inf I<p$ and $p<\sup I$, respectively. Therefore, $p\in(\inf I,\sup I)$, as desired. Now let $p$ be an arbitrary element of $(\inf I,\sup I)$. Then $\inf I<p<\sup I$. It follows by consecutive applications of Lemma \ref{lem:5.11} that there exist $x,y\in I$ such that $\inf I\leq x<p$ and $p<y\leq\sup I$. Since $x\in I$, $y\in I$, and $x<y$ (by transitivity), $[x,y]\subset I$ by Definition \ref{dfn:8.2}. This combined with the fact that $p\in[x,y]$ (we know that $x<p<y$, so $x\leq p\leq y$) implies that $p\in I$, as desired.\par\bigskip
        Now suppose that $I\subsetneq\R$ takes one of the eight forms in Equations \ref{eqn:8.1}. To prove that $I$ is an interval, Definition \ref{dfn:8.2} tells us that it will suffice to show that for all $x,y\in I$ with $x<y$, $[x,y]\subset I$. Let $x,y$ be arbitrary elements of $I$ with $x<y$. We divide into eight cases (one for each equation in Equations \ref{eqn:8.1}).\par\smallskip
        First, suppose that $I=(a,b)$. To demonstrate that $[x,y]\subset I$, Definition \ref{dfn:1.3} tells us that it will suffice to confirm that every $z\in[x,y]$ is an element of $I$. Let $z$ be an arbitrary element of $[x,y]$. Then by Corollary \ref{cly:5.15}, $x\leq z\leq y$. But since $a<x<y<b$ by Equations \ref{eqn:8.1}, the fact that $a<x\leq z\leq y<b$ implies by Equations \ref{eqn:8.1} that $z\in(a,b)$, as desired.\par
        The proofs of the second, third, and fourth equations are symmetric to that of the first.\par
        Fifth, suppose that $I=(a,\infty)$. To demonstrate that $[x,y]\subset I$, Definition \ref{dfn:1.3} tells us that it will suffice to confirm that every $z\in[x,y]$ is an element of $I$. Let $z$ be an arbitrary element of $[x,y]$. Then by Corollary \ref{cly:5.15}, $x\leq z\leq y$. But since $a<x$ by Equations \ref{eqn:8.1}, the fact that $a<x\leq z$ implies by Equations \ref{eqn:8.1} that $z\in(a,\infty)$, as desired.\par
        The proofs of the sixth, seventh, and eighth equations are symmetric to that of the first.
    \end{proof}
\end{lemma}

\begin{definition}\label{dfn:8.4}
    The \textbf{absolute value} of a real number $x$ is the non-negative number $|x|$ defined by
    \begin{equation*}
        |x| =
        \begin{cases}
            x & x\geq 0\\
            -x & x<0
        \end{cases}
    \end{equation*}
\end{definition}

\begin{exercise}\label{exr:8.5}
    Show that $|x|=|-x|$ for all $x\in\R$. (Note that this also means that $|x-y|=|y-x|$ for any $x,y\in\R$.)
    \begin{proof}
        Let $x$ be an arbitrary element of $\R$. We divide into three cases ($x=0$, $x>0$, and $x<0$). First, suppose that $x=0$. Then since $0=-0$, clearly $|0|=|-0|$, as desired. Second, suppose that $x>0$. Then by Lemma \ref{lem:7.23}\footnote{And, technically, Theorem 7.47.} $-x<0$. Thus, by consecutive applications of Definition \ref{dfn:8.4}, $|x|=x$ and $|-x|=-(-x)$. Therefore, since $-(-x)=x$ by Corollary \ref{cly:7.11}, $|x|=x=|-x|$, as desired. Third, suppose that $x<0$. Then by Lemma \ref{lem:7.23}, $-x>0$. Thus, by consecutive applications of Definition \ref{dfn:8.4}, $|x|=-x$ and $|-x|=-x$. Therefore, $|x|=-x=|-x|$, as desired.
    \end{proof}
\end{exercise}

\begin{definition}\label{dfn:8.6}
    The \textbf{distance} between $x\in\R$ and $y\in\R$ is defined to be $|x-y|$.
\end{definition}

\begin{remark}\label{rmk:8.7}
    It follows from Definition \ref{dfn:8.6} that $|x|$ is the distance between $x$ and $0$.
\end{remark}

\begin{lemma}\label{lem:8.8}
    For any real numbers $x,y,z$, we have
    \begin{enumerate}[label={\textup{(}\alph*\textup{)}}]
        \item $|x+y|\leq|x|+|y|$.
        \item $|x-z|\leq|x-y|+|y-z|$.
        \item $||x|-|y||\leq|x-y|$.
    \end{enumerate}
    \begin{proof}[Proof of a]
        We divide into four cases ($x\geq 0$ and $y\geq 0$, $x\geq 0$ and $y<0$, $x<0$ and $y\geq 0$, and $x<0$ and $y<0$).\par
        First, suppose that $x\geq 0$ and $y\geq 0$. Then by Definition \ref{dfn:7.21}, $x+y\geq 0$. Thus, by consecutive applications of Definition \ref{dfn:8.4}, $|x+y|=x+y$, $|x|=x$, and $|y|=y$. Therefore, $|x+y|=x+y\leq x+y=|x|+|y|$, as desired.\par
        Second, suppose that $x\geq 0$ and $y<0$. By Definition \ref{dfn:8.4}, $|x|=x$ and $|y|=-y$. We now divide into two cases ($x+y\geq 0$ and $x+y<0$). If $x+y\geq 0$, then $|x+y|=x+y$. Additionally, since $y<0$, Lemma \ref{lem:7.23} implies that $0<-y$. Consequently, by transitivity, $y<-y=|y|$. It follows by Definition \ref{dfn:7.21} that $x+y<x+|y|$. Therefore, $|x+y|=x+y<x+|y|=|x|+|y|$, so $|x+y|\leq|x|+|y|$, as desired. On the other hand, if $x+y<0$, then $|x+y|=-(x+y)=-x+(-y)=-x+|y|$. Additionally, by Lemma \ref{lem:7.23}, $x\geq 0$ implies that $-x\leq 0$. It follows by Definition \ref{dfn:7.21} since $-x\leq x$ that $-x+|y|\leq x+|y|$. Therefore, $|x+y|=-x+|y|\leq x+|y|=|x|+|y|$, as desired.\par
        The proof of the third case is symmetric to that of the second.\par
        The proof of the fourth case is symmetric to that of the first.
    \end{proof}
    \begin{proof}[Proof of b]
        By part (a), $|x-z|=|x-y+y-z|\leq |x-y|+|y-z|$, as desired.
    \end{proof}
    \begin{proof}[Proof of c]
        To prove that $||x|-|y||\leq|x-y|$, Definition \ref{dfn:8.4} tells us that it will suffice to show that $|x|-|y|\leq|x-y|$ and $-(|x|-|y|)\leq|x-y$. By part (a), $|x|=|x-y+y|\leq|x-y|+|y|$, so $|x|-|y|\leq|x-y|$. Similarly, $|y|-|x|\leq|x-y|$, so $-(|x|-|y|)\leq|x-y|$, as desired.
    \end{proof}
\end{lemma}

\begin{exercise}\label{exr:8.9}
    Let $a,\delta\in\R$ with $\delta>0$. Prove that
    \begin{equation*}
        (a-\delta,a+\delta) = \{x\in\R\mid|x-a|<\delta\}
    \end{equation*}
    \begin{lemma*}
        For any $a,b\in\R$ such that $0<b$, $|a|<b$ if and only if $-b<a<b$.
        \begin{proof}
            Suppose first that $|a|<b$. We divide into two cases ($a\geq 0$ and $a<0$). If $a\geq 0$, then by Definition \ref{dfn:8.4}, $0\leq a=|a|<b$. Additionally, by Lemma \ref{lem:7.23}, $-b<0$. Therefore, $-b<0\leq a<b$, as desired. If $a<0$, then by Definition \ref{dfn:8.4}, $-a=|a|<b$. It follows by Definition \ref{dfn:7.21} (by adding $a-b$ to both sides) that $-b<a$. Additionally, by Lemma \ref{lem:7.23}, $a<0$ implies $0<-a$, so we know that $a<-a$. Therefore, $-b<a<-a<b$, as desired.\par
            Now suppose that $-b<a<b$. We divide into two cases ($a\geq 0$ and $a>0$). If $a\geq 0$, then by Definition \ref{dfn:8.4}, $|a|=a<b$, as desired. If $a<0$, then by Definition \ref{dfn:8.4}, $|a|=-a$. Since $-b<a$, Definition \ref{dfn:7.21} implies (by adding $b-a$ to both sides) that $-a<b$. Therefore, $|a|=-a<b$, as desired.
        \end{proof}
    \end{lemma*}
    \begin{proof}[Proof of Exercise \ref{exr:8.9}]
        To prove that $(a-\delta,a+\delta)=\{x\in\R\mid|x-a|<\delta\}$, Definition \ref{dfn:1.2} tells us that it will suffice to show that every $p\in(a-\delta,a+\delta)$ is an element of $\{x\in\R\mid|x-a|<\delta\}$ and vice versa.\par
        Suppose first that $p\in(a-\delta,a+\delta)$. Then by Equations \ref{eqn:8.1}, $a-\delta<p$ and $p<a+\delta$. It follows by consecutive applications of Definition \ref{dfn:7.21} from the former condition that $-\delta<p-a$, and from the latter condition that $p-a<\delta$. Since $-\delta<p-a<\delta$, the lemma asserts that $|p-a|<\delta$. Therefore, $p\in\{x\in\R\mid|x-a|<\delta\}$.\par
        Now suppose that $p\in\{x\in\R\mid|x-a|<\delta\}$. Then $|p-a|<\delta$. Thus, by the lemma, $-\delta<p-a$ and $p-a<\delta$. It follows by consecutive applications of Definition \ref{dfn:7.21} from the former condition that $a-\delta<p$, and from the latter condition that $p<a+\delta$. Therefore, since $a-\delta<p<a+\delta$, we have that $p\in(a-\delta,a+\delta)$.
    \end{proof}
\end{exercise}

\begin{lemma}\label{lem:8.10}\marginnote{\emph{2/11:}}
    Let $I$ be an open interval containing the point $p\in\R$. Then
    \begin{enumerate}[label={\alph*\textup{)}},ref={\thelemma\alph*}]
        \item \label{lem:8.10a}There exists a number $\delta>0$ such that $(p-\delta,p+\delta)\subset I$.
        \item \label{lem:8.10b}There exists a natural number $N$ such that for all natural numbers $k\geq N$ we have $(p-\frac{1}{k},p+\frac{1}{k})\subset I$.
    \end{enumerate}
    \begin{proof}[Proof of a]
        Since $I$ is open, we have by Theorem \ref{trm:4.10} that there exists a region $(a,b)$ such that $p\in(a,b)\subset I$. Let $\delta=\min(p-a,b-p)$. To show that $(p-\delta,p+\delta)\subset I$, we will demonstrate that $(p-\delta,p+\delta)\subset(a,b)\subset I$. To do this, Definition \ref{dfn:1.3} tells us that it will suffice to verify that every element $x\in(p-\delta,p+\delta)$ is an element of $(a,b)$. Let $x$ be an arbitrary element of $(p-\delta,p+\delta)$. Then by Equations \ref{eqn:8.1}, $p-\delta<x<p+\delta$. We divide into two cases ($\delta=p-a$ and $\delta=b-p$). Suppose first that $\delta=p-a$. Then $p-(p-a)<x<p+(p-a)$, i.e., $a<x<p+(p-a)$. Additionally, the fact that $p-a=\min(p-a,b-p)$ implies that $p-a\leq b-p$. Combining these last two results gives us $a<x<p+(p-a)\leq p+(b-p)=b$. Since $a<x<b$, Equations \ref{eqn:8.1} imply that $x\in(a,b)$, as desired. The proof is symmetric if $\delta=b-p$.
    \end{proof}
    \begin{proof}[Proof of b]
        By Lemma \ref{lem:8.10a}, there exists a number $\delta>0$ such that $(p-\delta,p+\delta)\subset I$. Since $\delta$ is a positive real number, Corollary \ref{cly:6.12} implies that there exists a nonzero natural number $N$ such that $\frac{1}{N}<\delta$. To prove that for all numbers $k\geq N$, we have $(p-\frac{1}{k},p+\frac{1}{k})\subset I$, we will show that $(p-\frac{1}{k},p+\frac{1}{k})\subset(p-\delta,p+\delta)\subset I$. To do this, Definition \ref{dfn:1.3} tells us that it will suffice to show that every $x\in(p-\frac{1}{k},p+\frac{1}{k})$ is an element of $(p-\delta,p+\delta)$. Let $k$ be an arbitrary natural number such that $k\geq N$, and let $x$ be an arbitrary element of $(p-\frac{1}{k},p+\frac{1}{k})$. It follows from the latter condition by Equations \ref{eqn:8.1} that $p-\frac{1}{k}<x<p+\frac{1}{k}$. Since $\frac{1}{k}\leq\frac{1}{N}$ by Scripts \ref{sct:2} and \ref{sct:3}, we have that $p-\frac{1}{N}<x<p+\frac{1}{N}$. Since $\frac{1}{N}<\delta$ by definition, $p-\delta<x<p+\delta$. Therefore, by Equations \ref{eqn:8.1}, $x\in(p-\delta,p+\delta)$, as desired.
    \end{proof}
\end{lemma}

\begin{definition}\label{dfn:8.11}
    Let $A\subset X\subset\R$. We say that $A$ is \textbf{open} (in $X$) if it is the intersection of $X$ with an open set, and \textbf{closed} (in $X$) if it is the intersection of $X$ with a closed set. (This is called the subspace topology on $X$.)
\end{definition}

\begin{remark}\label{rmk:8.12}
    $A\subset\R$ open, as defined in Script \ref{sct:3}, is equivalent to $A$ open in $\R$.
\end{remark}

\begin{exercise}\label{exr:8.13}
    Let $A\subset X\subset\R$. Show that $X\setminus A$ is closed in $X$ if and only if $A$ is open in $X$.
    \begin{proof}
        % Suppose first that $X\setminus A$ is closed in $X$. Then by Definition \ref{dfn:8.11}, $X\setminus A=X\cap B$ where $B$ is a closed set. Since $\R\setminus B$ is open by Definition \ref{dfn:4.4}, to prove that $A$ is open in $X$, Definition \ref{dfn:8.11} tells us that it will suffice to show that $A=X\cap(\R\setminus B)$. To do this, Definition \ref{dfn:1.2} tells us that it will suffice to confirm that every $x\in A$ is an element of $X\cap(\R\setminus B)$ and vice versa. First, let $x$ be an arbitrary element of $A$. Then since $A\subset X$, Definition \ref{dfn:1.3} asserts that $x\in X$. Additionally, the fact that $x\in A$ implies by Definition \ref{dfn:1.11} that $x\notin X\setminus A$. Consequently, by Definition \ref{dfn:1.2}, $x\notin X\cap B$, which means that $x\notin B$. This combined with the fact that $x\in\R$ by definition implies by Definition \ref{dfn:1.11} that $x\in\R\setminus B$. Since $x\in X$ and $x\in\R\setminus B$, we have by Definition \ref{dfn:1.6} that $x\in X\cap(\R\setminus B)$, as desired. Now let $x$ be an arbitrary element of $X\cap(\R\setminus B)$. Then by Definitions \ref{dfn:1.6} and \ref{dfn:1.11}, $x\in X$ and $x\notin B$. It follows from the latter condition by Definition \ref{dfn:1.6} that $x\notin X\cap B$, i.e., $x\notin X\setminus A$. But since $x\in X$, if $x\notin X\setminus A$, then we must have $x\in A$, as desired.\par
        Suppose first that $X\setminus A$ is closed in $X$. Then by Definition \ref{dfn:8.11}, $X\setminus A=X\cap B$ where $B$ is a closed set. It follows by Script \ref{sct:1} that
        \begin{align*}
            X\setminus A &= X\cap B\\
            \R\setminus(X\setminus A) &= \R\setminus(X\cap B)\\
            (\R\setminus X)\cup A &= (\R\setminus X)\cup(\R\setminus B)\\
            X\cap((\R\setminus X)\cup A) &= X\cap((\R\setminus X)\cup(\R\setminus B))\\
            (X\cap(\R\setminus X))\cup(X\cap A) &= (X\cap(\R\setminus X))\cup(X\cap(\R\setminus B))\\
            \emptyset\cup(X\cap A) &= \emptyset\cup(X\cap(\R\setminus B))\\
            A &= X\cap(\R\setminus B)
        \end{align*}
        Since $\R\setminus B$ is open by Definition \ref{dfn:4.4}, we have by Definition \ref{dfn:8.11} that $A$ is open in $X$.\par
        Now suppose that $A$ is open in $X$. Then by Definition \ref{dfn:8.11}, $A=X\cap B$ where $B$ is an open set. It follows by Script \ref{sct:1} that
        \begin{align*}
            A &= X\cap B\\
            \R\setminus A &= \R\setminus(X\cap B)\\
            &= (\R\setminus X)\cup(\R\setminus B)\\
            X\cap(\R\setminus A) &= X\cap((\R\setminus X)\cup(\R\setminus B))\\
            X\setminus A &= (X\cap(\R\setminus X))\cup(X\cap(\R\setminus B))\\
            &= X\cap(\R\setminus B)
        \end{align*}
        Since $\R\setminus B$ is closed by Definition \ref{dfn:4.4}, we have by Definition \ref{dfn:8.11} that $X\setminus A$ is closed in $X$.
    \end{proof}
\end{exercise}

\begin{exercise}\label{exr:8.14}\leavevmode
    \begin{enumerate}[label={\alph*)},ref={\theexercise\alph*}]
        \item \label{exr:8.14a}Let $[a,b]\subset\R$. Give an example of a set $A\subset[a,b]$ such that $A$ is open in $[a,b]$ but not in $\R$.
        \item \label{exr:8.14b}Give an example of sets $A\subset X\subset\R$ such that $A$ is closed in $X$ but not in $\R$.
    \end{enumerate}
    \begin{proof}[Proof of a]
        We first briefly consider the case where $a=b$. In this case, let $c<a<d$; then $\{a\}=[a,a]\cap(c,d)$ is a subset of $[a,b]$ that is open in $[a,b]$ (by Definition \ref{dfn:8.11}; $(c,d)$ is open by Exercise \ref{exr:8.1}) but closed in $\R$ (by Corollary \ref{cly:3.23}, Definition \ref{dfn:4.1}, and Theorem \ref{trm:5.1}).\par
        We now direct our attention to the case where $a\neq b$. Let $c\in[a,b]$ be a point such that $a<c<b$ (we know at least one such point exists by Theorem \ref{trm:5.2}). If we define the set $(c,b]=[a,b]\cap(c,\infty)$, we have by Definition \ref{dfn:8.11} that $(c,b]$ is open in $[a,b]$ (since $(c,\infty)$ is open as per Exercise \ref{exr:8.1}). However, we know that $(c,b]$ is not open in $\R$ by Theorem \ref{trm:4.10} ($b$ is an element of $(c,b]$ such that any region containing $b$ necessarily contains an element that is not in $(c,b]$; this element will be greater than $b$ but less than the right bound of the region, and its existence is guaranteed by Theorem \ref{trm:5.2}).
    \end{proof}
    \begin{proof}[Proof of b]
        Let $X=(a,b)\subset\R$. Then $(a,b)=X\cap[a,b]$, so $(a,b)=X\cap[a,b]$ is closed in $(a,b)$ by Definition \ref{dfn:8.11}. However, by Corollary \ref{cly:5.14}, $a,b$ are limit points of $(a,b)$ that are not contained within $(a,b)$. It follows by Definition \ref{dfn:4.1} that $(a,b)$ is not closed in $\R$.
    \end{proof}
\end{exercise}

\begin{theorem}\label{trm:8.15}
    Let $X\subset\R$. Then $X$ is connected if and only if $X$ is an interval.
    \begin{proof}
        Suppose first that $X$ is connected. To prove that $X$ is an interval, Definition \ref{dfn:8.2} tells us that it will suffice to show that for all $x,y\in X$ with $x<y$, $[x,y]\subset X$. Let $x,y$ be arbitrary elements of $X$ satisfying $x<y$, and suppose for the sake of contradiction that $[x,y]\not\subset X$. Then there exists $z\in[x,y]$ such that $z\notin X$. Let $A=\{a\in X\mid a<z\}$ and $B=\{b\in X\mid z<b\}$. It follows from Script \ref{sct:1} that $X=A\cup B$ and $A\cap B=\emptyset$. To verify that $A$ is nonempty, Definition \ref{dfn:1.8} tells us that it will suffice to find an element in it. Since $z\notin X$ but $x\in X$, we know that $z\neq x$. This combined with the fact that $x\leq z$ by Equations \ref{eqn:8.1} implies that $x<z$. Thus, since $x\in X$ and $x<z$, $x\in A$. Similarly, $y\in B$. To verify that $A$ is open in $X$, Definition \ref{dfn:8.11} tells us that it will suffice to demonstrate that $A$ is the intersection of $X$ with an open set. Since we clearly have $A=X\cap(-\infty,z)$ where $(-\infty,z)$ is open by Exercise \ref{exr:8.1}, we are done. We can do something similar for $B$. But the existence of two disjoint, nonempty, open (in $X$) sets $A,B$ whose union equals $X$ demonstrates by Definition \ref{dfn:4.22} that $X$ is disconnected, a contradiction.\par\medskip
        % \begin{itemize}
        %     \item Setup:
        %     \begin{itemize}
        %         \item Suppose $X$ is an interval.
        %         \item Suppose (contradiction): $X$ is disconnected.
        %         \item Definition \ref{dfn:4.22}: $X=A\cup B$ where $A,B$ are disjoint, nonempty, and open in $X$.
        %         \item $A,B$ nonempty, disjoint: $a\in A$ and $b\in B$ such that $a\neq b$.
        %         \item WLOG, let $a<b$.
        %     \end{itemize}
        %     \item Prove $\sup(A\cap[a,b])$ exists (Theorem \ref{trm:5.17}: Show $A\cap[a,b]$ is nonempty and bounded above).
        %     \begin{itemize}
        %         \item Show nonempty (Definition \ref{dfn:1.8}: Find an element of $A\cap[a,b]$).
        %         \begin{itemize}
        %             \item Equations \ref{eqn:8.1}: $a\in[a,b]$.
        %             \item By definition: $a\in A$.
        %             \item Definition \ref{dfn:1.6}: $a\in A\cap[a,b]$, as desired.
        %         \end{itemize}
        %         \item Show bounded above (consecutive Definition \ref{dfn:5.6}: Verify that $x\leq b$ for all $x\in A\cap[a,b]$).
        %         \begin{itemize}
        %             \item Let $x\in A\cap[a,b]$.
        %             \item Definition \ref{dfn:1.6}: $x\in[a,b]$.
        %             \item Equations \ref{eqn:8.1}: $x\leq b$, as desired.
        %         \end{itemize}
        %     \end{itemize}
        %     \item Let $s=\sup(A\cap[a,b])$.
        %     \begin{itemize}
        %         \item Prove $\inf(B\cap[s,b])$ exists (a symmetric argument to the previous).
        %     \end{itemize}
        %     \item Let $i=\inf(B\cap[s,b])$.
        %     \begin{itemize}
        %         \item We divide into three cases ($s>i$, $s=i$, and $s<i$).
        %     \end{itemize}
        %     \item First, suppose: $s>i$.
        %     \begin{itemize}
        %         \item Show $s$ is a lower bound of $B\cap[s,b]$ (Definition \ref{dfn:5.6}: Verify that $s\leq x$ for all $x\in B\cap[s,b]$).
        %         \begin{itemize}
        %             \item Let $x\in B\cap[s,b]$.
        %             \item Definition \ref{dfn:1.6}: $x\in [s,b]$.
        %             \item Equations \ref{eqn:8.1}: $s\leq x$, as desired.
        %         \end{itemize}
        %         \item Definition \ref{dfn:5.7}: $i\geq l$ for all lower bounds $l$, including $s$.
        %         \item Therefore: $i\geq s$, contradicting the hypothesis that $s>i$.
        %     \end{itemize}
        %     \item Second, suppose: $s=i$.
        %     \begin{itemize}
        %         \item We divide into three cases ($s\in A$, $s\in B$, and $s\notin A$ and $s\notin B$).
        %         \item If $s\in A$, then\dots
        %         \begin{itemize}
        %             \item Since $A$ open in $X$ (+ Definition \ref{dfn:8.11}): $A=X\cap G$ where $G$ is open.
        %             \item $s\in A$ (+ Definitions \ref{dfn:1.2} and \ref{dfn:1.6}): $s\in G$.
        %             \item Theorem \ref{trm:4.10}: There exists a region $(c,d)$ such that $s\in(c,d)$ and $(c,d)\subset G$.
        %             \item Equations \ref{eqn:8.1}: $i=s<d$.
        %             \item Lemma \ref{lem:5.11}: There exists a point $x\in B\cap[s,b]$ such that $i\leq x<d$.
        %             \item $c<s\leq x<d$ (+ Equations \ref{eqn:8.1}): $x\in(c,d)$.
        %             \item $(c,d)\subset G$ (+ Definition \ref{dfn:1.3}): $x\in G$.
        %             \item $x\in B\cap[s,b]$ (+ Definition \ref{dfn:1.6}): $x\in B$.
        %             \item $X=A\cup B$ (+ Definitions \ref{dfn:1.5} and \ref{dfn:1.2}): $x\in X$.
        %             \item $x\in G$ and $x\in X$ (+ Definition \ref{dfn:1.6}): $x\in X\cap G$.
        %             \item $A=X\cap G$ (+ Definition \ref{dfn:1.2}): $x\in A$.
        %             \item $x\in A$, $x\in B$ (+ Definition \ref{dfn:1.6}) $x\in A\cap B$, contradicting the supposition that $A,B$ are disjoint.
        %         \end{itemize}
        %         \item If $s\in B$, then the proof is symmetric to the previous case.
        %         \item If $s\notin A$ and $s\notin B$, then\dots
        %         \begin{itemize}
        %             \item Definition \ref{dfn:1.5}: $s\notin A\cup B$.
        %             \item Definition \ref{dfn:1.2}: $s\notin X$.
        %             \item $a\in A$, $b\in B$, $X=A\cup B$: $a,b\in X$.
        %             \item $a,b\in X$, $a<b$ (+ Definition \ref{dfn:8.2}): $[a,b]\subset X$.
        %             \item Show $s\in[a,b]$ (Equations \ref{eqn:8.1}: Verify that $a\leq s\leq b$).
        %             \begin{itemize}
        %                 \item As previously shown: $b$ is an upper bound of $A\cap[a,b]$.
        %                 \item Definition \ref{dfn:5.7}: $s\leq b$.
        %                 \item As previously shown: $a\in A\cap[a,b]$.
        %                 \item Definitions \ref{dfn:5.7} and \ref{dfn:5.6}: $s\geq x$ for all $x\in A\cap[a,b]$, including $a$.
        %                 \item Thus, $s\geq a$.
        %             \end{itemize}
        %             \item Definition \ref{dfn:1.3}: $s\in X$, a contradiction.
        %         \end{itemize}
        %     \end{itemize}
        %     \item Third, suppose: $s<i$.
        %     \begin{itemize}
        %         \item Theorem \ref{trm:5.2}: There exists $z\in\R$ such that $s<z<i$.
        %         \item Show $i\in[a,b]$ (Equations \ref{eqn:8.1}: Verify that $a\leq i\leq b$).
        %         \begin{itemize}
        %             \item As previously shown: $s$ is a lower bound of $B\cap[s,b]$.
        %             \item Definition \ref{dfn:5.7}: $i\geq s$.
        %             \item $s\geq a$ (as previously shown + transitivity): $i\geq a$.
        %             \item Confirm $b\in B\cap[s,b]$.
        %             \begin{itemize}
        %                 \item Equations \ref{eqn:8.1}: $b\in[s,b]$.
        %                 \item By definition: $b\in B$.
        %                 \item Definition \ref{dfn:1.6}: $b\in B\cap[s,b]$.
        %             \end{itemize}
        %             \item Definitions \ref{dfn:5.7} and \ref{dfn:5.6}: $i\leq x$ for all $x\in B\cap[s,b]$, including $b$.
        %             \item Thus, $i\leq b$.
        %         \end{itemize}
        %         \item $s<z$ (+ Definition \ref{dfn:5.6}): $z\notin A\cap[a,b]$.
        %         \item $a\leq s<z<i\leq b$: $z\in[a,b]$.
        %         \item $z\in[a,b]$ (+ Definition \ref{dfn:1.6}): $z\notin A$.
        %         \item $z<i$ (+ Definition \ref{dfn:5.6}): $z\notin B\cap[s,b]$.
        %         \item $s<z<i\leq b$: $z\in[s,b]$.
        %         \item $z\in[s,b]$ (+ Definition \ref{dfn:1.6}): $z\notin B$.
        %         \item $z\notin A$, $z\notin B$ (+ Definition \ref{dfn:1.5}): $z\notin A\cup B=X$.
        %         \item As before: $[a,b]\subset X$.
        %         \item Definition \ref{dfn:1.3}: $z\in X$, a contradiction.
        %     \end{itemize}
        % \end{itemize}
        Now suppose that $X$ is an interval, and suppose for the sake of contradiction that $X$ is disconnected. Then by Definition \ref{dfn:4.22}, $X=A\cup B$ where $A,B$ are disjoint, nonempty sets that are open in $X$. Since $A,B$ are disjoint and nonempty, we know that there exist distinct objects $a\in A$ and $b\in B$. WLOG, let $a<b$.\par\smallskip
        To prove that $\sup(A\cap[a,b])$ exists, Theorem \ref{trm:5.17} tells us that it will suffice to show that $A\cap[a,b]$ is nonempty and bounded above. To show that $A\cap[a,b]$ is nonempty, Definition \ref{dfn:1.8} tells us that it will suffice to find an element of $A\cap[a,b]$. By Equations \ref{eqn:8.1}, $a\in[a,b]$. By Definition, $a\in A$. Thus, by Definition \ref{dfn:1.6}, $a\in A\cap[a,b]$, as desired. To show that $A\cap[a,b]$ is bounded above, consecutive applications of Definition \ref{dfn:5.6} tell us that it will suffice to verify that $x\leq b$ for all $x\in A\cap[a,b]$. Let $x$ be an arbitrary element of $A\cap[a,b]$. It follows by Definition \ref{dfn:1.6} that $x\in[a,b]$. Thus, by Equations \ref{eqn:8.1}, $x\leq b$, as desired.\par\smallskip
        Let $s=\sup(A\cap[a,b])$. To prove that $\inf(B\cap[s,b])$ exists, it will suffice to utilize a symmetric argument to the above.\par\smallskip
        Let $i=\inf(B\cap[s,b])$. We divide into three cases ($s>i$, $s=i$, and $s<i$).\par\smallskip
        First, suppose that $s>i$. To show that $s$ is a lower bound of $B\cap[s,b]$, Definition \ref{dfn:5.6} tells us that it will suffice to verify that $s\leq x$ for all $x\in B\cap[s,b]$. Let $x$ be an arbitrary element of $B\cap[s,b]$. By Definition \ref{dfn:1.6}, $x\in[s,b]$. Thus, by Equations \ref{eqn:8.1}, $s\leq x$, as desired. Since $s$ is a lower bound of $B\cap[s,b]$, Definition \ref{dfn:5.7} asserts that $i\geq s$, contradicting the hypothesis that $s>i$.\par\smallskip
        Second, suppose that $s=i$. We divide into three cases ($s\in A$, $s\in B$, and $s\notin A$ and $s\notin B$).\par
        If $s\in A$, then since $A$ is open in $X$, Definition \ref{dfn:8.11} implies that $A=X\cap G$ where $G$ is open. It follows by the hypothesis that $s\in A$ along with Definitions \ref{dfn:1.2} and \ref{dfn:1.6} that $s\in G$. Consequently, by Theorem \ref{trm:4.10}, there exists a region $(c,d)$ such that $s\in(c,d)$ and $(c,d)\subset G$. From the former condition, we have by Equations \ref{eqn:8.1} that $c<s<d$. Thus, by Lemma \ref{lem:5.11}, there exists a point $x\in B\cap[s,b]$ such that $s=i\leq x<d$. Since $c<s\leq x<d$, Equations \ref{eqn:8.1} imply that $x\in(c,d)$. This combined with the fact that $(c,d)\subset G$ implies by Definition \ref{dfn:1.3} that $x\in G$. Additionally, we know that $x\in B$ (since $x\in B\cap[s,b]$ by Definition \ref{dfn:1.6}). It follows from this and the fact that $X=A\cup B$ by Definitions \ref{dfn:1.5} and \ref{dfn:1.2} that $x\in X$. Thus, since $x\in X$ and $x\in G$, Definition \ref{dfn:1.6} asserts that $x\in X\cap G$, meaning that $x\in A$. But if $x\in A$ and $x\in B$, then Definition \ref{dfn:1.6} implies that $x\in A\cap B$, contradicting the supposition that $A$ and $B$ are disjoint.\par
        If $s\in B$, then the proof is symmetric to the previous case.\par
        If $s\notin A$ and $s\notin B$, then by Definition \ref{dfn:1.5}, $s\notin A\cup B$, implying that $s\notin X$. Additionally, the facts that $a\in A$, $b\in B$, and $X=A\cup B$ imply that $a,b\in X$. It follows since $a<b$ by Definition \ref{dfn:8.2} that $[a,b]\subset X$. We now show that $s\in[a,b]$ via Equations \ref{eqn:8.1}, which tell us that it will suffice to verify that $a\leq s\leq b$. As previously shown, $b$ is an upper bound of $A\cap[a,b]$. Thus, by Definition \ref{dfn:5.7}, we have that $s\leq b$, and we are half done. As to the other half, we have also previously shown that $a\in A\cap[a,b]$. Additionally, by Definitions \ref{dfn:5.7} and \ref{dfn:5.6}, $s\geq x$ for all $x\in A\cap[a,b]$, including $a$. Thus, $s\geq a$. Having shown that $s\in[a,b]$ and $[a,b]\subset X$, we may invoke Definition \ref{dfn:1.3} to learn that $s\in X$, contradicting the previously proven statement that $s\notin X$.\par\smallskip
        Third, suppose that $s<i$. Then by Theorem \ref{trm:5.2} and Definition \ref{dfn:3.6}, there exists a $z\in\R$ such that $s<z<i$. We now show that $i\in[a,b]$ via Equations \ref{eqn:8.1}, which tell us that it wil suffice to verify that $a\leq i\leq b$. As previously shown, $s$ is a lower bound of $B\cap[s,b]$. Thus, by Definition \ref{dfn:5.7}, we have that $i\geq s$. We have also previously shown that $s\geq a$, so by transitivity, $i\geq a$, and we are half done. As to the other half, we now confirm that $b\in B\cap[s,b]$. By Equations \ref{eqn:8.1}, $b\in[s,b]$. By definition, $b\in B$. Thus, by Definition \ref{dfn:1.6}, $b\in B\cap[s,b]$, as desired. Additionally, by Definitions \ref{dfn:5.7} and \ref{dfn:5.6}, $i\leq x$ for all $x\in B\cap[s,b]$, including $b$. Thus, $i\leq b$, concluding our argument that $i\in[a,b]$. Moving on, the fact that $s<z$ implies by Definition \ref{dfn:5.6} that $z\notin A\cap[a,b]$. Additionally, we know from the facts that $s,i\in[a,b]$ that $a\leq s<z<i\leq b$, meaning that $z\in[a,b]$. Combining the previous two results with Definition \ref{dfn:1.6}, we have that $z\notin A$. By a symmetric argument, we can show that $z\notin B$. Since $z\notin A$ and $z\notin B$, Definition \ref{dfn:1.5} asserts that $z\notin A\cup B$, i.e., $z\notin X$. But as before, $[a,b]\subset X$, so the fact that $z\in[a,b]$ combined with Definition \ref{dfn:1.3} implies that $z\in X$, a contradiction.
    \end{proof}
\end{theorem}

\begin{definition}\label{dfn:8.16}\marginnote{2/16:}
    Let $I$ be an  interval and let $f:I\to\R$.
    \begin{enumerate}[label={\alph*)}]
        \item We say that $f$ is \textbf{increasing} on $I$ if, whenever $x,y\in I$ with $x<y$, $f(x)\leq f(y)$.
        \item We say that $f$ is \textbf{decreasing} on $I$ if, whenever $x,y\in I$ with $x<y$, $f(x)\geq f(y)$.
        \item We say that $f$ is \textbf{strictly increasing} on $I$ if, whenever $x,y\in I$ with $x<y$, $f(x)<f(y)$.
        \item We say that $f$ is \textbf{strictly decreasing} on $I$ if, whenever $x,y\in I$ with $x<y$, $f(x)>f(y)$.
    \end{enumerate}
\end{definition}

\begin{lemma}\label{lem:8.17}
    If $f$ is strictly increasing or strictly decreasing on an interval $I$ then $f$ is injective on $I$.
    \begin{proof}
        We divide into two cases ($f$ is strictly increasing, and $f$ is strictly decreasing). Suppose first that $f$ is strictly increasing. To prove that $f$ is injective on $I$, Definition \ref{dfn:1.20} tells us that it will suffice to show that for all $a,b\in I$, $a\neq b$ implies that $f(a)\neq f(b)$. Let $a,b$ be arbitrary elements of $I$ such that $a\neq b$. WLOG, let $a<b$. Then by Definition \ref{dfn:8.16}, $f(a)<f(b)$. Therefore, $f(a)\neq f(b)$, as desired. The proof is symmetric for the other case.
    \end{proof}
\end{lemma}




\end{document}