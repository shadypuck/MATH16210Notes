\documentclass[../main.tex]{subfiles}

\pagestyle{main}
\renewcommand{\chaptermark}[1]{\markboth{\chaptername\ \thechapter}{}}
\setcounter{chapter}{6}

\begin{document}




\chapter{The Field Axioms}\label{sct:7}
\section{Journal}
\begin{definition}\label{dfn:7.1}\marginnote{1/28:}
    A \textbf{binary operation} on a set $X$ is a function
    \begin{equation*}
        f:X\times X\to X
    \end{equation*}
    We say that $f$ is \textbf{associative} if
    \begin{center}
        $f(f(x,y),z)=f(x,f(y,z))$\quad for all $x,y,z\in X$
    \end{center}
    We say that $f$ is \textbf{commutative} if
    \begin{center}
        $f(x,y)=f(y,x)$\quad for all $x,y\in X$
    \end{center}
    An \textbf{identity element} of a binary operation $f$ is an element $e\in X$ such that
    \begin{center}
        $f(x,e)=f(e,x)=x$\quad for all $x\in X$
    \end{center}
\end{definition}

\begin{remark}\label{rmk:7.2}
    Frequently, we denote a binary operation differently. If $*:X\times X\to X$ is the binary operation, we often write $a*b$ in place of $*(a,b)$. We sometimes indicate this same operation by writing $(a,b)\mapsto a*b$.
\end{remark}

\begin{exercise}\label{exr:7.3}
    Rewrite Definition \ref{dfn:7.1} using the notation of Remark \ref{rmk:7.2}.
    \begin{proof}[Answer]
        A \textbf{binary operation} on a set $X$ is a function
        \begin{equation*}
            *:X\times X\to X
        \end{equation*}
        We say that $*$ is \textbf{associative} if
        \begin{center}
            $(x*y)*z=x*(y*z)$\quad for all $x,y,z\in X$
        \end{center}
        We say that $*$ is \textbf{commutative} if
        \begin{center}
            $x*y=y*x$\quad for all $x,y\in X$
        \end{center}
        An \textbf{identity element} of a binary operation $*$ is an element $e\in X$ such that
        \begin{center}
            $x*e=e*x=x$\quad for all $x\in X$
        \end{center}
    \end{proof}
\end{exercise}

\begin{examples}\label{xml:7.4}\leavevmode
    \begin{enumerate}
        \item The function $+:\Z\times\Z\to\Z$ which sends a pair of integers $(m,n)$ to $+(m,n)=m+n$ is a binary operation on the integers, called addition. Addition is associative, commutative, and has identity element 0.
        \item The maximum of $m$ and $n$, denoted $\max(m,n)$, is an associative and commutative binary operation on $\Z$. Is there an identity element for $\max$?
        \begin{proof}
            Suppose for the sake of contradiction that there exists an identity element $e$ for $\max$. But $\max(e-1,e)=e\neq e-1$, a contradiction. Therefore, no identity element exists for $\max$.
        \end{proof}
        \item Let $\wp(Y)$ be the power set of a set $Y$. Recall that the power set consists of all subsets of $Y$. Then the intersection of sets, $(A,B)\mapsto A\cap B$, defines an associative and commutative binary operation on $\wp(Y)$. Is there an identity element for $\cap$?
        \begin{proof}
            Clearly, $Y\in\wp(Y)$. By Script \ref{sct:1}, $Y\cap A=A\cap Y=A$ where $A\subset Y$. Therefore, $Y$ is an identity element for $\cap$.
        \end{proof}
    \end{enumerate}
\end{examples}

\begin{exercise}\label{exr:7.5}
    Find a binary operation on a set that is not commutative. Find a binary operation on a set that is not associative.
    \begin{proof}
        We will prove that the subtraction operation on the integers ($-:\Z\times\Z\to\Z$) is not commutative or associative. To prove that it's not commutative, Definition \ref{dfn:7.1} tells us that it will suffice to show that $x-y\neq y-x$ for some $x,y\in\Z$. Since $2-1=1$ but $1-2=-1$, we can see that $1,2\in\Z$ clearly meet this requirement. To prove that it's not associative, Definition \ref{dfn:7.1} tells us that it will suffice to show that $(x-y)-z\neq x-(y-z)$ for some $x,y,z\in\Z$. Since $(3-2)-1=0$ but $3-(2-1)=2$, we can see that $1,2,3\in\Z$ clearly meet this requirement.
    \end{proof}
\end{exercise}

\begin{exercise}\label{exr:7.6}
    Let $X$ be a finite set, and let $Y=\{f:X\to X\mid f\text{ is bijective}\}$. Consider the binary operation of composition of functions, denoted $\circ:Y\times Y\to Y$ and defined by $(f\circ g)(x)=f(g(x))$ as seen in Definition \ref{dfn:1.25}. Decide whether or not composition is commutative and/or associative and whether or not it has an identity.
    \begin{proof}
        To prove that composition is not commutative, Definition \ref{dfn:7.1} tells us that it will suffice to find a finite set $X$ paired with two bijections in $Y$ that do not commute. Let $X=\{1,2,3\}$ and consider the bijections $f:X\to X$ (defined by $f(1)=2$, $f(2)=3$, $f(3)=1$) and $g:X\to X$ (defined by $g(1)=1$, $g(2)=3$, $g(3)=2$). In this case, $f\circ g$ would be defined by $f(g(1))=2$, $f(g(2))=1$, and $f(g(3))=3$, but $g\circ f$ would be defined by $g(f(1))=3$, $g(f(2))=2$, and $g(f(3))=1$.\par
        To prove that composition is associative, Definition \ref{dfn:7.1} tells us that it will suffice to show that $((f\circ g)\circ h)(x)=(f\circ(g\circ h))(x)$. We may do this with the following algebra.
        \begin{align*}
            ((f\circ g)\circ h)(x) &= (f\circ g)(h(x))\\
            &= f(g(h(x)))\\
            &= f((g\circ h)(x))\\
            &= (f\circ(g\circ h))(x)
        \end{align*}
        With respect to any finite set $X$, there will always be a bijection $i:X\to X$ defined by $i(x)=x$. To prove that $i$ is an identity element, Definition \ref{dfn:7.1} tells us that it will suffice to show that for all $f\in Y$, $f\circ i=i\circ f=f$. We may do this with the following algebra.
        \begin{align*}
            (f\circ i)(x) &= f(i(x))\\
            &= f(x)\\
            &= i(f(x))\\
            &= (i\circ f)(x)
        \end{align*}
    \end{proof}
\end{exercise}

\begin{theorem}\label{trm:7.7}
    Identity elements are unique. That is, suppose that $f$ is a binary operation on a set $X$ that has two identity elements $e$ and $e'$. Then $e=e'$.
    \begin{proof}
        Let $f:X\times X\to X$ be a binary operation on a set $X$ with two identity elements $e,e'$. By Definition \ref{dfn:7.1}, we know that $f(e,e')=e$ and $f(e,e')=e'$. Since $f$ is a well-defined function by definition, it must be that $e=f(e,e')=e'$.
    \end{proof}
\end{theorem}

\begin{definition}\label{dfn:7.8}
    A \textbf{field} is a set $F$ with two binary operations on $F$ called addition, denoted $+$, and multiplication, denoted $\cdot$, satisfying the following \textbf{field axioms}:
    \begin{enumerate}[label={FA\arabic*}]
        \item (Commutativity of Addition) For all $x,y\in F$, $x+y=y+x$.
        \item (Associativity of Addition) For all $x,y,z\in F$, $(x+y)+z=x+(y+z)$.
        \item (Additive Identity) There exists an element $0\in F$ such that $x+0=0+x=x$ for all $x\in F$.
        \item (Additive Inverses) For any $x\in F$, there exists $y\in F$ such that $x+y=y+x=0$, called an additive inverse of $x$.
        \item (Commutativity of Multiplication) For all $x,y\in F$, $x\cdot y=y\cdot x$.
        \item (Associativity of Multiplication) For all $x,y,z\in F$, $(x\cdot y)\cdot z=x\cdot(y\cdot z)$.
        \item (Multiplicative Identity) There exists an element $1\in F$ such that $x\cdot 1=1\cdot x=x$ for all $x\in F$.
        \item (Multiplicative Inverses) For any $x\in F$ such that $x\neq 0$, there exists $y\in F$ such that $x\cdot y=y\cdot x=1$, called a multiplicative inverse of $x$.
        \item (Distributivity of Multiplication over Addition) For all $x,y,z\in F$, $x\cdot(y+z)=x\cdot y+x\cdot z$.
        \item (Distinct Additive and Multiplicative Identities) $1\neq 0$.
    \end{enumerate}
\end{definition}

\begin{exercise}\label{exr:7.9}
    Consider the set $\F_2=\{0,1\}$, and define binary operations $+$ and $\cdot$ on $\F_2$ by
    \begin{align*}
        \begin{split}
            0+0 = 0\\
            0\cdot 0 = 0
        \end{split}&&
        \begin{split}
            0+1 = 1\\
            0\cdot 1 = 0
        \end{split}&&
        \begin{split}
            1+0 = 1\\
            1\cdot 0 = 0
        \end{split}&&
        \begin{split}
            1+1 = 0\\
            1\cdot 1 = 1
        \end{split}
    \end{align*}
    Show that $\F_2$ is a field.
    \begin{proof}
        To prove that $\F_2$ obeys FA1 from Definition \ref{dfn:7.8}, it will suffice to show that $0+0=0+0$, $0+1=1+0$, and $1+1=1+1$. The first and third of these are evidently true. For the second, we have $0+1=1=1+0$, so it is good, too.\par
        To prove that $\F_2$ obeys FA2 from Definition \ref{dfn:7.8}, the following casework will suffice.
        \begin{align*}
            (0+0)+0 &= 0 = 0+(0+0)&
            (0+0)+1 &= 1 = 0+(0+1)\\
            (0+1)+0 &= 1 = 0+(1+0)&
            (1+0)+0 &= 1 = 1+(0+0)\\
            (0+1)+1 &= 0 = 0+(1+1)&
            (1+1)+0 &= 0 = 1+(1+0)\\
            (1+0)+1 &= 0 = 1+(0+1)&
            (1+1)+1 &= 1 = 1+(1+1)
        \end{align*}
        To prove that $\F_2$ obeys FA3 from Definition \ref{dfn:7.8}, it will suffice to find an element $0\in\F_2$ such that $x+0=0+x=x$. Since $0+0=0$, $1+0=0$, and with commutativity, it is clear that 0 is an additive identity in $\F_2$.\par
        To prove that $\F_2$ obeys FA4 from Definition \ref{dfn:7.8}, it will suffice to show that for all $x\in\F_2$, there exists a $y\in\F_2$ such that $x+y=y+x=0$. For 0, this object is 0 (since $0+0=0+0=0$), and for 1, this object is 1 (since $1+1=1+1=0$).\par
        To prove that $\F_2$ obeys FA5 from Definition \ref{dfn:7.8}, it will suffice to show that $0\cdot 0=0\cdot 0$, $0\cdot 1=1\cdot 0$, and $1\cdot 1=1\cdot 1$. The first and third of these are evidently true. For the second, we have $0\cdot 1=0=1\cdot 0$, so it is good, too.\par
        To prove that $\F_2$ obeys FA6 from Definition \ref{dfn:7.8}, the following casework will suffice.
        \begin{align*}
            (0\cdot 0)\cdot 0 &= 0 = 0\cdot (0\cdot 0)&
            (0\cdot 0)\cdot 1 &= 0 = 0\cdot (0\cdot 1)\\
            (0\cdot 1)\cdot 0 &= 0 = 0\cdot (1\cdot 0)&
            (1\cdot 0)\cdot 0 &= 0 = 1\cdot (0\cdot 0)\\
            (0\cdot 1)\cdot 1 &= 0 = 0\cdot (1\cdot 1)&
            (1\cdot 1)\cdot 0 &= 0 = 1\cdot (1\cdot 0)\\
            (1\cdot 0)\cdot 1 &= 0 = 1\cdot (0\cdot 1)&
            (1\cdot 1)\cdot 1 &= 1 = 1\cdot (1\cdot 1)
        \end{align*}
        To prove that $\F_2$ obeys FA7 from Definition \ref{dfn:7.8}, it will suffice to find an element $1\in\F_2$ such that $x\cdot 1=1\cdot x=x$. Since $0\cdot 1=0$, $1\cdot 1=1$, and with commutativity, it is clear that 1 is a multiplicative identity in $\F_2$.\par
        To prove that $\F_2$ obeys FA8 from Definition \ref{dfn:7.8}, it will suffice to show that for all $x\in\F_2$ such that $x\neq 0$, there exists a $y\in\F_2$ such that $x\cdot y=y\cdot x=1$. For 1, this object is 1 (since $1\cdot 1=1\cdot 1=1$).\par
        To prove that $\F_2$ obeys FA9 from Definition \ref{dfn:7.8}, the following casework will suffice.
        \begin{align*}
            0\cdot(0+0) &= 0 = 0\cdot 0+0\cdot 0&
            0\cdot(0+1) &= 0 = 0\cdot 0+0\cdot 1\\
            0\cdot(1+0) &= 0 = 0\cdot 1+0\cdot 0&
            1\cdot(0+0) &= 0 = 1\cdot 0+1\cdot 0\\
            0\cdot(1+1) &= 0 = 0\cdot 1+0\cdot 1&
            1\cdot(1+0) &= 1 = 1\cdot 1+1\cdot 0\\
            1\cdot(0+1) &= 1 = 1\cdot 0+1\cdot 1&
            1\cdot(1+1) &= 0 = 1\cdot 1+1\cdot 1
        \end{align*}
        To prove that $\F_2$ obeys FA10 from Definition \ref{dfn:7.8}, it will suffice to show that $0\neq 1$. Clearly this is true.
    \end{proof}
\end{exercise}

\begin{theorem}\label{trm:7.10}
    Suppose that $F$ is a field. Then additive inverses are unique. This means: Let $x\in F$. If $y,y'\in F$ satisfy $x+y=0$ and $x+y'=0$, then $y=y'$.
    \begin{proof}
        Let $x,y,y'\in F$ be such that $x+y=0$ and $x+y'=0$. From Definition \ref{dfn:7.8}, we have
        \begin{align*}
            y'+(x+y) &= (y'+x)+y\tag*{FA2}\\
            y'+0 &= 0+y\tag*{FA4}\\
            y' &= y\tag*{FA3}
        \end{align*}
    \end{proof}
\end{theorem}

We usually write $-x$ for the additive inverse of $x$.

\begin{corollary}\label{cly:7.11}
    If $x\in F$, then $-(-x)=x$.
    \begin{proof}
        Let $x\in F$. Then by consecutive applications of FA4 from Definition \ref{dfn:7.8}, $-x+(-(-x))=0$ and $-x+x=0$. Therefore, by Theorem \ref{trm:7.10}, we have that $-(-x)=x$.
    \end{proof}
\end{corollary}

\begin{theorem}\label{trm:7.12}
    Let $F$ be a field, and let $a,b,c\in F$. If $a+b=a+c$, then $b=c$.
    \begin{proof}
        Let $a,b,c\in F$ be such that $a+b=a+c$. By FA4 from Definition \ref{dfn:7.8}, there exists $-a\in F$ such that $-a+a=a+(-a)=0$. Having established that $-a$ exists, we can prove from Definition \ref{dfn:7.8} that
        \begin{align*}
            -a+(a+b) &= -a+(a+c)\\
            (-a+a)+b &= (-a+a)+c\tag*{FA2}\\
            0+b &= 0+c\tag*{FA4}\\
            b &= c\tag*{FA3}
        \end{align*}
    \end{proof}
\end{theorem}

\begin{theorem}\label{trm:7.13}
    Let $F$ be a field. If $a\in F$, then $a\cdot 0=0$.
    \begin{proof}
        Let $a\in F$. From Definition \ref{dfn:7.8}, we have
        \begin{align*}
            a &= a\cdot 1\tag*{FA7}\\
            &= a\cdot(1+0)\tag*{FA3}\\
            &= a\cdot 1+a\cdot 0\tag*{FA9}\\
            &= a+a\cdot 0\tag*{FA7}\\
            0 &= a\cdot 0\tag*{Theorem \ref{trm:7.12}}
        \end{align*}
    \end{proof}
\end{theorem}

\begin{theorem}\label{trm:7.14}\marginnote{\emph{2/2:}}
    Suppose that $F$ is a field. Then multiplicative inverses are unique. This means: Let $x\in F$. If $y,y'\in F$ satisfy $x\cdot y=1$ and $x\cdot y'=1$, then $y=y'$.
    \begin{proof}
        Let $x,y,y'\in F$ be such that $x\cdot y=1$ and $x\cdot y'=1$. From Definition \ref{dfn:7.8}, we have
        \begin{align*}
            (y\cdot x)\cdot y' &= y\cdot (x\cdot y')\tag*{FA6}\\
            1 \cdot y' &= y\cdot 1\tag*{FA8}\\
            y' &= y\tag*{FA7}
        \end{align*}
    \end{proof}
\end{theorem}

We usually write $x^{-1}$ or $\frac{1}{x}$ for the multiplicative inverse of $x$.

\begin{corollary}\label{cly:7.15}
    If $x\in F$ and $x\neq 0$, then $(x^{-1})^{-1}=x$.
    \begin{proof}
        Let $x\in F\setminus\{0\}$. Then by FA8 from Definition \ref{dfn:7.8}, there exists $x^{-1}\in F$ such that $x\cdot x^{-1}=x^{-1}\cdot x=1$. It follows from Theorem \ref{trm:7.13} that $x^{-1}\neq 0$ (if $x^{-1}=0$, then Theorem \ref{trm:7.13} would imply that $x\cdot x^{-1}=0$, a contradiction). Thus, by FA8 from Definition \ref{dfn:7.8} again, there exists $(x^{-1})^{-1}\in F$ such that $x^{-1}\cdot(x^{-1})^{-1}=(x^{-1})^{-1}\cdot x^{-1}=1$. Having established that $(x^{-1})^{-1}$ exists, $x^{-1}\cdot(x^{-1})^{-1}=1$, and $x^{-1}\cdot x=1$, we have by Theorem \ref{trm:7.14} that $(x^{-1})^{-1}=x$.
    \end{proof}
\end{corollary}

\begin{theorem}\label{trm:7.16}
    Let $F$ be a field, and let $a,b,c\in F$. If $a\cdot b=a\cdot c$ and $a\neq 0$, then $b=c$.
    \begin{proof}
        Let $a,b,c\in F$ be such that $a\cdot b=a\cdot c$ and $a\neq 0$. By FA8 from Definition \ref{dfn:7.8}, there exists $a^{-1}\in F$ such that $a\cdot a^{-1}=a^{-1}\cdot a=1$. Having established that $a^{-1}$ exists, we can prove from Definition \ref{dfn:7.8} that
        \begin{align*}
            a^{-1}\cdot(a\cdot b) &= a^{-1}\cdot(a\cdot c)\\
            (a^{-1}\cdot a)\cdot b &= (a^{-1}\cdot a)\cdot c\tag*{FA6}\\
            1\cdot b &= 1\cdot c\tag*{FA8}\\
            b &= c\tag*{FA7}
        \end{align*}
    \end{proof}
\end{theorem}

\begin{theorem}\label{trm:7.17}
    Let $F$ be a field, and let $a,b\in F$. If $a\cdot b=0$, then $a=0$ or $b=0$.
    \begin{proof}
        Let $a,b\in F$ be such that $a\cdot b=0$, and suppose for the sake of contradiction that $a\neq 0$ and $b\neq 0$. It follows from the supposition by consecutive applications of FA8 from Definition \ref{dfn:7.8} that $a^{-1}$ and $b^{-1}$ exist. Thus, from Definition \ref{dfn:7.8}, we have
        \begin{align*}
            1 &= 1\cdot 1\tag*{FA7}\\
            &= (a\cdot a^{-1})\cdot(b\cdot b^{-1})\tag*{FA8}\\
            &= (a\cdot b)\cdot(a^{-1}\cdot b^{-1})\tag*{FA6 and FA7}\\
            &= 0\cdot(a^{-1}\cdot b^{-1})\tag*{Substitution}\\
            &= 0\tag*{Theorem \ref{trm:7.13}}
        \end{align*}
        But this contradicts FA10 from Definition \ref{dfn:7.8}.
    \end{proof}
\end{theorem}

\begin{lemma}\label{lem:7.18}
    Let $F$ be a field. If $a\in F$, then $-a=(-1)a$.
    \begin{proof}
        Let $a\in F$. From Definition \ref{dfn:7.8}, we have
        \begin{align*}
            0 &= 0\cdot a\tag*{Theorem \ref{trm:7.13}}\\
            a+(-a) &= (1+(-1))\cdot a\tag*{FA4}\\
            a+(-a) &= 1\cdot a+(-1)\cdot a\tag*{FA9}\\
            a+(-a) &= a+(-1)a\tag*{FA7}\\
            -a &= (-1)a\tag*{Theorem \ref{trm:7.12}}
        \end{align*}
    \end{proof}
\end{lemma}

\begin{lemma}\label{lem:7.19}
    Let $F$ be a field. If $a,b\in F$, then $a\cdot(-b)=-(a\cdot b)=(-a)\cdot b$.
    \begin{proof}
        Let $a,b\in F$. From Definition \ref{dfn:7.8}, we have
        \begin{align*}
            a\cdot(-b) &= a\cdot((-1)\cdot b)\tag*{Lemma \ref{lem:7.18}}\\
            &= a\cdot(b\cdot(-1))\tag*{FA5}\\
            &= (a\cdot b)\cdot(-1)\tag*{FA6}\\
            &= (-1)\cdot(a\cdot b)\tag*{FA5}\\
            \Aboxed{&= -(a\cdot b)}\tag*{Lemma \ref{lem:7.18}}\\
            &= (-1)\cdot(a\cdot b)\tag*{Lemma \ref{lem:7.18}}\\
            &= ((-1)\cdot a)\cdot b\tag*{FA6}\\
            \Aboxed{&= (-a)\cdot b}\tag*{Lemma \ref{lem:7.18}}
        \end{align*}
    \end{proof}
\end{lemma}

\begin{lemma}\label{lem:7.20}
    Let $F$ be a field. If $a,b\in F$, then $a\cdot b=(-a)\cdot(-b)$.
    \begin{proof}
        Let $a,b\in F$. Thus, we have
        \begin{align*}
            (-a)\cdot(-b) &= -(-a)\cdot b\tag*{Lemma \ref{lem:7.19}}\\
            &= a\cdot b\tag*{Corollary \ref{cly:7.11}}
        \end{align*}
    \end{proof}
\end{lemma}

\begin{definition}\label{dfn:7.21}
    An \textbf{ordered field} is a field $F$ equipped with an ordering $<$ (satisfying Definition \ref{dfn:3.1}) such that also:
    \begin{enumerate}[label={(\alph*)},ref={\thedefinition\alph*}]
        \item \label{dfn:7.21a}Addition respects the ordering: if $x<y$, then $x+z<y+z$ for all $z\in F$.
        \item \label{dfn:7.21b}Multiplication respects the ordering: if $0<x$ and $0<y$, then $0<x\cdot y$.
    \end{enumerate}
\end{definition}

\begin{definition}\label{dfn:7.22}
    Suppose $F$ is an ordered field and $x\in F$. If $0<x$, we say that $x$ is \textbf{positive}. If $x<0$, we say that $x$ is \textbf{negative}.
\end{definition}

\begin{lemma}\label{lem:7.23}
    Let $F$ be an ordered field, and let $x\in F$. If $0<x$, then $-x<0$. Similarly, if $x<0$, then $0<-x$.
    \begin{proof}
        Let $x\in F$ be such that $0<x$. Then by Definition \ref{dfn:7.21a}, $0+(-x)<x+(-x)$. Consequently, from Definition \ref{dfn:7.8}, we have
        \begin{align*}
            -x &< x+(-x)\tag*{FA3}\\
            -x &< 0\tag*{FA4}
        \end{align*}
        The proof is symmetric if $x<0$.
    \end{proof}
\end{lemma}

\begin{lemma}\label{lem:7.24}
    Let $F$ be an ordered field, and let $x,y,z\in F$.
    \begin{enumerate}[label={\textup{(}\alph*\textup{)}},ref={\thelemma\alph*}]
        \item \label{lem:7.24a}If $x>0$ and $y<z$, then $x\cdot y<x\cdot z$.
        \item \label{lem:7.24b}If $x<0$ and $y<z$, then $x\cdot z<x\cdot y$.
    \end{enumerate}
    \begin{proof}[Proof of a]
        Let $x,y,z\in F$ be such that $x>0$ and $y<z$. It follows from the latter condition by Definition \ref{dfn:7.21a} that $y+(-y)<z+(-y)$. Thus, by FA4 from Definition \ref{dfn:7.8}, we have $0<z+(-y)$. This combined with the fact that $0<x$ implies by Definition \ref{dfn:7.21b} that $0<x\cdot(z+(-y))$. Consequently, from Definition \ref{dfn:7.8}, we have
        \begin{align*}
            0 &< x\cdot z+x\cdot(-y)\tag*{FA9}\\
            0 &< x\cdot z+(-(x\cdot y))\tag*{Lemma \ref{lem:7.19}}\\
            0+x\cdot y &< (x\cdot z+(-(x\cdot y)))+x\cdot y\tag*{Definition \ref{dfn:7.21a}}\\
            0+x\cdot y &< x\cdot z+(-(x\cdot y)+x\cdot y)\tag*{FA2}\\
            0+x\cdot y &< x\cdot z+0\tag*{FA4}\\
            x\cdot y &< x\cdot z\tag*{FA3}
        \end{align*}
    \end{proof}
    \begin{proof}[Proof of b]
        Let $x,y,z\in F$ be such that $x<0$ and $y<z$. It follows from the former condition by Lemma \ref{lem:7.23} that $0<-x$. Thus, by Lemma \ref{lem:7.24a}, $(-x)\cdot y<(-x)\cdot z$. Consequently, from Definition \ref{dfn:7.8}, we have
        \begin{align*}
            -(x\cdot y) &< -(x\cdot z)\tag*{Lemma \ref{lem:7.19}}\\
            -(x\cdot y)+(x\cdot y+x\cdot z) &< -(x\cdot z)+(x\cdot y+x\cdot z)\tag*{Definition \ref{dfn:7.21a}}\\
            -(x\cdot y)+(x\cdot y+x\cdot z) &< -(x\cdot z)+(x\cdot z+x\cdot y)\tag*{FA1}\\
            (-(x\cdot y)+x\cdot y)+x\cdot z &< (-(x\cdot z)+x\cdot z)+x\cdot y\tag*{FA2}\\
            0+x\cdot z &< 0+x\cdot y\tag*{FA4}\\
            x\cdot z &< x\cdot y\tag*{FA3}
        \end{align*}
    \end{proof}
\end{lemma}

\begin{remark}\label{rmk:7.25}
    An immediate consequence of this lemma is the fact that if $x$ and $y$ are both positive or both negative, their product is positive.
\end{remark}

\begin{lemma}\label{lem:7.26}
    Let $F$ be an ordered field, and let $x\in F$. Then $0\leq x^2$. Moreover, if $x\neq 0$, then $0<x^2$.
    \begin{proof}
        We divide into two cases ($x=0$ and $x\neq 0$). Suppose first that $x=0$. Then by Theorem \ref{trm:7.13}, $0\leq 0=0\cdot 0=0^2=x^2$, as desired. Now suppose that $x\neq 0$. We divide into two cases again ($x>0$ and $x<0$). If $x>0$, then by Lemma \ref{lem:7.24a}, $x>0$ and $0<x$ imply that $x\cdot 0<x\cdot x$, from which it follows by Theorem \ref{trm:7.13} that $0<x^2$, as desired. On the other hand, if $x<0$, then by Lemma \ref{lem:7.24b}, $x<0$ and $x<0$ imply that $x\cdot 0<x\cdot x$, from which it follows for the same reason as before that $0<x^2$, as desired. Both cases together prove the first statement, while the second case alone proves the second statement.
    \end{proof}
\end{lemma}

\begin{corollary}\label{cly:7.27}
    Let $F$ be an ordered field. Then $0<1$.
    \begin{proof}
        By FA10 from Definition \ref{dfn:7.8}, $1\neq 0$. Thus, by Lemma \ref{lem:7.26}, $0<1^2=1$, as desired.
    \end{proof}
\end{corollary}

\begin{theorem}\label{trm:7.28}
    If $F$ is an ordered field, then $F$ has no first or last point.
    \begin{proof}
        Suppose for the sake of contradiction that $F$ has a first point $a$. By Corollary \ref{cly:7.27}, we have that $0<1$, which implies by Lemma \ref{lem:7.23} that $-1<0$. It follows by Definition \ref{dfn:7.21a} that $-1+a<0+a$. Thus, by FA3 from Definition \ref{dfn:7.8}, $-1+a<a$. Since there exists an object in $F$ (namely $-1+a$) that is less than $a$, Definition \ref{dfn:3.3} tells us that $a$ is not the first point of $F$, a contradiction.\par
        The proof is symmetric in the other case.
    \end{proof}
\end{theorem}
\pagebreak

\begin{theorem}\label{trm:7.29}
    The rational numbers $\Q$ form an ordered field.
    \begin{proof}
        To prove that $\Q$ forms an ordered field, Definition \ref{dfn:7.21} tells us that it will suffice to show that $\Q$ forms a field; has an ordering $<$; satisfies $x+z<y+z$ if $x<y$ for all $z\in\Q$; and satisfies $0<x\cdot y$ if $0<x$ and $0<y$. We will take this one constraint at a time.\par\smallskip
        To show that $\Q$ forms a field, Definition \ref{dfn:7.8} tells us that it will suffice to verify that $\Q$ has two binary operations ($+$ and $\cdot$), and satisfies field axioms 1-10. Define $+$ and $\cdot$ as in Definition \ref{dfn:2.7}. Under these definitions, parts a-i of Theorem \ref{trm:2.10} guarantee that $\Q$ satisfies FA1-FA9, respectively. As to FA10, to verify that $\eqclass{1}{1}\neq\eqclass{0}{1}$, Exercise \ref{exr:2.6} tells us that it will suffice to confirm that $(1,1)\nsim(1,0)$. But since $1\cdot 0=0\neq 1=1\cdot 1$, Exercise \ref{exr:2.2e} confirms that $(1,1)\nsim(1,0)$, as desired.\par
        $\Q$ has an ordering by Exercise \ref{exr:3.9d}, as desired.\par
        To show that $x+z<y+z$ if $x<y$ for all $z\in\Q$, let $\eqclass{a}{b},\eqclass{c}{d},\eqclass{x}{z}$ be arbitrary elements of $\Q$ with positive denominators (we can choose these WLOG by Exercise \ref{exr:3.9b}) and the first two satisfying $\eqclass{a}{b}<\eqclass{c}{d}$; we seek to verify that $\eqclass{a}{b}+\eqclass{x}{z}<\eqclass{c}{d}+\eqclass{x}{z}$. Since $\eqclass{a}{b}<\eqclass{c}{d}$, we have by Exercise \ref{exr:3.9c} that $ad<bc$. It follows by Script \ref{sct:0} that
        \begin{align*}
            ad &< bc\\
            adzz &< bczz\\
            adzz+bdxz &< bczz+bdxz\\
            azdz+bxdz &< bzcz+bzdx\\
            (az+bx)(dz) &< (bz)(cz+dx)
        \end{align*}
        Thus, by Exercise \ref{exr:3.9c}, $\eqclass{az+bx}{bz}<\eqclass{cz+dx}{dz}$. Therefore, by Definition \ref{dfn:2.7}, $\eqclass{a}{b}+\eqclass{x}{z}<\eqclass{c}{d}+\eqclass{x}{z}$, as desired.\par
        To show that $0<x\cdot y$ if $0<x$ and $0<y$, let $\eqclass{a}{b},\eqclass{c}{d}$ be arbitrary elements of $\Q$ with positive denominators (which we can choose for the same reason as before) such that $\eqclass{0}{1}<\eqclass{a}{b}$ and $\eqclass{0}{1}<\eqclass{c}{d}$; we seek to verify that $\eqclass{0}{1}<\eqclass{a}{b}\cdot\eqclass{c}{d}$. Since $\eqclass{0}{1}<\eqclass{a}{b}$ and $\eqclass{0}{1}<\eqclass{c}{d}$, we have by Exercise \ref{exr:3.9c} that $0\cdot b<1\cdot a$ and $0\cdot d<1\cdot c$. It follows by Script \ref{sct:0} that $0\cdot bd<1\cdot ac$. Thus, by Exercise \ref{exr:3.9c}, $\eqclass{0}{1}<\eqclass{ac}{bd}$. Therefore, by Definition \ref{dfn:2.7}, $\eqclass{0}{1}<\eqclass{a}{b}\cdot\eqclass{c}{d}$, as desired.
    \end{proof}
\end{theorem}

\stepcounter{theorem}

\begin{definition}\label{dfn:7.31}\marginnote{2/4:}
    We define $\oplus$ on $\R$ as follows. Let $A,B\in\R$ be Dedekind cuts. Define
    \begin{equation*}
        A\oplus B = \{a+b\mid a\in A\text{ and }b\in B\}
    \end{equation*}
\end{definition}

\begin{exercise}\label{exr:7.32}\leavevmode
    \begin{enumerate}[label={(\alph*)},ref={\theexercise\alph*}]
        \item \label{exr:7.32a}Prove that $A\oplus B$ is a Dedekind cut.
        \item \label{exr:7.32b}Prove that $\oplus$ is commutative and associative.
        \item \label{exr:7.32c}Prove that if $A\in\R$, then $A=\bm{0}\oplus A$.
    \end{enumerate}
    \begin{proof}[Proof of a]
        To prove that $A\oplus B$ is a Dedekind cut, Definition \ref{dfn:6.1} tells us that it will suffice to show that $A\oplus B\neq\emptyset$; $A\oplus B\neq\Q$; if $r\in A\oplus B$ and $s\in\Q$ satisfy $s<r$, then $s\in A\oplus B$; and if $r\in A\oplus B$, then there is some $s\in A\oplus B$ with $s>r$. We will take this one claim at a time.\par
        To show that $A\oplus B\neq\emptyset$, Definition \ref{dfn:1.8} tells us that it will suffice to find an element of $A\oplus B$. Since $A,B$ are Dedekind cuts, Definition \ref{dfn:6.1} asserts that they are nonempty. Thus, there exist rational numbers $x\in A$ and $y\in B$. Therefore, by the definition of $A\oplus B$, the sum $x+y\in A\oplus B$, as desired.\par
        To show that $A\oplus B\neq\Q$, Definition \ref{dfn:1.2} tells us that it will suffice to find an element of $\Q$ that is not an element of $A\oplus B$. For an analogous reason to before, we can choose $x,y\in\Q$ such that $x\notin A$ and $y\notin B$. It follows by Lemma \ref{lem:6.2} and Definition \ref{dfn:5.6} that $x\geq a$ for all $a\in A$ and $y\geq b$ for all $b\in B$. Additionally, since $x\notin A$, we have that $x\neq a$ for any $a\in A$; thus, $x>a$ for all $a\in A$. Similarly, $y>b$ for all $b\in B$. Consequently, by Script \ref{sct:2}, $x+y>a+b$ for all $a+b\in A\oplus B$. Therefore, $x+y\notin A\oplus B$, as desired.\par
        To show that if $r\in A\oplus B$ and $s\in\Q$ satisfy $s<r$, then $s\in A\oplus B$, we let $r\in A\oplus B$ and $s\in\Q$ be arbitrary elements of their respective sets that satisfy $s<r$ and seek to verify that $s\in A\oplus B$. Since $r\in A\oplus B$, $r=x+y$ for some $x\in A$ and $y\in B$. Additionally, it follows from the fact that $s<r$ that $s=r-q=x+y-q$ for some $q\in\Q^+$. Since $y\in B$ and $y-q\in\Q$ satisfy $y-q<y$, we have by Definition \ref{dfn:6.1b} that $y-q\in B$. Therefore, $s=(x)+(y-q)$ is an element of $A\oplus B$, as desired.\par
        To show that if $r\in A\oplus B$, then there is some $s\in A\oplus B$ with $s>r$, we let $r\in A\oplus B$ and seek to find such an $s$. Since $r\in A\oplus B$, $r=x+y$ for some $x\in A$ and $y\in B$. It follows from the fact that $x\in A$ by Definition \ref{dfn:6.1c} that there exists a $z\in A$ with $z>x$. Consequently, by Script \ref{sct:0}, $z+y>x+y$ is the desired element of $A\oplus B$.
    \end{proof}
    \begin{proof}[Proof of b]
        To prove that $\oplus$ is commutative, Definition \ref{dfn:7.1} tells us that it will suffice to show that for all $A,B\in\R$, we have $A\oplus B=B\oplus A$. Let $A,B$ be arbitrary elements of $\R$. Then by Definition \ref{dfn:7.31}, we clearly have
        \begin{align*}
            A\oplus B &= \{a+b\mid a\in A\text{ and }b\in B\}\\
            &= \{b+a\mid b\in B\text{ and }a\in A\}\\
            &= B\oplus A
        \end{align*}
        To prove that $\oplus$ is associative, Definition \ref{dfn:7.1} tells us that it will suffice to show that for all $A,B,C\in\R$, we have $(A\oplus B)\oplus C=A\oplus(B\oplus C)$. Let $A,B,C$ be arbitrary elements of $\R$. Then by Definition \ref{dfn:7.31}, we clearly have
        \begin{align*}
            (A\oplus B)\oplus C &= \{a+b\mid a\in A\text{ and }b\in B\}\oplus C\\
            &= \{d+c\mid d\in\{a+b\mid a\in A\text{ and }b\in B\}\text{ and }c\in C\}\\
            &= \{d+c\mid d=a+b\text{ for some }a\in A\text{ and }b\in B\text{, and }c\in C\}\\
            &= \{a+b+c\mid a\in A\text{ and }b\in B\text{ and }c\in C\}\\
            &= \{a+e\mid a\in A\text{, and }e=b+c\text{ for some }b\in B\text{ and }c\in C\}\\
            &= \{a+e\mid c\in C\text{ and }e\in\{b+c\mid b\in B\text{ and }c\in C\}\}\\
            &= A\oplus\{b+c\mid b\in B\text{ and }c\in C\}\\
            &= A\oplus(B\oplus C)
        \end{align*}
        Note that we also make use of Exercise \ref{exr:7.32a} to guarantee $A\oplus B\in\R$, so that we can apply $\oplus$ to $A\oplus B$ and $C$. We similarly invoke Exercise \ref{exr:7.32a} to take the sum of $A$ and $B\oplus C$.
    \end{proof}
    \begin{proof}[Proof of c]
        To prove that for all $A\in\R$, $A=\bm{0}\oplus A$, we will show for an arbitrary $A\in\R$ that every element of $A$ is an element of $\bm{0}\oplus A$ and vice versa. Let $A$ be an arbitrary element of $\R$. Suppose first that $x\in A$. Then by Definition \ref{dfn:6.1c}, there exists $y\in A$ such that $y>x$. Let $z=x-y$. Clearly, $z\in\Q$ and $z<0$, so we know that $z\in\bm{0}$. Additionally, since $x-z=y$, we know that $x-z\in A$. Therefore, since $x=(z)+(x-z)$, we have by Definition \ref{dfn:7.31} that $x\in\bm{0}\oplus A$. Now suppose that $z\in\bm{0}\oplus A$. Then by Definition \ref{dfn:7.31}, $z=x+y$ for some $x\in\bm{0}$ and $y\in A$. Since $x\in\bm{0}$, we know that $x<0$, which means that $y>z$. This combined with the fact that $y\in A$ and $z\in\Q$ implies by Definition \ref{dfn:6.1b} that $z\in A$.
    \end{proof}
\end{exercise}




\end{document}