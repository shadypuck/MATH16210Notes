\documentclass[../main.tex]{subfiles}

\pagestyle{main}
\renewcommand{\chaptermark}[1]{\markboth{\chaptername\ \thechapter}{}}
\setcounter{chapter}{8}

\begin{document}




\chapter{Continuous Functions}\label{sct:9}
\section{Journal}
\begin{lemma}\label{lem:9.1}\marginnote{\emph{2/16:}}
    Let $X\subset\R$ and $f:X\to\R$. If $A,B\subset\R$, then
    \begin{gather*}
        f^{-1}(A\cup B) = f^{-1}(A)\cup f^{-1}(B)\\
        f^{-1}(A\cap B) = f^{-1}(A)\cap f^{-1}(B)\\
        f^{-1}(A\setminus B) = f^{-1}(A)\setminus f^{-1}(B)\\
        f^{-1}(\R) = X
    \end{gather*}
    \begin{proof}
        To prove that $f^{-1}(A\cup B)=f^{-1}(A)\cup f^{-1}(B)$, Definition \ref{dfn:1.2} tells us that it will suffice to show that every $x\in f^{-1}(A\cup B)$ is an element of $f^{-1}(A)\cup f^{-1}(B)$ and vice versa. Suppose first that $x$ is an arbitrary element of $f^{-1}(A\cup B)$. Then by Definition \ref{dfn:1.18}, $f(x)\in A\cup B$. Thus, by Definition \ref{dfn:1.5}, $f(x)\in A$ or $f(x)\in B$. We now divide into two cases. If $f(x)\in A$, then by Definition \ref{dfn:1.18}, $x\in f^{-1}(A)$. It follows by Definition \ref{dfn:1.5} that $x\in f^{-1}(A)\cup f^{-1}(B)$, as desired. The argument is symmetric in the other case. Now suppose that $x\in f^{-1}(A)\cup f^{-1}(B)$. Then by Definition \ref{dfn:1.5}, $x\in f^{-1}(A)$ or $x\in f^{-1}(B)$. We now divide into two cases. If $x\in f^{-1}(A)$, then by Definition \ref{dfn:1.18}, $f(x)\in A$. It follows by Definition \ref{dfn:1.5} that $f(x)\in A\cup B$. Therefore, by Definition \ref{dfn:1.18}, $x\in f^{-1}(A\cup B)$. The argument is symmetric in the other case, as desired.\par
        To prove that $f^{-1}(A\cap B)=f^{-1}(A)\cap f^{-1}(B)$, Definition \ref{dfn:1.2} tells us that it will suffice to show that every $x\in f^{-1}(A\cap B)$ is an element of $f^{-1}(A)\cap f^{-1}(B)$ and vice versa. Suppose first that $x$ is an arbitrary element of $f^{-1}(A\cap B)$. Then by Definition \ref{dfn:1.18}, $f(x)\in A\cap B$. Thus, by Definition \ref{dfn:1.6}, $f(x)\in A$ and $f(x)\in B$. It follows by consecutive applications of Definition \ref{dfn:1.18} that $x\in f^{-1}(A)$ and $x\in f^{-1}(B)$. Therefore, by Definition \ref{dfn:1.6}, $x\in f^{-1}(A)\cap f^{-1}(B)$, as desired. Now suppose that $x\in f^{-1}(A)\cap f^{-1}(B)$. Then by Definition \ref{dfn:1.6}, $x\in f^{-1}(A)$ and $x\in f^{-1}(B)$. It follows by consecutive applications of Definition \ref{dfn:1.18} that $f(x)\in A$ and $f(x)\in B$. Thus, by Definition \ref{dfn:1.6}, $f(x)\in A\cap B$. Therefore, by Definition \ref{dfn:1.18}, $x\in f^{-1}(A\cap B)$, as desired.\par
        To prove that $f^{-1}(A\setminus B)=f^{-1}(A)\setminus f^{-1}(B)$, Definition \ref{dfn:1.2} tells us that it will suffice to show that every $x\in f^{-1}(A\setminus B)$ is an element of $f^{-1}(A)\setminus f^{-1}(B)$ and vice versa. Suppose first that $x$ is an arbitrary element of $f^{-1}(A\setminus B)$. Then by Definition \ref{dfn:1.18}, $f(x)\in A\setminus B$. Thus, by Definition \ref{dfn:1.11}, $f(x)\in A$ and $f(x)\notin B$. It follows by consecutive applications of Definition \ref{dfn:1.18} that $x\in f^{-1}(A)$ and $x\notin f^{-1}(B)$. Therefore, by Definition \ref{dfn:1.11}, $x\in f^{-1}(A)\setminus f^{-1}(B)$, as desired. Now suppose that $x\in f^{-1}(A)\setminus f^{-1}(B)$. Then by Definition \ref{dfn:1.11}, $x\in f^{-1}(A)$ and $x\notin f^{-1}(B)$. It follows by consecutive applications of Definition \ref{dfn:1.18} that $f(x)\in A$ and $f(x)\notin B$. Thus, by Definition \ref{dfn:1.11}, $f(x)\in A\setminus B$. Therefore, by Definition \ref{dfn:1.18}, $x\in f^{-1}(A\setminus B)$, as desired.\par
        To prove that $f^{-1}(\R)=X$, Definition \ref{dfn:1.2} tells us that it will suffice to show that every $x\in f^{-1}(\R)$ is an element of $X$ and vice versa. Suppose first that $x$ is an arbitrary element of $f^{-1}(\R)$. Then by Definition \ref{dfn:1.18}, $x\in X$, as desired. Now suppose that $x\in X$. Then by Definition \ref{dfn:1.16}, $f(x)\in\R$. It follows by Definition \ref{dfn:1.18} that $x\in f^{-1}(\R)$, as desired.
    \end{proof}
\end{lemma}

\begin{exercise}\label{exr:9.2}
    Let $f:X\to\R$. Let $A\subset X$ and $B\subset\R$. Show that
    \begin{gather*}
        f(f^{-1}(B)) \subset B\\
        A\subset f^{-1}(f(A))
    \end{gather*}
    Give examples to show that the inclusions can be proper.
    \begin{proof}
        To prove that $f(f^{-1}(B))\subset B$, Definition \ref{dfn:1.3} tells us that it will suffice to show that every $y\in f(f^{-1}(B))$ is an element of $B$. Let $y$ be an arbitrary element of $f(f^{-1}(B))$. Then by Definition \ref{dfn:1.18}, $y=f(x)$ for some $x\in f^{-1}(B)$. By Definition \ref{dfn:1.18} again, $f(x)\in B$. Therefore, since $y=f(x)$, it follows that $y\in B$, as desired.\par
        To prove that $A\subset f^{-1}(f(A))$, Definition \ref{dfn:1.3} tells us that it will suffice to show that every $x\in A$ is an element of $f^{-1}(f(A))$. Let $x$ be an arbitrary element of $A$. Then by Definition \ref{dfn:1.18}, $f(x)\in f(A)$. Therefore, by Definition \ref{dfn:1.18}, we have $x\in f^{-1}(f(A))$, as desired.\par
        Let $X=\{1,2\}$ and let $f:X\to\R$ be defined by $f(1)=3$ and $f(2)=3$. If we let $B=\{3,4\}$, then $f(f^{-1}(B))=\{3\}\subsetneq\{3,4\}$. Additionally, if we let $A=\{1\}$, then $A\subsetneq f^{-1}(f(A))=\{1,2\}$.
    \end{proof}
\end{exercise}

\begin{exercise}\label{exr:9.3}
    Let $f:X\to\R$. Let $A\subset X$ and $B\subset\R$. Then $f(A)\subset B\Longleftrightarrow A\subset f^{-1}(B)$.
    \begin{proof}
        Suppose first that $f(A)\subset B$. To prove that $A\subset f^{-1}(B)$, Definition \ref{dfn:1.3} tells us that it will suffice to show that every $x\in A$ is an element of $f^{-1}(B)$. Let $x$ be an arbitrary element of $A$. Then by Definition \ref{dfn:1.18}, $f(x)\in f(A)$. It follows by the hypothesis and Definition \ref{dfn:1.3} that $f(x)\in B$. Therefore, by Definition \ref{dfn:1.18} again, $x\in f^{-1}(B)$.\par
        Now suppose that $A\subset f^{-1}(B)$. To prove that $f(A)\subset B$, Definition \ref{dfn:1.3} tells us that it will suffice to show that every $y\in f(A)$ is an element of $B$. Let $y$ be an arbitrary element of $f(A)$. Then by Definition \ref{dfn:1.18}, $y=f(x)$ for some $x\in A$. It follows by the hypothesis and Definition \ref{dfn:1.3} that $x\in f^{-1}(B)$. Consequently, by Definition \ref{dfn:1.18} again, $f(x)\in B$. Therefore, since $y=f(x)$, $y\in B$.
    \end{proof}
\end{exercise}

\begin{definition}\label{dfn:9.4}
    Let $X\subset\R$. A function $f:X\to\R$ is \textbf{continuous} if for every open set $U\subset\R$, the preimage $f^{-1}(U)$ is open in $X$.
\end{definition}

\begin{proposition}\label{prp:9.5}
    Let $X\subset\R$. A function $f:X\to\R$ is continuous if and only if for every closed set $F\subset\R$, the preimage $f^{-1}(F)$ is closed in $X$.
    \begin{proof}
        Suppose first that $f$ is continuous. We seek to prove that for every closed set $F\subset\R$, the preimage $f^{-1}(F)$ is closed in $X$. Let $F$ be an arbitrary closed subset of $\R$. Then by Definition \ref{dfn:4.8}, $F=\R\setminus U$ for some open set $U\subset\R$. It follows by Definition \ref{dfn:9.4} since $f$ is continuous that $f^{-1}(U)$ is open in $X$. Additionally, by consecutive applications of Lemma \ref{lem:9.1}, $f^{-1}(F)=f^{-1}(\R\setminus U)=f^{-1}(\R)\setminus f^{-1}(U)=X\setminus f^{-1}(U)$. Therefore, since $f^{-1}(U)$ is open in $X$, Exercise \ref{exr:8.13} implies that $X\setminus f^{-1}(U)=f^{-1}(F)$ is closed in $X$.\par
        The proof is symmetric in the other direction.
    \end{proof}
\end{proposition}

\begin{definition}\label{dfn:9.6}
    Let $X\subset Y\subset\R$ and let $f:Y\to\R$. Then the \textbf{restriction} (of $f$ to $X$), written $f|_X$ is the function $f|_X:X\to\R$ defined by
    \begin{equation*}
        f|_X(x) = f(x)
    \end{equation*}
    for all $x\in X$.
\end{definition}

\begin{proposition}\label{prp:9.7}
    Let $X\subset Y\subset\R$. If $f:Y\to\R$ is continuous, then the restriction of $f$ to $X$ is continuous.
    \begin{proof}
        To prove that $f|_X$ is continuous, Definition \ref{dfn:9.4} tells us that it will suffice to show that for every open set $U\subset\R$, the preimage $f|_X^{-1}(U)$ is open in $X$. Let $U$ be an open subset of $\R$. Then
        \begingroup
        \allowdisplaybreaks
        \begin{align*}
            f|_X^{-1}(U) &= \{x\in X\mid f|_X(x)\in U\}\tag*{Definition \ref{dfn:1.18}}\\
            &= \{x\in X\mid f(x)\in U\}\tag*{Definition \ref{dfn:9.6}}\\
            &= \{x\in Y\mid f(x)\in U\}\cap X\tag*{Script \ref{sct:1}}\\
            &= f^{-1}(U)\cap X\tag*{Definition \ref{dfn:1.18}}\\
            &= (Y\cap G)\cap X\tag*{Definitions \ref{dfn:9.4} and \ref{dfn:8.11}}\\
            &= X\cap G\tag*{Script \ref{sct:1}}
        \end{align*}
        \endgroup
        Since $f|_X^{-1}(U)=X\cap G$ where $G$ is an open set, Definition \ref{dfn:8.11} asserts that $f|_X^{-1}(U)$ is open in $X$.
    \end{proof}
\end{proposition}

\begin{exercise}\label{exr:9.8}
    Show that for any $X\subsetneq\R$ that is not open and any continuous function $f:X\to\R$, there is an open set $U$ for which $f^{-1}(U)$ is open in $X$ but is not open in $\R$.
    \begin{proof}
        We will prove that $\R$ is an open set for which $f^{-1}(\R)$ is open in $X$ but not in $\R$. First, by Theorem \ref{trm:5.1}, $\R$ is open. Next, by Lemma \ref{lem:9.1}, $f^{-1}(\R)=X$. It follows since $f^{-1}(\R)=X=X\cap\R$ (where $\R$ is an open set) by Definition \ref{dfn:8.11} that $f^{-1}(\R)$ is open in $X$. Last, since $X$ is not open (in $\R$) by definition, $f^{-1}(\R)=X$ is not open in $\R$.
    \end{proof}
\end{exercise}

\begin{definition}\label{dfn:9.9}
    The function $f:X\to\R$ is \textbf{continuous} (at $x\in X$) if for every region $R$ containing $f(x)$, there exists an open set $S$ containing $x$ such that $S\cap X\subset f^{-1}(R)$.
\end{definition}

\begin{theorem}\label{trm:9.10}
    The function $f:X\to\R$ is continuous if and only if it is continuous at every $x\in X$.
    \begin{proof}
        Suppose first that $f$ is continuous, and suppose for the sake of contradiction that $f$ is not continuous at every $x\in X$. Then by Definition \ref{dfn:9.9}, there exists some $x\in X$ such that $f$ is not continuous at $x$. Thus, there exists a region $R$ with $f(x)\in R$ such that for all open sets $S$ containing $x$, $S\cap X\not\subset f^{-1}(R)$. Since $f$ is continuous by hypothesis and $R$ is open by Corollary \ref{cly:4.11}, $f^{-1}(R)$ is open in $X$. It follows by Definition \ref{dfn:8.11} that $f^{-1}(R)=X\cap S$ for some open set $S$. But this implies that $f^{-1}(R)\not\subset f^{-1}(R)$, a contradiction.\par
        Now suppose that $f$ is continuous at every $x\in X$. To prove that $f$ is continuous, Definition \ref{dfn:9.4} tells us that it will suffice to show that for every open set $U\subset\R$, the preimage $f^{-1}(U)$ is open in $X$. We divide into two cases ($f^{-1}(U)=\emptyset$ and $f^{-1}(U)\neq\emptyset$). If $f^{-1}(U)=\emptyset$, then since $\emptyset\cap X=\emptyset$ by Script \ref{sct:1} where $\emptyset$ is open by Theorem \ref{trm:5.1}, Definition \ref{dfn:8.11} tells us that $\emptyset=f^{-1}(U)$ is open in $X$, as desired. On the other hand, if $f^{-1}(U)\neq\emptyset$, Definition \ref{dfn:8.11} tells us that it will suffice to show that $f^{-1}(U)=S\cap X$ where $S$ is an open set. We first seek to show that for every $x\in f^{-1}(U)$, there exists an open set $S_x$ containing $x$ such that $S_x\cap X\subset f^{-1}(U)$. Let $x$ be an arbitrary element of $f^{-1}(U)$. It follows by Definition \ref{dfn:1.18} that $f(x)\in U$. Thus, since $U$ is open, we have by Theorem \ref{trm:4.10} that there exists a region $R$ such that $f(x)\in R$ and $R\subset U$. Consequently, since $R$ is open by Corollary \ref{cly:4.11}, we have by Definition \ref{dfn:9.9} that there exists an open set $S_x$ containing $x$ such that $S_x\cap X\subset f^{-1}(R)$. Additionally, Script \ref{sct:1} tells us based off of the fact that $R\subset U$ that $f^{-1}(R)\subset f^{-1}(U)$. Thus, by subset transitivity, $S_x\cap X\subset f^{-1}(U)$. At this point, let $S=\bigcup_{x\in f^{-1}(U)}S_x$. It follows immediately from Corollary \ref{cly:4.18} that $S$ is open. Additionally, since the intersection of each set in the union with $X$ is a subset of $f^{-1}(U)$, it follows by Script \ref{sct:1} that $S\cap X\subset f^{-1}(U)$. Furthermore, for all $x\in f^{-1}(U)$, Definition \ref{dfn:1.18} asserts that $x\in X$. In addition, we have defined an $S_x$ such that $x\in S_x$. These last two results combined demonstrate by Definition \ref{dfn:1.6} that $x\in S\cap X$. Thus, by Definition \ref{dfn:1.3}, $f^{-1}(U)\subset S\cap X$. Consequently, by Theorem \ref{trm:1.7}, $f^{-1}(U)=S\cap X$. Since $f^{-1}(U)$ is the intersection of $X$ with an open set, Definition \ref{dfn:8.11} asserts that it is open in $X$, as desired.
    \end{proof}
\end{theorem}

\begin{theorem}\label{trm:9.11}\marginnote{\emph{2/18:}}
    Suppose that $f:X\to\R$ is continuous. If $X$ is connected, then $f(X)$ is connected.
    \begin{proof}
        This will be a proof by contrapositive; as such, suppose that $f(X)$ is disconnected. Then by Definition \ref{dfn:4.22}, $f(X)=A\cup B$ where $A,B$ are nonempty, disjoint sets that are open in $f(X)$. It follows from the last condition by Definition \ref{dfn:8.11} that $A=G\cap f(X)$ and $B=H\cap f(X)$, where $G,H$ are open sets. Since for all $x\in X$, $f(x)\in A$ or $f(x)\in B$, Definitions \ref{dfn:1.2} and \ref{dfn:1.6} imply that for all $x\in X$, $f(x)\in G$ or $f(x)\in H$. Thus, by Script \ref{sct:1}, $X\subset f^{-1}(G)\cup f^{-1}(H)$. Additionally, we know by Definition \ref{dfn:1.18} that for all $x\in f^{-1}(G)\cup f^{-1}(H)$, $x\in X$. Thus, by Definition \ref{dfn:1.3}, $f^{-1}(G)\cup f^{-1}(H)\subset X$. Consequently, by Theorem \ref{trm:1.7}, we have that $X=f^{-1}(G)\cup f^{-1}(H)$.\par\smallskip
        To show that $f^{-1}(G)$ and $f^{-1}(H)$ are disjoint, Definition \ref{dfn:1.9} tells us that it will suffice to verify that $f^{-1}(G)\cap f^{-1}(H)=\emptyset$. As such, suppose for the sake of contradiction that $x\in f^{-1}(G)\cap f^{-1}(H)$. Then by Definition \ref{dfn:1.6}, $x\in f^{-1}(G)$ and $x\in f^{-1}(H)$. Thus, by multiple applications of Definition \ref{dfn:1.18}, $x\in X$, $f(x)\in G$, and $f(x)\in H$. It follows from the first condition by Definition \ref{dfn:1.18} that $f(x)\in f(X)$. The facts that $f(x)\in f(X)$ and $f(x)\in G$ imply by Definitions \ref{dfn:1.6} and \ref{dfn:1.2} that $f(x)\in A$. Similarly, $f(x)\in B$. But these last two statements imply by Definition \ref{dfn:1.6} that $f(x)\in A\cap B$, a contradiction.\par
        To show that $f^{-1}(G)$ and $f^{-1}(H)$ are nonempty, Definition \ref{dfn:1.8} tells us that it will suffice to find an element of each set. As previously mentioned, $A,B$ are nonempty. Thus, by consecutive applications of Definition \ref{dfn:1.8}, there exist $f(x)\in A$ and $f(y)\in B$. Consequently, by Definitions \ref{dfn:1.2} and \ref{dfn:1.6}, $f(x)\in G$ and $f(y)\in H$. Therefore, by consecutive applications of Definition \ref{dfn:1.18}, $x\in f^{-1}(G)$ and $y\in f^{-1}(H)$, as desired.\par
        To show that $f^{-1}(G)$ and $f^{-1}(H)$ are open in $X$, Definition \ref{dfn:9.4} tells us that it will suffice to verify (since $f$ is continuous by hypothesis) that $G,H$ are open subsets of $\R$. But by definition, they are exactly that.
    \end{proof}
\end{theorem}

\begin{exercise}\label{exr:9.12}
    Use Theorem \ref{trm:9.11} to prove that if $f:[a,b]\to\R$ is continuous, then for every point $p$ between $f(a)$ and $f(b)$, there exists $c$ such that $a<c<b$ and $f(c)=p$.
    \begin{proof}
        Suppose that $a<b$. Then by Lemma \ref{lem:8.3}, $[a,b]$ is an interval. Thus, by Theorem \ref{trm:8.15}, $[a,b]$ is connected. It follows by Theorem \ref{trm:9.11} that $f([a,b])$ is connected. Consequently, by Theorem \ref{trm:8.15}, $f([a,b])$ is an interval. We divide into three cases ($f(a)<f(b)$, $f(a)>f(b)$, and $f(a)=f(b)$).\par
        First, suppose that $f(a)<f(b)$, and let $p$ be an arbitrary point between $f(a)$ and $f(b)$ (we know that at least one such point exists by Theorem \ref{trm:5.2}). Then by Definition \ref{dfn:3.6}, $f(a)<p<f(b)$. Now $a,b\in[a,b]$ by Equations \ref{eqn:8.1}, so by Definition \ref{dfn:1.18}, $f(a),f(b)\in f([a,b])$. It follows by Definition \ref{dfn:8.2} since $f([a,b])$ is an interval that $[f(a),f(b)]\subset f([a,b])$. Thus, since $f(a)<p<f(b)$ implies $p\in[f(a),f(b)]$ by Equations \ref{eqn:8.1}, Definition \ref{dfn:1.3} asserts that $p\in f([a,b])$. Consequently, by Definition \ref{dfn:1.18}, $p=f(c)$ for some $c\in[a,b]$. Additionally, since $f(a)<p<f(b)$, we know that $p\neq f(a)$ and $p\neq f(b)$. It follows that $p=f(c)$ for some $c\in(a,b)$, as desired.\par
        The proof of the second case is symmetric to that of the first.\par
        Third, suppose that $f(a)=f(b)$. This implies that there are no points $p$ between $f(a)$ and $f(b)$ by Definition \ref{dfn:3.6}, so the statement is vacuously true in this case.
    \end{proof}
\end{exercise}

\begin{lemma}\label{lem:9.13}
    If $f:(a,b)\to\R$ is continuous and injective, then $f$ is either strictly increasing or strictly decreasing on $(a,b)$.
    \begin{proof}
        Suppose for the sake of contradiction that $f$ is neither strictly increasing nor strictly decreasing. Then by Definition \ref{dfn:8.16} there exist $x,y,z\in(a,b)$ with $x<y<z$ such that $f(x)\leq f(y)$ and $f(z)\leq f(y)$. Additionally, since $f$ is injective and $x,y,z$ are distinct, we have by Definition \ref{dfn:1.20} that $f(x)<f(y)$ and $f(z)<f(y)$.\par
        We divide into two cases ($f(x)<f(z)$ and $f(x)>f(z)$). Suppose first that $f(x)<f(z)$. Since $x,y\in(a,b)$, we have that $[x,y]\subset(a,b)$. It follows by Proposition \ref{prp:9.7} that $f|_{[x,y]}:[x,y]\to\R$ is continuous. Thus, we have by the supposition and Definition \ref{dfn:9.6} that $f|_{[x,y]}(x)<f|_{[x,y]}(z)<f|_{[x,y]}(y)$. It follows by Exercise \ref{exr:9.12} that there exists $c$ with $x<c<y$ such that $f|_{[x,y]}(c)=f|_{[x,y]}(z)$. Consequently, by Definition \ref{dfn:9.6}, $f(c)=f(z)$. Thus, by Definition \ref{dfn:1.20}, $c=z$. But this implies that $x<z<y$, contradicting the fact that $x<y<z$. The proof is symmetric in the other case.
    \end{proof}
\end{lemma}

\begin{theorem}\label{trm:9.14}\marginnote{\emph{2/23:}}
    If $f:(a,b)\to\R$ is continuous and injective, then the inverse function $g:f((a,b))\to(a,b)$ is continuous.
    \begin{lemma*}
        Let $f:(a,b)\to\R$ be continuous and injective, and let $(x,y)\subset(a,b)$ be a region. Then $f((x,y))$ is also a region.
        \begin{proof}
            Since $f:(a,b)\to\R$ is continuous and injective, Lemma \ref{lem:9.13} implies that $f$ is either strictly increasing or strictly decreasing on $(a,b)$. We now divide into two cases.\par
            Suppose first that $f$ is strictly increasing. To prove that $f((x,y))$ is a region, Definition \ref{dfn:3.10} tells us that it will suffice to show that $f((x,y))=(f(x),f(y))$. To show this, Definition \ref{dfn:1.2} tells us that it will suffice to verify that every $p\in f((x,y))$ is an element of $(f(x),f(y))$ and vice versa. Let $p$ be an arbitrary element of $f((x,y))$. Then by Definition \ref{dfn:1.18}, $p=f(z)$ for some $z\in(x,y)$. Since $z\in(x,y)$, we have by Equations \ref{eqn:8.1} that $x<z<y$. Since $f$ is strictly increasing on $(a,b)$, by Definition \ref{dfn:8.16}, $x<z<y$ implies that $f(x)<f(z)<f(y)$. But this implies by Equations \ref{eqn:8.1} that $f(z)=p$ is an element of $(f(x),f(y))$, as desired. Now let $p$ be an arbitrary element of $(f(x),f(y))$. Then by Equations \ref{eqn:8.1}, $f(x)<p<f(y)$. We seek to prove that $[x,y]\subset(a,b)$. Let $q$ be an arbitrary element of $[x,y]$. Then by Equations \ref{eqn:8.1}, $x\leq q\leq y$. Additionally, since $x,y\in(a,b)$, Equations \ref{eqn:8.1} imply that $a<x<b$ and $a<y<b$. Thus, $a<x\leq q\leq y<b$, meaning by Equations \ref{eqn:8.1} that $q\in(a,b)$. Consequently, by Definition \ref{dfn:1.3}, $[x,y]\subset(a,b)$. If we now consider the restriction $f|_{[x,y]}$, we have by Proposition \ref{prp:9.7} that $f|_{[x,y]}$ is continuous. Thus, since $f|_{[x,y]}:[x,y]\to\R$ is continuous and $f|_{[x,y]}(x)=f(x)<p<f(y)=f|_{[x,y]}(y)$ (by Definition \ref{dfn:9.6}), Exercise \ref{exr:9.12} implies that there exists $c\in(x,y)$ such that $f|_{[x,y]}(c)=f(c)=p$. But by Definition \ref{dfn:1.18}, this implies that $p\in f((x,y))$.\par
            The proof is symmetric in the other case.
        \end{proof}
    \end{lemma*}
    \begin{proof}[Proof of Theorem \ref{trm:9.14}]
        We first show that $g$ exists, and then show that it is continuous.\par
        To prove that $g$ is a function, Definition \ref{dfn:1.16} tells us that it will suffice to show that for all $y\in f((a,b))$, there exists a unique $x\in (a,b)$ such that $g(y)=x$. We will first show that for each $y$, such an element exists, and then show that it is unique. Let $y$ be an arbitrary element of $f((a,b))$. Then by Definition \ref{dfn:1.18}, $y=f(x)$ for some $x\in(a,b)$. Thus, since we require that $g(f(x'))=x'$ and $f(g(y'))=y'$ for $g$ to be an inverse function, we assign $g(y)=x$. Now suppose that $g(y)=x_1$ and $g(y)=x_2$. Then by the definition of $g$, $f(x_1)=y$ and $f(x_2)=y$. It follows that $f(x_1)=f(x_2)$, implying since $f$ is injective by Definition \ref{dfn:1.20} that $x_1=x_2$, as desired.\par
        To prove that $g$ is continuous, Definition \ref{dfn:9.15} tells us that it will suffice to show that for every $U\subset(a,b)$ that is open in $(a,b)$, the preimage $g^{-1}(U)$ is open in $f((a,b))$. Let $U$ be an arbitrary subset of $(a,b)$ that is open in $(a,b)$. To show that $g^{-1}(U)$ is open in $f((a,b))$, Definition \ref{dfn:8.11} tells us that it will suffice to confirm that $g^{-1}(U)=f((a,b))\cap G$, where $G$ is an open set.\par
        To begin, we have
        \begin{align*}
            g^{-1}(U) &= \{y\in f((a,b))\mid g(y)\in U\}\tag*{Definition \ref{dfn:1.18}}\\
            &= \{f(x)\in f((a,b))\mid g(f(x))\in U\}\tag*{Definition \ref{dfn:1.18}}
            \intertext{By the definition of $g$, we have $g(f(x))=x$.}
            &= \{f(x)\in f((a,b))\mid x\in U\}\\
            &= \{f(x)\in\{f(x')\in\R\mid x'\in(a,b)\}\mid x\in U\}\tag*{Definition \ref{dfn:1.18}}
            \intertext{This next transition is mostly notational in nature. $f(x)$ being an element of the set of all $f(x')\in\R$ that meet a certain condition means that $f(x)\in\R$. Additionally, since that condition is $x'\in(a,b)$, we know that $x\in(a,b)$. But if $x\in(a,b)$ and (from the condition in the original set) $x\in U$, we have by Definition \ref{dfn:1.6} that $x\in U\cap(a,b)$.}
            &= \{f(x)\in\R\mid x\in U\cap(a,b)\}\\
            &= f(U\cap(a,b))\tag*{Definition \ref{dfn:1.18}}
            \intertext{By definition, $U$ is open in $(a,b)$. Consequently, by Definition \ref{dfn:8.11}, $U=(a,b)\cap V$ where $V$ is open.}
            &= f(((a,b)\cap V)\cap(a,b))\\
            &= f((a,b)\cap V)\tag*{Definition \ref{dfn:1.6}}\\
            &= f((a,b))\cap f(V)\tag*{Additional Exercise \ref{axr:9.2b}}
        \end{align*}\par
        All that's left at this point is to prove that $f(V)$ is open. By Theorem \ref{trm:4.14}, $V=\bigcup_{\lambda\in I}\{R_\lambda\}$ is a collection of regions. It follows by an extension of Additional Exercise \ref{axr:9.2a} that $f(V)=\bigcup_{\lambda\in I}\{f(R_\lambda)\}$. Additionally, by the lemma, each $f(R_\lambda)$ is a region; hence, by Corollary \ref{cly:4.11}, each $f(R_\lambda)$ is open. Thus, $f(V)$ is the union of a collection of open subsets of $\R$, so by Corollary \ref{cly:4.18}, $f(V)$ is open.
    \end{proof}
\end{theorem}

We denote the inverse function $g$ by $f^{-1}$.
In this result, $g$ has codomain $(a,b)$ but our definition of continuity (Definition \ref{dfn:9.4}) only applies to functions with codomain $\R$. Our definitions/results are easily adapted. The definitions are as given below and we give a sample theorem. Other results can be adjusted in a similar fashion.

\begin{definition}\label{dfn:9.15}
    Let $X,Y\subset\R$. A function $f:X\to Y$ is \textbf{continuous} if for every $U$ that is open in $Y$, the preimage $f^{-1}(U)$ is open in $X$.
\end{definition}

\begin{definition}\label{dfn:9.16}
    The function $f:X\to Y$ is \textbf{continuous} (at $x\in X$) if for every region $R$ containing $f(x)$, there exists an open set $S$ containing $x$ such that $S\cap X\subset f^{-1}(R\cap Y)$.
\end{definition}

\begin{theorem}\label{trm:9.17}
    The function $f:X\to Y$ is continuous if and only if it is continuous at every $x\in X$.
\end{theorem}


\subsection*{Additional Exercises}
\begin{enumerate}[ref={\thechapter.\arabic*}]
    \stepcounter{enumi}
    \item \label{axr:9.2}Let $X\subset\R$ and let $f:X\to\R$. Let $A,B\subset\R$. Either prove or give a counterexample to each of the following:
    \begin{enumerate}[label={\alph*)},ref={\theenumi\alph*}]
        \item \label{axr:9.2a}$f(A\cup B)=f(A)\cup f(B)$.
        \item \label{axr:9.2b}$f(A\cap B)=f(A)\cap f(B)$.
        \item \label{axr:9.2c}$f(A\setminus B)=f(A)\setminus f(B)$.
    \end{enumerate}
\end{enumerate}




\end{document}