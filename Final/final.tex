\documentclass[../main.tex]{subfiles}

\pagestyle{main}
\renewcommand{\chaptermark}[1]{\markboth{#1}{}}

\begin{document}




\chapter*{Final-Specific Questions}\label{sct:finalSpecificQuestions2}
\addcontentsline{toc}{chapter}{Final-Specific Questions}
\chaptermark{Final-Specific Questions}
\begin{enumerate}
    \item Suppose that $X$ and $Y$ are compact subsets of $\R$. For this problem, use only results up to and including Theorem \ref{trm:10.11}, and not any of the subsequent results in Script \ref{sct:10}.
    \begin{enumerate}
        \item Show that $X\cup Y$ is compact.
        \item Show that $X\cap Y$ is compact.
        \item Suppose $X_1,X_2,\dots$ are compact sets. Are the following compact? Either prove that the set is always compact or provide a counterexample that is not compact.
        \begin{align*}
            \bigcap_{n\in\N}X_n&&
            \bigcup_{n\in\N}X_n
        \end{align*}
    \end{enumerate}
    \begin{proof}[Proof of a]
        % To prove that $X\cup Y$ is compact, Definition \ref{dfn:10.4} tells us that it will suffice to show that for every open cover $\mathcal{G}$ of $X\cup Y$, there exists a finite subset $\mathcal{G}'\subset\mathcal{G}$ that is also an open cover. Let $\mathcal{G}$ be an arbitrary open cover of $X\cup Y$. Define $\mathcal{H}=\bigcup_{x\in X}\{G\in\mathcal{G}:x\in G\}$ and $\mathcal{I}=\bigcup_{y\in Y}\{G\in\mathcal{G}:y\in G\}$. We now seek to demonstrate that $\mathcal{H}$ and $\mathcal{I}$ are open covers of $X$ and $Y$, respectively, starting with $\mathcal{H}$. To do so, Definition \ref{dfn:10.3} tells us that it will suffice to confirm that every $x\in X$ is an element of $H$ for some $H\in\mathcal{H}$, and that every $H$ is open. For the first condition, let $x$ be an arbitrary element of $X$. Then by Definition \ref{dfn:1.5}, $x\in X\cup Y$. It follows by Definition \ref{dfn:10.3} that $x\in G$ for some $G\in\mathcal{G}$. But by the definition of $\mathcal{H}$, this $G\in\mathcal{H}$, so $x$ is an element of a set in $\mathcal{H}$, as desired. As to the other condition, since every $H\in\mathcal{H}$ is an element of $\mathcal{G}$ (i.e., open by Definition \ref{dfn:10.3}), every $H\in\mathcal{H}$ is open, as desired. The argument is symmetric for $\mathcal{I}$.\par
        % Having established that $\mathcal{H}$ and $\mathcal{I}$ are open covers of compact sets $X$ and $Y$, respectively, we now invoke Definition \ref{dfn:10.4} to find finite subcovers $\mathcal{H}'\subset\mathcal{H}$ and $\mathcal{I}'\subset\mathcal{I}$. Let $\mathcal{G}'=\mathcal{H}'\cup\mathcal{I}'$. By Script \ref{sct:1}, $\mathcal{G}'\subset\mathcal{G}$. Additionally, since $\mathcal{H}'$ and $\mathcal{I}'$ are both finite, Script \ref{sct:1} also implies that $\mathcal{G}'$ is finite. However, to demonstrate that $\mathcal{G}'$ is an open cover of $X\cup Y$, Definition \ref{dfn:10.3} tells us that we must confirm that every $z\in X\cup Y$ is an element of $G$ for some $G\in\mathcal{G}'$, and that every $G$ is open. For the first condition, let $z$ be an arbitrary element of $X\cup Y$. Thus, by Definition \ref{dfn:1.5}, $z\in X$ or $z\in Y$. We now divide into two cases. If $z\in X$, then by Definition \ref{dfn:10.3}, $z\in H$ for some $H\in\mathcal{H}'$. But since $\mathcal{H}'\subset\mathcal{G}'$ by Script \ref{sct:1}, $z\in H$ implies that $z$ is an element of a set that is an element of $\mathcal{G}'$, as desired. The argument is symmetric in the other case. As to the other condition, since every $G\in\mathcal{G}'$ is an element of $\mathcal{H}'$ or $\mathcal{I}'$ (i.e., open either way by Definition \ref{dfn:10.3}), every $G\in\mathcal{G}'$ is open, as desired.

        To prove that $X\cup Y$ is compact, Definition \ref{dfn:10.4} tells us that it will suffice to show that for every open cover $\mathcal{G}$ of $X\cup Y$, there exists a finite subset $\mathcal{G}'\subset\mathcal{G}$ that is also an open cover. Let $\mathcal{G}$ be an arbitrary open cover of $X\cup Y$. We now seek to demonstrate that $\mathcal{G}$ is an open cover of $X$ and $Y$, starting with $X$. To do so, Definition \ref{dfn:10.3} tells us that it will suffice to confirm that every $x\in X$ is an element of $G$ for some $G\in\mathcal{G}$, and that every $G$ is open. For the first condition, let $x$ be an arbitrary element of $X$. Then by Definition \ref{dfn:1.5}, $x\in X\cup Y$. It follows by Definition \ref{dfn:10.3} that $x\in G$ for some $G\in\mathcal{G}$. As to the other condition, every $G\in\mathcal{G}$ is open by Definition \ref{dfn:10.3}, as desired. The argument is symmetric for $Y$.\par
        We now invoke Definition \ref{dfn:10.4} to find finite subcovers $\mathcal{G}_X\subset\mathcal{G}$ and $\mathcal{G}_Y\subset\mathcal{G}$ of $X$ and $Y$, respectively. Let $\mathcal{G}'=\mathcal{G}_X\cup\mathcal{G}_Y$. By Script \ref{sct:1}, $\mathcal{G}'\subset\mathcal{G}$. Additionally, since $\mathcal{G}_X$ and $\mathcal{G}_Y$ are both finite, Script \ref{sct:1} also implies that $\mathcal{G}'$ is finite. However, to demonstrate that $\mathcal{G}'$ is an open cover of $X\cup Y$, Definition \ref{dfn:10.3} tells us that we must confirm that every $z\in X\cup Y$ is an element of $G$ for some $G\in\mathcal{G}'$, and that every $G$ is open. For the first condition, let $z$ be an arbitrary element of $X\cup Y$. Thus, by Definition \ref{dfn:1.5}, $z\in X$ or $z\in Y$. We now divide into two cases. If $z\in X$, then by Definition \ref{dfn:10.3}, $z\in G$ for some $G\in\mathcal{G}_X$. But since $\mathcal{G}_X\subset\mathcal{G}'$ by Script \ref{sct:1}, $z\in G$ implies that $z$ is an element of a set that is an element of $\mathcal{G}'$, as desired. The argument is symmetric in the other case. As to the other condition, every $G\in\mathcal{G}$ is open by Definition \ref{dfn:10.3}, as desired.
    \end{proof}
    \begin{proof}[Proof of b]
        % To prove that $X\cap Y$ is compact, Definition \ref{dfn:10.4} tells us that it will suffice to show that for every open cover $\mathcal{G}$ of $X\cap Y$, there exists a finite subset $\mathcal{G}'\subset\mathcal{G}$ that is also an open cover. Let $\mathcal{G}$ be an arbitrary open cover of $X\cap Y$. Define $\mathcal{G}'$ as in part (a). To demonstrate that $\mathcal{G}'$ is an open cover of $X\cap Y$, Definition \ref{dfn:10.3} tells us that we must confirm that every $z\in X\cap Y$ is an element of $G$ for some $G\in\mathcal{G}'$, and that every $G$ is open. For the first condition, let $z$ be an arbitrary element of $X\cap Y$. Thus, by Definition \ref{dfn:1.5}, $z\in X$ and $z\in Y$. Since $z\in X$\footnote{This would work equally well if we considered $Y$.}, then by Definition \ref{dfn:10.3}, $z\in H$ for some $H\in\mathcal{H}'$. But since $\mathcal{H}'\subset\mathcal{G}'$ by Script \ref{sct:1}, $z\in H$ implies that $z$ is an element of a set that is an element of $\mathcal{G}'$, as desired. As to the other condition, since every $G\in\mathcal{G}'$ is an element of $\mathcal{H}'$ or $\mathcal{I}'$ (i.e., open either way by Definition \ref{dfn:10.3}), every $G\in\mathcal{G}'$ is open, as desired.

        To prove that $X\cap Y$ is compact, Definition \ref{dfn:10.4} tells us that it will suffice to show that for every open cover $\mathcal{G}$ of $X\cap Y$, there exists a finite subset $\mathcal{G}'\subset\mathcal{G}$ that is also an open cover. To do this, we will first demonstrate that $\mathcal{H}=\mathcal{G}\cup\{\R\setminus(X\cap Y)\}$ is an open cover of $X$. It will follow that there exists a finite subset $\mathcal{H}'\subset\mathcal{H}$ that is an open cover of $X$. Lastly, we will demonstrate that $\mathcal{G}'=\mathcal{H}'\setminus\{\R\setminus(X\cap Y)\}$ is the desired finite open cover subset of $\mathcal{G}$.\par
        Let $\mathcal{G}$ be an arbitrary open cover of $X\cap Y$, and let $\mathcal{H}=\mathcal{G}\cup\{\R\setminus(X\cap Y)\}$. To demonstrate that $\mathcal{H}$ is an open cover of $X$, Definition \ref{dfn:10.3} tells us that it will suffice to confirm that every $x\in X$ is an element of $H$ for some $H\in\mathcal{H}$, and that every $H$ is open. For the first condition, let $x$ be an arbitrary element of $X$. We divide into two cases ($x\in X\cap Y$ and $x\notin X\cap Y$). If $x\in X\cap Y$, then since $\mathcal{G}$ is an open cover of $X\cap Y$, Definition \ref{dfn:10.3} implies that $x\in G$ for some $G\in\mathcal{G}$. But since $\mathcal{G}\subset\mathcal{H}$, $x\in G$ for some $G\in\mathcal{H}$, as desired. On the other hand, if $x\notin X\cap Y$, then this combined with the fact that $x\in\R$ implies by Definition \ref{dfn:1.11} that $x\in\R\setminus(X\cap Y)\in\mathcal{H}$, as desired. As to the other condition, let $H$ be an arbitrary element of $\mathcal{H}$. We divide into two cases ($H\in\mathcal{G}$ and $H\notin\mathcal{G}$). If $H\in\mathcal{G}$, then by Definition \ref{dfn:10.3}, $H$ is open, as desired. On the other hand, if $H\notin\mathcal{G}$, then $H=\R\setminus(X\cap Y)$ by Script \ref{sct:1}. $X$ and $Y$ are closed by Theorem \ref{trm:10.11}, so $X\cap Y$ is closed by Theorem \ref{trm:4.16}. It follows by Definition \ref{dfn:4.8} that $\R\setminus(X\cap Y)$ is open, so $H$ is open, as desired.\par
        Since $\mathcal{H}$ is an open cover of $X$ and $X$ is compact, we have by Definition \ref{dfn:10.4} that there exists a finite subset $\mathcal{H}'\subset\mathcal{H}$ that is also an open cover of $X$. Let $\mathcal{G}'=\mathcal{H}'\setminus\{\R\setminus(X\cap Y)\}$. By Script \ref{sct:1}, $\mathcal{G}'\subset\mathcal{G}$. Additionally, since $\mathcal{H}'$ is finite, Script \ref{sct:1} also implies that $\mathcal{G}'$ is finite. However, to demonstrate that $\mathcal{G}'$ is an open cover of $X\cap Y$, Definition \ref{dfn:10.3} tells us that we must confirm that every $z\in X\cap Y$ is an element of $G$ for some $G\in\mathcal{G}'$, and that every $G$ is open. For the first condition, let $z$ be an arbitrary element of $X\cap Y$. Then since $X\cap Y\subset X$ by Theorem \ref{trm:1.7}, Definition \ref{dfn:10.3} asserts that $z\in H$ for some $H\in\mathcal{H}'$. Additionally, since $z\in X\cap Y$, Definition \ref{dfn:1.11} implies that $z\notin\R\setminus(X\cap Y)$. Thus, $H\neq\R\setminus(X\cap Y)$, which implies by Definition \ref{dfn:1.11} that $H\in\mathcal{H}'\setminus\{\R\setminus(X\cap Y)\}=\mathcal{G}'$. Therefore, $z\in H$ for some $H\in\mathcal{G}'$, as desired. As to the other condition, since every $G\in\mathcal{G}'$ is an element of $\mathcal{G}$ (i.e., open by Definition \ref{dfn:10.3}), every $G\in\mathcal{G}'$ is open, as desired.
    \end{proof}
    \begin{proof}[Proof of c]
        To prove that $\bigcap_{n\in\N}X_n$ is compact, Definition \ref{dfn:10.4} tells us that it will suffice to show that for every open cover $\mathcal{G}$ of $\bigcap_{n\in\N}X_n$, there exists a finite subset $\mathcal{G}'\subset\mathcal{G}$ that is also an open cover. To do this, we will use an analogous process to part (b).\par
        Let $\mathcal{G}$ be an arbitrary open cover of $\bigcap_{n\in\N}X_n$, and let $\mathcal{H}=\mathcal{G}\cup\{\R\setminus(\bigcap_{n\in\N}X_n)\}$. To demonstrate that $\mathcal{H}$ is an open cover of $X_1$, Definition \ref{dfn:10.3} tells us that it will suffice to confirm that every $x\in X_1$ is an element of $H$ for some $H\in\mathcal{H}$, and that every $H$ is open. For the first condition, let $x$ be an arbitrary element of $X_1$. We divide into two cases ($x\in\bigcap_{n\in\N}X_n$ and $x\notin\bigcap_{n\in\N}X_n$). If $x\in\bigcap_{n\in\N}X_n$, then since $\mathcal{G}$ is an open cover of $\bigcap_{n\in\N}X_n$, Definition \ref{dfn:10.3} implies that $x\in G$ for some $G\in\mathcal{G}$. But since $\mathcal{G}\subset\mathcal{H}$, $x\in G$ for some $G\in\mathcal{H}$, as desired. On the other hand, if $x\notin\bigcap_{n\in\N}X_n$, then this combined with the fact that $x\in\R$ implies by Definition \ref{dfn:1.11} that $x\in\R\setminus(\bigcap_{n\in\N}X_n)\in\mathcal{H}$, as desired. As to the other condition, let $H$ be an arbitrary element of $\mathcal{H}$. We divide into two cases ($H\in\mathcal{G}$ and $H\notin\mathcal{G}$). If $H\in\mathcal{G}$, then by Definition \ref{dfn:10.3}, $H$ is open, as desired. On the other hand, if $H\notin\mathcal{G}$, then $H=\R\setminus(\bigcap_{n\in\N}X_n)$ by Script \ref{sct:1}. Each $X_n$ is closed by Theorem \ref{trm:10.11}, so $\bigcap_{n\in\N}X_n$ is closed by Theorem \ref{trm:4.16}. It follows by Definition \ref{dfn:4.8} that $\R\setminus(\bigcap_{n\in\N}X_n)$ is open, so $H$ is open, as desired.\par
        Since $\mathcal{H}$ is an open cover of $X_1$ and $X_1$ is compact, we have by Definition \ref{dfn:10.4} that there exists a finite subset $\mathcal{H}'\subset\mathcal{H}$ that is also an open cover of $X_1$. Let $\mathcal{G}'=\mathcal{H}'\setminus\{\R\setminus(\bigcap_{n\in\N}X_n)\}$. By Script \ref{sct:1}, $\mathcal{G}'\subset\mathcal{G}$. Additionally, since $\mathcal{H}'$ is finite, Script \ref{sct:1} also implies that $\mathcal{G}'$ is finite. However, to demonstrate that $\mathcal{G}'$ is an open cover of $\bigcap_{n\in\N}X_n$, Definition \ref{dfn:10.3} tells us that we must confirm that every $y\in\bigcap_{n\in\N}X_n$ is an element of $G$ for some $G\in\mathcal{G}'$, and that every $G$ is open. For the first condition, let $y$ be an arbitrary element of $\bigcap_{n\in\N}X_n$. Then since $\bigcap_{n\in\N}X_n\subset X_1$, Definition \ref{dfn:10.3} asserts that $y\in H$ for some $H\in\mathcal{H}'$. Additionally, since $y\in\bigcap_{n\in\N}X_n$, Definition \ref{dfn:1.11} implies that $y\notin\R\setminus(\bigcap_{n\in\N}X_n)$. Thus, $H\neq\R\setminus(\bigcap_{n\in\N}X_n)$, which implies by Definition \ref{dfn:1.11} that $H\in\mathcal{H}'\setminus\{\R\setminus(\bigcap_{n\in\N}X_n)\}=\mathcal{G}'$. Therefore, $y\in H$ for some $H\in\mathcal{G}'$, as desired. As to the other condition, since every $G\in\mathcal{G}'$ is an element of $\mathcal{G}$ (i.e., open by Definition \ref{dfn:10.3}), every $G\in\mathcal{G}'$ is open, as desired.\par\medskip
        Let $X_n=\{n\}$ for all $n\in\N$. By Exercise \ref{exr:10.5}, each $X_n$ is compact, as desired. However, $\bigcup_{n\in\N}X_n=\N$ is not compact: if we let $\mathcal{G}=\{(n-1,n+1):n\in\N\}$, then we have an open cover (clearly) that is infinite (clearly) and yet from which no term can be removed without revoking its status as an open cover.
    \end{proof}
    \item Let $f,g:A\to\R$. In each of the following, justify your answer fully:
    \begin{enumerate}
        \item If $\lim_{x\to a}f(x)$ and $\lim_{x\to a}g(x)$ do not exist, can $\lim_{x\to a}[f(x)+g(x)]$ exist?
        \item If $\lim_{x\to a}f(x)$ exists and $\lim_{x\to a}[f(x)+g(x)]$ exists, must $\lim_{x\to a}g(x)$ exist?
    \end{enumerate}
    \begin{proof}[Justification of a]
        Let $f,g:A\to\R$ be defined by
        \begin{align*}
            f(x) &=
            \begin{cases}
                1 & x\geq 0\\
                0 & x<0
            \end{cases}&
            g(x) &=
            \begin{cases}
                0 & x\geq 0\\
                1 & x<0
            \end{cases}
        \end{align*}
        By Exercise \ref{exr:11.4}, $\lim_{x\to 0}f(x)$ does not exist. Similarly, $\lim_{x\to 0}g(x)$ does not exist. However, by Exercise \ref{exr:11.6}, $\lim_{x\to 0}[f(x)+g(x)]=1$.
    \end{proof}
    \begin{proof}[Justification of b]
        Let $\lim_{x\to a}f(x)=L$ and $\lim_{x\to a}[f(x)+g(x)]=M$. If we let $h(x)=-1$, then we have by Theorem \ref{trm:11.9} that $\lim_{x\to a}-f(x)=-L$. Applying Theorem \ref{trm:11.9} again, we have $\lim_{x\to a}[f(x)+g(x)-f(x)]=\lim_{x\to a}g(x)=M-L$. 
    \end{proof}
    \item For the problem you may assume $f:\R\to\R$.
    \begin{enumerate}
        \item Show that for all $x\in\R$ there exists a unique $y\in\R$ such that $y^3=x$.
        \item Using part (a), we can define the cube root function $g(x)=x^{1/3}$ in the usual way. Show that $g$ is continuous and strictly increasing.
        \item Suppose $\lim_{x\to 0}f(x)=L$. Show that $\lim_{x\to 0}f(x^3)$ exists and equals $L$.
        \item Suppose $\lim_{x\to 0}f(x^3)=L$. Show that $\lim_{x\to 0}f(x)$ exists and equals $L$.
    \end{enumerate}
    \begin{proof}[Proof of a]
        We first show that there always \emph{exists} a $y$ such that $y^3=x$. Let $f:\R\to\R$ be defined by $f(y)=y^3$, and let $x$ be an arbitrary element of $\R$. By Definition \ref{dfn:11.11}, $f$ is a polynomial, meaning by Corollary \ref{cly:11.12} that $f$ is continuous. We divide into three cases ($x=0$, $x>0$, and $x<0$). If $x=0$, then $x=0=0^3=y^3$. If $x>0$, consider the closed interval $[0,x+1]$. By Proposition \ref{prp:9.7}, $f|_{[0,x+1]}:[0,x+1]\to\R$ is continuous. Additionally, $f(0)=0<x<x^3+3x^2+3x+1=(x+1)^3=f(x+1)$. Thus, we have by Exercise \ref{exr:9.12} that there exists $y\in(0,x+1)$ such that $f(y)=y^3=x$. The argument is symmetric if $x<0$.\par
        % We now show that this $y$ is unique, by proving that $f$ is injective. To do so, let $x,y$ be arbitrary elements of $\R$ such that $x\neq y$. WLOG let $x>y$. To ensure that $x^3\neq y^3$, we must have $x^3-y^3\neq 0$. Thus, we investigate the conditions under which $x^3-y^3=0$, namely when $0=(x-y)(x^2+xy+y^2)$, i.e., when $x=y$ or $x=y=0$ (the latter result is trivial, but obtained by setting $x^2+xy+y^2=0$). Essentially, what this tells us is that $x^3\neq y^3$ only when $x\neq y$, as desired.
        We now show that this $y$ is unique, by proving that $f$ is injective. To do so, Definition \ref{dfn:1.20} tells us that it will suffice to show that if $f(x)=f(x')$, then $x=x'$. Let $x^3=x'^3$. Then $0=x^3-x'^3=(x-x')(x^2+xx'+x'^2)$. It follows by Script \ref{sct:0} that $x=x'$ or $x=x'=0$ (the latter result is trivial, but obtained by setting $x^2+xx'+x'^2=0$), as desired.
    \end{proof}
    \begin{proof}[Proof of b]
        Suppose for the sake of contradiction that $g$ is not continuous. Then by Theorem \ref{trm:9.10}, there exists some $x\in\R$ at which $g$ is not continuous. It follows by Theorem \ref{trm:11.5} that $\lim_{y\to x}g(y)\neq g(x)$. Additionally, by part (a), there exists some $z\in\R$ such that $f(z)=x$. Also, since $f$ is continuous by part (a), Theorem \ref{trm:9.10} asserts that it is continuous at at $x$. Thus, by Theorem \ref{trm:11.5}, $\lim_{y\to z}f(y)=f(z)$ (we know that $z\in LP(\R)$ by Corollary \ref{cly:5.4}). Consequently, $\lim_{y\to z}g(f(y))$ does not exist. But this contradicts Exercise \ref{exr:11.6}, which asserts that $g(f(x))=x$ is continuous, i.e., $\lim_{y\to z}g(f(y))$ should exist.\par
        By the proof of part (a), $f$ is injective. Additionally, by part (a), $f$ is surjective. Thus, by Definition \ref{dfn:1.20}, $f$ is bijective. It follows by Proposition \ref{prp:1.27} that $g$ is bijective. Thus, by Definition \ref{dfn:1.20} again, $g$ is injective. Consequently, by Proposition \ref{prp:9.7}, $g|_{(a,b)}:(a,b)\to\R$ is continuous for any open interval $(a,b)$, and by Script \ref{sct:1}, $g|_{(a,b)}$ is also injective. Thus, by Lemma \ref{lem:9.13}, $g$ is strictly increasing on any open interval $(a,b)$. It follows that $g$ is strictly increasing overall.
    \end{proof}
    \begin{proof}[Proof of c]
        To prove that $\lim_{x\to 0}f(x^3)=L$, Definition \ref{dfn:11.1} tells us that it will suffice to show that for every $\epsilon>0$, there exists a $\delta>0$ such that if $x\in\R$ and $0<|x|<\delta$, $|f(x^3)-L|<\epsilon$. Let $\epsilon>0$ be arbitrary. It follows from the hypothesis that $\lim_{x\to 0}f(x)=L$ by Definition \ref{dfn:11.1} that there exists $\delta_1>0$ such that if $x\in\R$ and $0<|x|<\delta_1$, then $|f(x)-L|<\epsilon$. Additionally, by Theorem \ref{trm:11.5}, the facts that the $x^3$ function is continuous and $0\in LP(\R)$ imply that $\lim_{x\to 0}x^3=0^3=0$. Thus, by Definition \ref{dfn:11.1}, there exists a $\delta_2>0$ such that if $x\in\R$ and $0<|x|<\delta_2$, then $|x^3-0|=|x^3|<\delta_1$. Choose $\delta=\delta_2$. Then if $x$ is an arbitrary element of $\R$ such that $0<|x|<\delta$, we have $|x^3|<\delta_1$. Additionally, since $x\neq 0$, $x^3\neq 0$, so $0<|x^3|<\delta_1$. It follows that $|f(x^3)-L|<\epsilon$, as desired.
    \end{proof}
    \begin{proof}[Proof of d]
        To prove that $\lim_{x\to 0}f(x)=L$, Definition \ref{dfn:11.1} tells us that it will suffice to show that for every $\epsilon>0$, there exists a $\delta>0$ such that if $x\in\R$ and $0<|x|<\delta$, $|f(x)-L|<\epsilon$. Let $\epsilon>0$ be arbitrary. It follows from the hypothesis that $\lim_{x\to 0}f(x^3)=L$ by Definition \ref{dfn:11.1} that there exists $\delta_1>0$ such that if $x\in\R$ and $0<|x|<\delta_1$, then $|f(x^3)-L|<\epsilon$. Additionally, by Theorem \ref{trm:11.5}, the facts that the $x^{1/3}$ function is continuous (part (b)) and $0\in LP(\R)$ imply that $\lim_{x\to 0}x^{1/3}=0^{1/3}=0$. Thus, by Definition \ref{dfn:11.1}, there exists a $\delta_2>0$ such that if $x\in\R$ and $0<|x|<\delta_2$, then $|x^{1/3}-0|=|x^{1/3}|<\delta_1$. Choose $\delta=\delta_2$. Then if $x$ is an arbitrary element of $\R$ such that $0<|x|<\delta$, we have $|x^{1/3}|<\delta_1$. Additionally, since $x\neq 0$, $x^{1/3}\neq 0$, so $0<|x^{1/3}|<\delta_1$. It follows that $|f((x^{1/3})^3)-L|=|f(x)-L|<\epsilon$, as desired.
    \end{proof}
    \item For this problem, suppose $f:\R\to\R$, $g:\R\to\R$, and that $A$ is a dense subset of $\R$.
    \begin{enumerate}
        \item Prove that if $f$ is continuous and $f(x)=0$ for all $x\in A$, then $f(x)=0$ for all $x\in\R$.
        \item Prove that if $f$ and $g$ are continuous and $f(x)=g(x)$ for all $x\in A$, then $f(x)=g(x)$ for all $x\in\R$.
    \end{enumerate}
    \begin{proof}[Proof of a]
        Suppose for the sake of contradiction that $f(x)\neq 0$ for some $x\in\R$. Since $f$ is continuous, Theorem \ref{trm:9.10} asserts that $f$ is continuous at $x$. It follows by Theorem \ref{trm:11.5} that $\lim_{y\to x}f(y)=f(x)$ (since $x\in LP(\R)$ by Corollary \ref{cly:5.4}). Choose $\epsilon=|f(x)|$. Consequently, by Definition \ref{dfn:11.1}, there exists a $\delta>0$ such that if $y\in\R$ and $|y-x|<\delta$, then $|f(y)-f(x)|<|f(x)|$.\par
        Switching gears for a moment, consider the fact that $A$ is dense in $\R$. It follows by Definition \ref{dfn:6.8} that $x\in LP(A)$. Thus, by Definition \ref{dfn:3.13}, for every region $R$ with $x\in R$, $R\cap(A\setminus\{x\})\neq\emptyset$.\par
        We merge the above two ideas by considering the region $R=(x-\delta,x+\delta)$, which is clearly an $x$-containing region. By the above, $R\cap(A\setminus\{x\})\neq\emptyset$, i.e., there exists a point $y$ such that $y\in R$, $y\in A$, and $y\neq x$. It follows from the former claim by Exercise \ref{exr:8.9} that $|y-x|<\delta$, from the middle claim by hypothesis that $f(y)=0$, and from the latter claim that $y-x\neq 0$, i.e., $0<|y-x|<\delta$. Therefore, $|f(y)-f(x)|<|f(x)|$. But this implies that $|0-f(x)|=|f(x)|<|f(x)|$, a contradiction.
    \end{proof}
    \begin{proof}[Proof of b]
        Let $h(x)=f(x)-g(x)$. We will prove that $h$ is continuous and that $h(x)=0$ for all $x\in A$. It will then follow from part (a) that $h(x)=0$ for all $x\in\R$, implying that $f(x)=g(x)$ for all $x\in\R$. Let's begin.\par
        Since $f,g$ are continuous by Theorem \ref{trm:9.10}, $f,g$ are continuous at every $x\in\R$. Thus, by consecutive applications of Corollary \ref{cly:11.10}, $-g$ is continuous at every $x\in\R$, so $f-g$ is continuous at every $x\in\R$. Consequently, by Theorem \ref{trm:9.10} again, $h=f-g$ is continuous.\par
        Since $f(x)=g(x)$ for all $x\in A$, it naturally follows that $h(x)=f(x)-g(x)=0$ for all $x\in A$.\par
        Since $h$ is continuous and $h(x)=0$ for all $x\in A$, part (a) asserts that $h(x)=0$ for all $x\in\R$. Thus, $f(x)-g(x)=0$ for all $x\in\R$, meaning that $f(x)=g(x)$ for all $x\in\R$, as desired.
    \end{proof}
\end{enumerate}




\end{document}