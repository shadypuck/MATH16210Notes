\documentclass[../main.tex]{subfiles}

\pagestyle{main}
\renewcommand{\chaptermark}[1]{\markboth{\chaptername\ \thechapter}{}}
\setcounter{chapter}{9}

\begin{document}




\chapter{Compactness}\label{sct:10}
\section{Journal}
\begin{definition}\label{dfn:10.1}\marginnote{2/23:}
    We say that a function $f:A\to\R$ is \textbf{bounded} if $f(A)$ is a bounded subset of $\R$. We say that $f$ is \textbf{bounded above} if $f(A)$ is bounded above and that $f$ is \textbf{bounded below} if $f(A)$ is bounded below.\par
    If $f:A\to\R$ is bounded above, we say that $f$ \textbf{attains} (its least upper bound) if there is some $a\in A$ such that $f(a)=\sup f(A)$. Similarly, if $f:A\to\R$ is bounded below, we say that $f$ \textbf{attains} (its greatest lower bound) if there is some $a\in A$ such that $f(a)=\inf f(A)$.
\end{definition}




\end{document}