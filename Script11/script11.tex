\documentclass[../main.tex]{subfiles}

\pagestyle{main}
\renewcommand{\chaptermark}[1]{\markboth{\chaptername\ \thechapter}{}}
\setcounter{chapter}{10}

\begin{document}




\chapter{Limits and Continuity}\label{sct:11}
\section{Journal}
\marginnote{3/4:}Throughout this sheet, we let $f,g:A\to\R$ be real-valued functions with domain $A\subset\R$, unless otherwise specified.
\begin{definition}\label{dfn:11.1}
    Let $a\in LP(A)\subset\R$. A \textbf{limit} of $f$ at $a$ is a number $L\in\R$ satisfying the following condition: for every $\epsilon>0$, there exists a $\delta>0$ such that if $x\in A$ and $0<|x-a|<\delta$, then $|f(x)-L|<\epsilon$.
\end{definition}

\begin{lemma}\label{lem:11.2}
    Limits are unique: if $L$ and $L'$ are both limits of $f$ at a point $a$, then $L=L'$.
    \begin{proof}
        Let the limit of $f$ at $a$ be $L$, and suppose for the sake of contradiction that the limit of $f$ at $a$ is also equal to $L'$ where $L\neq L'$. If we let $\epsilon=\frac{|L-L'|}{2}$, then $\epsilon>0$ by Script \ref{sct:8}. Thus, by consecutive applications of Definition \ref{dfn:11.1}, we have that there exists a $\delta_1>0$ such that if $x\in A$ and $0<|x-a|<\delta_1$, then $|f(x)-L|<\epsilon$; and that there exists a $\delta_2>0$ such that if $x\in A$ and $0<|x-a|<\delta_2$, then $|f(x)-L'|<\epsilon$. Now choose $\delta=\min(\delta_1,\delta_2)$. This makes it so that for any $x\in A$ such that $0<|x-a|<\delta$, we have $|f(x)-L|<\epsilon$ and $|f(x)-L'|<\epsilon$, so
        \begin{align*}
            |L-L'| &= |L-f(x)+f(x)-L'|\\
            &\leq |L-f(x)|+|f(x)-L'|\tag*{Lemma \ref{lem:8.8}}\\
            &= |f(x)-L|+|f(x)-L'|\tag*{Exercise \ref{exr:8.5}}\\
            &< 2\epsilon\\
            &= |L-L'|
        \end{align*}
        But this implies that $|L-L'|<|L-L'|$, a contradiction.
    \end{proof}
\end{lemma}

\begin{definition}\label{dfn:11.3}
    If $L$ is the limit of $f$ at $a$, we write
    \begin{equation*}
        \lim_{x\to a}f(x) = L
    \end{equation*}
\end{definition}

\begin{exercise}\label{exr:11.4}
    Give an example of a set $A\subset\R$, a function $f:A\to\R$, and a point $a\in LP(A)$ such that $\lim_{x\to a}f(x)$ does not exist.
    \begin{proof}
        Let $A=\R$, let $f:\R\to\R$ be defined by
        \begin{equation*}
            f(x) =
            \begin{cases}
                1 & x\geq 0\\
                0 & x<0
            \end{cases}
        \end{equation*}
        and consider $0\in LP(\R)$ (by Corollary \ref{cly:5.4}). Now suppose for the sake of contradiction that $\lim_{x\to a}f(x)=L$. Then by Definitions \ref{dfn:11.3}\footnote{I will not cite this definition again for the sake of concision.} and \ref{dfn:11.1}, for all $\epsilon>0$, there exists some $\delta>0$ such that if $x\in\R$ and $0<|x-0|=|x|<\delta$, then $|f(x)-L|<\epsilon$. If we let $\epsilon=0.5$, then $\epsilon>0$. Choosing a corresponding $\delta$, we have by an extension of Exercise \ref{exr:8.9} that all $x\in(-\delta,0)\cup(0,\delta)$ satisfy $|f(x)-L|<\epsilon$. This would include objects $y\in(0,\delta)$ and $z\in(-\delta,0)$. We have by the definition of $f$ that $f(y)=1$ and $f(z)=0$; thus, we have
        \begin{align*}
            1 &= |f(y)-f(z)|\\
            &= |f(y)-L+L-f(z)|\\
            &\leq |f(y)-L|+|f(z)-L|\\
            &< 0.5+0.5\\
            &= 1
        \end{align*}
        But this implies that $1<1$, a contradiction.
    \end{proof}
\end{exercise}

\begin{theorem}\label{trm:11.5}
    Let $x\in A$. Then the following are equivalent:
    \begin{enumerate}[label={\textup{(}\alph*\textup{)}}]
        \item $f$ is continuous at $x$.
        \item For every $\epsilon>0$, there exists $\delta>0$ such that if $y\in A$ and $|y-x|<\delta$, then $|f(y)-f(x)|<\epsilon$.
        \item Either $x\notin LP(A)$ or $\lim_{y\to x}f(y)=f(x)$.
    \end{enumerate}
    \begin{proof}
        To illustrate that statements $a$-$c$ are equivalent, it will suffice to verify that $a\Rightarrow b$, $b\Rightarrow c$, and $c\Rightarrow a$. Note that this foregoes the need for explicit proofs of "backwards implications" such as $b\Rightarrow a$ since that implication, for example, follows from $b\Rightarrow c\Rightarrow a$. Let's begin.\par\smallskip
        To prove that $a\Rightarrow b$, let $\epsilon>0$ be arbitrary and look to find a $\delta>0$ such that if $y\in A$ and $|y-x|<\delta$, then $|f(y)-f(x)|$.\par
        We first locate $\delta$. To do so, begin by defining the region $R=(f(x)-\epsilon,f(x)+\epsilon)$ (clearly $R$ contains $f(x)$). Since $R$ is open by Corollary \ref{cly:4.11} and $f$ is continuous at $x$, we have by Definition \ref{dfn:9.9} that there exists an open set $S$ with $x\in S$ such that $S\cap A\subset f^{-1}(R)$. It follows by Theorem \ref{trm:4.10} that there exists a region $(a,b)$ such that $x\in(a,b)$ and $(a,b)\subset S$. Thus, since $(a,b)$ is an open interval by Corollary \ref{cly:4.11} and Lemma \ref{lem:8.3}, we have by Lemma \ref{lem:8.10} that there exists a number $\delta>0$ such that $(x-\delta,x+\delta)\subset(a,b)$.\par
        As we will now show, this $\delta$ satisfies the desired property. Let $y$ be an arbitrary element of $A$ such that $|y-x|<\delta$. Then by Exercise \ref{exr:8.9}, $y\in(x-\delta,x+\delta)$. It follows by consecutive applications of Definition \ref{dfn:1.3} that $y\in (a,b)$, hence $y\in S$. This result combined with the fact that $y\in A$ by definition implies by Definition \ref{dfn:1.6} that $y\in S\cap A$. Thus, by Definition \ref{dfn:1.3} again, $y\in f^{-1}(R)$. Consequently, by Definition \ref{dfn:1.18}, $f(y)\in R$. Therefore, by Exercise \ref{exr:8.9} one more time, $|f(y)-f(x)|<\epsilon$.\par\smallskip
        To prove that $b\Rightarrow c$, let $x$ be an arbitrary element of $\R$. We divide into two cases ($x\notin LP(A)$ and $x\in LP(A)$). If $x\notin LP(A)$, then we are done. If $x\in LP(A)$, then by the hypothesis, we know that for every $\epsilon>0$, there exists a $\delta>0$ such that if $y\in A$ and $0<|y-x|<\delta$, then $|f(y)-f(x)|<\epsilon$. It follows by Definition \ref{dfn:11.1} that $f(x)$ is the limit of $f$ at $x$, meaning that $\lim_{y\to x}f(y)=f(x)$, and we are done.\par\smallskip
        To prove that $c\Rightarrow a$, we divide into two cases ($x\notin LP(A)$ and $\lim_{y\to x}f(y)=f(x)$).\par
        Suppose first that $x\notin LP(A)$. To demonstrate that $f$ is continuous at $x$, Definition \ref{dfn:9.9} tells us that it will suffice to confirm that for every region $R$ containing $f(x)$, there exists an open set $S$ containing $x$ such that $S\cap A\subset f^{-1}(R)$. Let $R$ be an arbitrary region with $f(x)\in R$. Since $x\notin LP(A)$, Definition \ref{dfn:3.13} asserts that there exists a region (hence an open set by Corollary \ref{cly:4.11}) $S$ such that $x\in S$ and $S\cap(A\setminus\{x\})=\emptyset$. It follows by Script \ref{sct:1} that $S\cap A=\{x\}$. But since $f(x)\in R$ implies by Definition \ref{dfn:1.18} that $x\in f^{-1}(R)$, we have by Definition \ref{dfn:1.3} that $S\cap A\subset f^{-1}(R)$. Therefore, $S$ is an open set containing $x$ such that $S\cap A\subset f^{-1}(R)$.\par
        Now suppose that $\lim_{y\to x}f(y)=f(x)$. To demonstrate that $f$ is continuous at $x$, Definition \ref{dfn:9.9} tells us that it will suffice to confirm that for every region $(a,b)$ containing $f(x)$, there exists an open set $S$ containing $x$ such that $S\cap A\subset f^{-1}((a,b))$. Let $(a,b)$ be an arbitrary region with $f(x)\in(a,b)$. Then since $(a,b)$ is an open interval by Lemma \ref{lem:8.3}, Lemma \ref{lem:8.10} asserts that there exists $\epsilon>0$ such that $(f(x)-\epsilon,f(x)+\epsilon)\subset(a,b)$. With regard to this $\epsilon$, since $\lim_{y\to x}f(y)=f(x)$ by hypothesis, we have by Definition \ref{dfn:11.1} that there exists a $\delta>0$ such that if $y\in A$ and $0<|y-x|<\delta$, then $|f(y)-f(x)|<\epsilon$. Let $S=(x-\delta,x+\delta)$. Clearly, $S$ contains $x$. Additionally, we can confirm that $S\cap A\subset f^{-1}((a,b))$: if we let $y$ be an arbitrary element of $S\cap A$, then Definition \ref{dfn:1.6} asserts that $y\in S$ and $y\in A$. It follows from the former condition by Exercise \ref{exr:8.9} that $|y-x|<\delta$. This combined with the fact that $y\in A$ implies that $|f(y)-f(x)|<\epsilon$. Thus, by Exercise \ref{exr:8.9} again, $f(y)\in(f(x)-\epsilon,f(x)+\epsilon)$. Consequently, by Definition \ref{dfn:1.3}, $f(y)\in(a,b)$. As such, we have by Definition \ref{dfn:1.18} that $y\in f^{-1}((a,b))$, as desired.
    \end{proof}
\end{theorem}

\begin{exercise}\label{exr:11.6}\leavevmode
    \begin{enumerate}[label={\textup{(}\alph*\textup{)}}]
        \item Let $a,b\in\R$ and let $f:\R\to\R$ be given by $f(x)=ax+b$. Show that $f$ is continuous at every $x\in\R$.
        \item Let $f:\R\to\R$ be given by $
            f(x) =
            \begin{cases}
                1 & x\neq 0\\
                0 & x=0
            \end{cases}
        $. Show that $f$ is not continuous at 0.
    \end{enumerate}
    \begin{proof}[Proof of a]
        To prove that $f$ is continuous at every $x\in\R$, let $x$ be an arbitrary element of $\R$; then by Theorem \ref{trm:11.5}, it will suffice to show that for every $\epsilon>0$, there exists a $\delta>0$ such that if $y\in A$ and $|y-x|<\delta$, then $|f(y)-f(x)|<\epsilon$. Let $\epsilon>0$ be arbitrary. We divide into two cases ($a=0$ and $a\neq 0$). If $a=0$, then choose $\delta=1$\footnote{This choice is arbitrary; it can be any nonzero value, as we will soon see.}. This makes it so that for any $y\in A$ such that $|y-x|<\delta=1$, we have $|f(y)-f(x)|=|b-b|=0<\epsilon$, as desired. If $a\neq 0$, then choose $\delta=\frac{\epsilon}{|a|}$. This makes it so that for any $y\in A$ such that $|y-x|<\delta=\frac{\epsilon}{|a|}$, we have
        \begin{align*}
            |a|\, |y-x| &< \epsilon\\
            |ay-ax| &< \epsilon\\
            |ay+b-(ax+b)| &< \epsilon\\
            |f(y)-f(x)| &< \epsilon
        \end{align*}
        as desired.
    \end{proof}
    \begin{proof}[Proof of b]
        To prove that $f$ is not continuous at $0$, Theorem \ref{trm:11.5} tells us that it will suffice to show that for some $\epsilon>0$, no $\delta>0$ exists such that if $x\in\R$ and $|x-0|=|x|<\delta$, then $|f(x)-1|<\epsilon$. Let $\epsilon=1$, and suppose for the sake of contradiction that $\delta>0$ is a number such that if $x\in\R$ and $|x|<\delta$, then $|f(x)-1|<\epsilon$. Clearly, $0\in\R$ and by the definition of $\delta$ and Definition \ref{dfn:8.4}, $|0|<\delta$. However, $|f(x)-1|=|0-1|=1\not<1=\epsilon$, a contradiction.
    \end{proof}
\end{exercise}

\begin{exercise}\label{exr:11.7}\marginnote{3/9:}
    Show that the absolute value function $f:\R\to\R$, $f(x)=|x|$ is continuous.
    \begin{proof}
        To prove that the absolute value function is continuous, Theorem \ref{trm:9.10} tells us that it will suffice to show that it is continuous at every $x\in\R$. To do this, let $x$ be an arbitrary element of $\R$; then by Theorem \ref{trm:11.5}, it will suffice to verify that for every $\epsilon>0$, there exists a $\delta>0$ such that if $y\in\R$ and $|y-x|<\delta$, then $||y|-|x||<\epsilon$. Let $\epsilon>0$ be arbitrary. Then choose $\delta=\epsilon$. This makes it so that for any $y\in A$ such that $|y-x|<\delta$, we have $||y|-|x||\leq|y-x|<\delta=\epsilon$ (with the first inequality coming from Lemma \ref{lem:8.8}), as desired.
    \end{proof}
\end{exercise}

Given real-valued functions $f$ and $g$, we define new functions $f+g$, $fg$, and $\frac{1}{f}$ by
\begin{align*}
    (f+g)(x) &= f(x)+g(x)&
    (fg)(x) &= f(x)\cdot g(x)&
    \frac{1}{f}(x) &= \frac{1}{f(x)}
\end{align*}
where $f(x)\neq 0$ in the definition of $\frac{1}{f}$. We wish to understand the limits of $f+g$, $fg$, and $\frac{1}{f}$ in terms of the limits of $f$ and $g$.

\begin{lemma}\label{lem:11.8}
    If $\lim_{x\to a}f(x)=L>0$, then there exists a region $R$ with $a\in R$ such that $f(x)>0$ for all $x\in R\cap A$ such that $x\neq a$. Moreover, if $f$ is continuous at $a$, then $f(x)>0$ for all $x\in R\cap A$. The analogous statement is true if $\lim_{x\to a}f(x)=L<0$.
    \begin{proof}
        We divide into two cases ($\lim_{x\to a}f(x)=L>0$ and $\lim_{x\to a}f(x)=L<0$).\par\smallskip
        Suppose first that $\lim_{x\to a}f(x)=L>0$. Choose $\epsilon=L$. Then we have by Definition \ref{dfn:11.1} that there exists a $\delta>0$ such that if $x\in A$ and $0<|x-a|<\delta$, then $|f(x)-L|<L$. Let $R=(a-\delta,a+\delta)$. Clearly, $a\in R$. Now let $x$ be an arbitrary element of $R\cap A$ such that $x\neq a$. It follows from the first condition by Definition \ref{dfn:1.6} that $x\in R$ and $x\in A$. Thus, we have by Exercise \ref{exr:8.9} that $|x-a|<\delta$. Additionally, since $x\neq a$, Definition \ref{dfn:3.1} asserts that $x>a$ or $x<a$, i.e., $x-a>0$ or $x-a<0$; either way, Script \ref{sct:8} implies that $0<|x-a|$. To recap, we know that $x\in A$ and $0<|x-a|<\delta$, so we have by the initial implication that $|f(x)-L|<L=\epsilon$. Therefore, by the lemma from Exercise \ref{exr:8.9}, we have $-L<f(x)-L<L$, i.e., $0<f(x)$ (which we obtain by adding $L$ to both sides of the inequality as permitted by Definition \ref{dfn:7.21}).\par
        Moreover, if $f$ is continuous at $a$, then by Theorem \ref{trm:11.5}, $a\notin LP(A)$ or $\lim_{x\to a}f(x)=f(a)$. But by Definition \ref{dfn:11.1}, $a\in LP(A)$, so we have $\lim_{x\to a}f(x)=f(a)$. The first part of this proof guarantees the existence of a region $R$ such that $f(x)>0$ for all $x\in R\cap A$ such that $x\neq a$. The fact that $f(a)=L>0$ takes care of the case where $x=a$.\par\smallskip
        The proof is symmetric in the other case.
    \end{proof}
\end{lemma}

\begin{theorem}\label{trm:11.9}\marginnote{\emph{3/11:}}
    Suppose that $\lim_{x\to a}f(x)=L$ and $\lim_{x\to a}g(x)=M$. Then
    \begin{enumerate}[label={\textup{(}\alph*\textup{)}}]
        \item $\lim_{x\to a}(f+g)(x)=L+M$.
        \item $\lim_{x\to a}(fg)(x)=L\cdot M$.
        \item Suppose that $\lim_{x\to a}f(x)=L\neq 0$. Then $\lim_{x\to a}\frac{1}{f}(x)=\frac{1}{L}$.
    \end{enumerate}
    \begin{proof}[Proof of a]To prove that $\lim_{x\to a}(f+g)(x)=L+M$, Definition \ref{dfn:11.1} tells us that it will suffice to show that for every $\epsilon>0$, there exists a $\delta>0$ such that if $x\in A$ and $0<|x-a|<\delta$, then $|(f+g)(x)-(L+M)|<\epsilon$. Let $\epsilon>0$ be arbitrary. Consider $\frac{\epsilon}{2}$. Since $\lim_{x\to a}f(x)=L$ and $\lim_{x\to a}g(x)=M$, we have by consecutive applications of Definition \ref{dfn:11.1} that there exists a $\delta_1>0$ such that if $x\in A$ and $0<|x-a|<\delta_1$, then $|f(x)-L|<\frac{\epsilon}{2}$; and there exists a $\delta_2>0$ such that if $x\in A$ and $0<|x-a|<\delta_2$, then $|g(x)-M|<\frac{\epsilon}{2}$. Now choose $\delta=\min(\delta_1,\delta_2)$. This makes it so that for any $x\in A$ such that $0<|x-a|<\delta$, we have $|f(x)-L|<\frac{\epsilon}{2}$ and $|g(x)-M|<\frac{\epsilon}{2}$, so
        \begin{align*}
            |(f+g)(x)-(L+M)| &= |f(x)-L+g(x)-M|\\
            &\leq |f(x)-L|+|g(x)-M|\\
            &< \frac{\epsilon}{2}+\frac{\epsilon}{2}\\
            &= \epsilon
        \end{align*}
        as desired.
    \end{proof}
    \begin{proof}[Proof of b]
        To prove that $\lim_{x\to a}(fg)(x)=LM$, Definition \ref{dfn:11.1} tells us that it will suffice to show that for every $\epsilon>0$, there exists a $\delta>0$ such that if $x\in A$ and $0<|x-a|<\delta$, then $|(fg)(x)-LM|<\epsilon$. Let $\epsilon>0$ be arbitrary. Since $\lim_{x\to a}f(x)=L$ and $\lim_{x\to a}g(x)=M$ (and $\min(\frac{\epsilon}{2(|L|+1)},1)$ and $\frac{\epsilon}{2(|M|+1)}$ are both greater than zero), we have by consecutive applications of Definition \ref{dfn:11.1} that there exists a $\delta_1>0$ such that if $x\in A$ and $0<|x-a|<\delta_1$, then $|f(x)-L|<\min(\frac{\epsilon}{2(|L|+1)},1)$\footnote{When we choose a compound $\epsilon$ (i.e., one that makes use of a min expression), it means that $|f(x)-L|<\frac{\epsilon}{2(|L|+1)}$ and $|f(x)-L|<1$. Basically, whichever quantity is smaller is what min evaluates to, and whichever is bigger is still greater than $|f(x)-L|$ by transitivity.}; and there exists a $\delta_2>0$ such that if $x\in A$ and $0<|x-a|<\delta_2$, then $|g(x)-M|<\frac{\epsilon}{2(|M|+1)}$. Now choose $\delta=\min(\delta_1,\delta_2)$. This makes it so that for any $x\in A$ such that $0<|x-a|<\delta$, we have $|f(x)-L|<\min(\frac{\epsilon}{2(|L|+1)},1)$ and $|g(x)-M|<\frac{\epsilon}{2(|M|+1)}$.\par
        Before we get into the body of the proof, we need a couple of preliminary results. By Script \ref{sct:8}, we have $|a|=|a-b+b|\leq|a-b|+|b|$, so $|a|-|b|\leq|a-b|$. Thus, since $|f(x)-L|<1$, we have $|f(x)|-|L|\leq|f(x)-L|<1$, which means that $|f(x)|<1+|L|$. Additionally, we have
        \begin{align*}
            f(x)g(x)-LM &= f(x)g(x)-f(x)M+f(x)M-LM\\
            &= f(x)(g(x)-M)+M(f(x)-L)
        \end{align*}\par
        With these results, we are ready to introduce the main inequality:
        \begin{align*}
            |(fg)(x)-LM| &= |f(x)(g(x)-M)+M(f(x)-L)|\\
            &\leq |f(x)|\cdot|g(x)-M|+|M|\cdot|f(x)-L|\\
            &< (1+|L|)\cdot\frac{\epsilon}{2(|L|+1)}+|M|\cdot\frac{\epsilon}{2(|M|+1)}\\
            &= \frac{\epsilon}{2}\cdot\frac{1+|L|}{1+|L|}+\frac{\epsilon}{2}\cdot\frac{|M|}{|M|+1}\\
            &< \frac{\epsilon}{2}+\frac{\epsilon}{2}\\
            &= \epsilon
        \end{align*}
    \end{proof}
    \begin{proof}[Proof of c]
        To prove that $\lim_{x\to a}\frac{1}{f}(x)=\frac{1}{L}$, Definition \ref{dfn:11.1} tells us that it will suffice  to show that for every $\epsilon>0$, there exists a $\delta>0$ such that if $x\in A$ and $0<|x-a|<\delta$, then $|\frac{1}{f}(x)-\frac{1}{L}|<\epsilon$. Let $\epsilon>0$ be arbitrary. Since $\lim_{x\to a}f(x)=L$ and $\min(\frac{|L|}{2},\frac{\epsilon|L|^2}{2})>0$, we have by Definition \ref{dfn:11.1} that there exists a $\delta_1>0$ such that if $x\in A$ and $0<|x-a|<\delta_1$, then $|f(x)-L|<\min(\frac{|L|}{2},\frac{\epsilon|L|^2}{2})$. Additionally, since $L\neq 0$, Theorem \ref{lem:11.8} asserts that there exists a region $R$ containing $a$ such that for all $x\in(R\cap A)\setminus\{a\}$, $f(x)\neq 0$ (since $L\neq 0$ implies $L>0$ or $L<0$). It follows by Lemma \ref{lem:8.10} that there exists a $\delta_2$ such that $(a-\delta_2,a+\delta_2)\subset R$. Let $\delta=\min(\delta_1,\delta_2)$.\par
        At this point, we know by this definition of $\delta$ that if $x\in A$ and $0<|x-a|<\delta$, then $|f(x)-L|<\min(\frac{|L|}{2},\frac{\epsilon|L|^2}{2})$ and $f(x)\neq 0$. Before we get into the body of the proof, we need one more preliminary result. It follows from the fact that $|f(x)-L|<\frac{|L|}{2}$ that
        \begin{align*}
            |L| &= 2|L|-|L|\\
            &= 2(|L|-|f(x)|+|f(x)|)-|L|\\
            &\leq 2(|L-f(x)|+|f(x)|)-|L|\\
            &= 2(|f(x)-L|+|f(x)|)-|L|\\
            &< 2\left( \frac{|L|}{2}+|f(x)| \right)-|L|\\
            &= |L|+2|f(x)|-|L|\\
            &= 2|f(x)|
        \end{align*}\par
        With these results, we are ready to introduce the main inequality:
        \begin{align*}
            \left| \frac{1}{f}(x)-\frac{1}{L} \right| &= \left| \frac{1}{f(x)}-\frac{1}{L} \right|\\
            &= \left| \frac{L-f(x)}{f(x)\cdot L} \right|\\
            &= \frac{|f(x)-L|}{|f(x)|\cdot|L|}\\
            &< \frac{\epsilon|L|^2}{2|f(x)|\cdot|L|}\\
            &= \frac{\epsilon|L|}{2|f(x)|}\\
            &< \frac{\epsilon|L|}{|L|}\\
            &= \epsilon
        \end{align*}
    \end{proof}
\end{theorem}

\begin{corollary}\label{cly:11.10}
    If $f$ and $g$ are continuous at $a$, then $f+g$ and $fg$ are continuous at $a$. Also, $\frac{1}{f}$ and $\frac{g}{f}$ are continuous at $a$, provided that $f(a)\neq 0$.
    \begin{proof}
        Since $f$ is continuous at $a$, we have by Theorem \ref{trm:11.5} that either $a\notin LP(A)$ or $\lim_{x\to a}f(x)=f(a)$. Similarly, we have that either $a\notin LP(A)$ or $\lim_{x\to a}g(x)=g(a)$. We divide into four cases ($a\notin LP(A)$ and $a\notin LP(A)$, $a\notin LP(A)$ and $\lim_{x\to a}g(x)=g(a)$, $\lim_{x\to a}f(x)=f(a)$ and $a\notin LP(A)$, and $\lim_{x\to a}f(x)=f(a)$ and $\lim_{x\to a}g(x)=g(a)$). If $a\notin LP(A)$ (this takes care of the first three cases), then by Theorem \ref{trm:11.5}, $f+g$ is continuous at $a$. If $\lim_{x\to a}f(x)=f(a)$ and $\lim_{x\to a}g(x)=g(a)$, then by Theorem \ref{trm:11.9}, $\lim_{x\to a}(f+g)(x)=f(a)+g(a)=(f+g)(a)$. Therefore, by Theorem \ref{trm:11.5}, $f+g$ is continuous at $a$.\par
        The proofs of the second and third cases are symmetric to that of the first. The fourth case can be handled by letting $\frac{g}{f}=g\cdot\frac{1}{f}$ and applying the third and second cases.
    \end{proof}
\end{corollary}

\begin{definition}\label{dfn:11.11}
    A \textbf{polynomial} (in one variable with real coefficients) is a function $f$ of the form
    \begin{equation*}
        f(x) = a_nx^n+a_{n-1}x^{n-1}+\cdots+a_1x+a_0
    \end{equation*}
    for some $n\in\N\cup\{0\}$, where $a_i\in\R$ for $0\leq i\leq n$. A \textbf{rational function} (in one variable with real coefficients) is a function of the form $h(x)=\frac{f(x)}{g(x)}$ where $f$ and $g$ are polynomials in one variable with real coefficients.
\end{definition}

\begin{corollary}\label{cly:11.12}
    Polynomials in one variable with real coefficients are continuous. A rational function in one variable with real coefficients $h(x)=\frac{f(x)}{g(x)}$ is continuous at all $a\in\R$ where $g(a)\neq 0$.
    \begin{proof}
        We induct on the degree $n$ of the polynomial. For the base case $n=0$, we have by Definition \ref{dfn:11.11} that $f$ is a function of the form $f(x)=a_0$ for some $a_0\in\R$. Thus, by Exercise \ref{exr:11.6}, $f$ is continuous at every $x\in\R$ (since $ax+b$ is continuous for $a=0$ and $b=a_0$). Therefore, by Theorem \ref{trm:9.10}, $f$ is continuous. Now suppose inductively that all functions of the form $f(x)=a_nx^n+\cdots+a_0$ are continuous; we seek to prove that all functions of the form $g(x)=a_{n+1}x^{n+1}+\cdots+a_0$ are continuous. By the inductive hypothesis, we know that $x^n$ is continuous. Thus, by Theorem \ref{trm:9.10}, $x^n$ is continuous at every $x\in\R$. Additionally, by Exercise \ref{exr:11.6}, $a_{n+1}x$ (where $a_{n+1}$ is an arbitrary element of $\R$) is continuous at every $x\in\R$. The last two results combined with Corollary \ref{cly:11.10} tell us that $a_{n+1}x\cdot x^n=a_{n+1}x^{n+1}$ is continuous at every $x\in\R$. In addition, since $f$ is continuous, Theorem \ref{trm:9.10} asserts that $f$ is continuous at every $x\in\R$. Thus, by Corollary \ref{cly:11.10} again, $g(x)=a_{n+1}x^{n+1}+f(x)=a_{n+1}x^{n+1}+\cdots+a_0$ is continuous at every $x\in\R$. By one more application of Theorem \ref{trm:9.10}, $g(x)$ is continuous.\par
        Now we move on to proving that $h(x)=\frac{f(x)}{g(x)}$ is continuous at all $a\in\R$ such that $g(a)\neq 0$. By the above, $f$ and $g$ (as polynomials) are continuous. Thus, by Theorem \ref{trm:9.10}, $f$ and $g$ are continuous at every $a\in\R$. Therefore, by Corollary \ref{cly:11.10}, $h=\frac{f}{g}$ is continuous at all $a\in\R$, provided that $g(a)\neq 0$, as desired.
    \end{proof}
\end{corollary}

Now we want to look at limits of the composition of functions. We assume here (for \ref{trm:11.13}-\ref{cly:11.15}) that $a\in A$, $g:A\to\R$, and $f:I\to\R$, where $I$ is an open interval containing $\overline{g(A)}$. It is not quite true in general that if $\lim_{x\to a}g(x)=M$ and $\lim_{y\to M}f(y)=L$, then $\lim_{x\to a}f(g(x))=L$, but it is true in some cases.

\begin{theorem}\label{trm:11.13}
    If $\lim_{x\to a}g(x)=M$ and $f$ is continuous at $M$, then $\lim_{x\to a}f(g(x))=f(M)$.
    \begin{proof}
        To prove that $\lim_{x\to a}f(g(x))=f(M)$, Definition \ref{dfn:11.1} tells us that it will suffice to show that for all $\epsilon>0$, there exists a $\delta>0$ such that if $x\in A$ and $0<|x-a|<\delta$, then $|f(g(x))-f(M)|<\epsilon$. Let $\epsilon>0$ be arbitrary. Then since $f$ is continuous at $M$, Theorem \ref{trm:11.5} implies that there exists a $\delta'>0$ such that if $y\in I$ and $|y-M|<\delta'$, then $|f(y)-f(M)|<\epsilon$. It follows by Definition \ref{dfn:11.1} that there exists a $\delta>0$ such that if $x\in A$ and $0<|x-a|<\delta$, then $|g(x)-M|<\delta'$.\par
        We will now prove that this $\delta$ is the desired $\delta$. Let $x$ be an arbitrary element of $A$ that satisfies $0<|x-a|<\delta$. Then we know that $|g(x)-M|<\delta'$. Additionally, it follows from the fact that $x\in A$ by Definition \ref{dfn:1.18} that $g(x)\in g(A)$. Thus, by Definition \ref{dfn:4.4}, $g(x)\in\overline{g(A)}$. Consequently, by Definition \ref{dfn:1.3}, $g(x)\in I$. Indeed, we now know that $g(x)\in I$ and $|g(x)-M|<\delta'$, so we can determine that $|f(g(x))-f(M)|<\epsilon$, as desired.
    \end{proof}
\end{theorem}

\begin{remark}\label{rmk:11.14}\marginnote{3/17:}
    This theorem can also be rewritten as
    \begin{equation*}
        \lim_{x\to a}f(g(x)) = f\left( \lim_{x\to a}g(x) \right)
    \end{equation*}
    which can be remembered as "limits pass through continuous functions."
\end{remark}

\begin{corollary}\label{cly:11.15}
    If $g$ is continuous at $a$ and $f$ is continuous at $g(a)$, then $f\circ g$ is continuous at $a$.
    \begin{proof}
        Since $g$ is continuous at $a$, we have by Theorem \ref{trm:11.5} that either $a\notin LP(A)$ or $\lim_{x\to a}g(x)=g(a)$. We now divide into two cases. If $a\notin LP(A)$, then by Theorem \ref{trm:11.5}, $f\circ g$ is continuous at $a$. If $\lim_{x\to a}g(x)=g(a)$, then this combined with the fact that $f$ is continuous at $g(a)$ implies by Theorem \ref{trm:11.13} that $\lim_{x\to a}f(g(x))=f(g(a))$. It follows by Definition \ref{dfn:1.25} that $\lim_{x\to a}(f\circ g)(x)=(f\circ g)(a)$. Therefore, by Theorem \ref{trm:11.5}, $f\circ g$ is continuous at $a$.
    \end{proof}
\end{corollary}




\end{document}